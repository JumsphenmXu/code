% 下面这句用以支持中文
% !Mode:: "TeX:UTF-8"

%%% Local Variables:
%%% mode: latex
%%% TeX-master: t
%%% End:

%\documentclass[bachelor,nofonts]{thuthesis}
%\documentclass[master]{thuthesis}
\documentclass[master]{thuthesis}
% \documentclass[%
%   bachelor|master|doctor|postdoctor, % mandatory option
%   winfonts|nofonts|adobefonts, % mandatory only for bachelor and Linuxer
%   secret,
%   openany|openright,
%   arialtoc,arialtitle]{thuthesis}
% 当使用 XeLaTeX 编译时,本科生、Linux 用户需要加上 nofonts 选项;
% 当使用 PDFLaTeX 编译时,adobefonts 选项等效于 winfonts 选项(缺省选项)。

% 所有其它可能用到的包都统一放到这里了,可以根据自己的实际添加或者删除。
\usepackage{thutils}
\usepackage{algorithmic}
\usepackage{algorithm}

% \theoremstyle{definition}
% \newtheorem{example}{Example}

\newcommand{\argmax} {\mathop{\rm argmax} \limits}
% \newtheorem{definition}{定义}
% \newtheorem{lemma}{引理}
\renewcommand{\algorithmicrequire} {\textbf{输入:}}
\renewcommand{\algorithmicensure} {\textbf{输出:}}

% 你可以在这里修改配置文件中的定义,导言区可以使用中文。
% \def\myname{薛瑞尼}

\begin{document}

% 定义所有的eps文件在 figures 子目录下
\graphicspath{{figures/}}


%%% 封面部分
\frontmatter

%%% Local Variables:
%%% mode: latex
%%% TeX-master: t
%%% End:
\secretlevel{绝密} \secretyear{2100}

\ctitle{竞争条件下基于社交网络的影响力最大化算法研究及其实现}
% 根据自己的情况选,不用这样复杂
\makeatletter
\ifthu@bachelor\relax\else
  \ifthu@doctor
    \cdegree{工学博士}
  \else
    \ifthu@master
      \cdegree{工学硕士}
    \fi
  \fi
\fi
\makeatother


\cdepartment[计算机]{计算机科学与技术系}
\cmajor{计算机科学与技术}
\cauthor{许信辉} 
\csupervisor{张勇副研究员}


\etitle{Competitive Influence Maximization over Social Network} 

\edegree{Master of Science} 
\emajor{Computer Science and Technology} 
\eauthor{XU Xinhui} 
\esupervisor{Associate Professor ZHANG Yong} 

\begin{cabstract}
互联网技术的日渐成熟,社交网络的蓬勃发展,使得越来越多的人开始加入互联网发布消息,订阅新闻,转发推文,评论热点问题。大量增长的用户带来了海量有用的数据,而研究分析这些数据可以为我们的决策提供十分有价值线索。影响力分析很早就开始为人们所注意并将其利用于产品推广,广告投放,舆论导向,政策宣传等目的,但是目前主要研究都集中于没有竞争的影响力分析,本文将依据双方针对相似资源的竞争来研究影响力的传播模型及相关的算法。本文主要的工作包括以下几个部分:
\begin{itemize}
\item 提出了一种新的网络结点代价估值模型PRBC(PageRank Based Cost Model)。
\item 扩展了经典的LT模型,提出了XLT4C(eXtended Linear Threshold for Competition Model)传播模型用于存在竞争的影响力分析领域。然后根据XLT4C模型提出了LDH4C(Local Degree Heuristic for Competition),LG4C(Local Greedy for Competition)算法。
\item 利用不同的数据集做了大量的实验对本文提出的估值模型,传播模型,影响力最大化算法进行分析验证。
\end{itemize}
\end{cabstract}

\ckeywords{社交网络,代价估值模型,竞争的影响力传播模型,影响力最大化算法}

\begin{eabstract} 
With Internet technology rapidly growing mature and social networks acquiring their great success, more and more people are now joining into the Internet to publish information, subscribe news, forward  tweets, and comment hot topics. The growing number of netizens bring us massive useful data which can be helpful to help us to make right decisions if we study them carefully. Influence analysis has been noticed for long time that it can be used for products promotion, ads distribution, public opinion transformation, policies advocation et al., but the as far as now, the attention has most been paid on the non-competitive influence analysis, and in this paper we will simulate the situation which two groups compete for their influence spread. The main contributions of this paper can be sumarized as follows:
\begin{itemize}
\item We put forward a new cost evaluation model called PageRank Based Cost Model(PRBC).
\item We extend the classical LT model to come up with eXtended Linear Threshold for Competition(XLT4C) which can be used in the competitive influence maxmization field. And then according to XLT4C, we propose two alogrithms to solve the competitive influence maxmization problem which we named Local Degree Heurisitc for Competition(LDH4C) and Local Greedy for Competition(LG4C).
\item We conducts a lot of experiments on different data sets to analysis the performance of our methods and prove the effectiveness of our methods. 
\end{itemize} 
\end{eabstract}

\ekeywords{Social Network, Cost Evaluation Model, Competitive Influence Spread Model, Influence Maximization Algorithm}

% 设置 PDF 文档的作者、主题等属性
\makeatletter
\thu@setup@pdfinfo
\makeatother
\makecover

% 目录
\tableofcontents

% 符号对照表
\begin{denotation}

\item[OSN] 在线社交网络(Online Social Network)
\item[IC] 独立级联模型(Independent Cascade Model)
\item[LT] 线性阈值模型(Linear Threshold Model)
\item[IM] 影响力最大化(Influence Maximization)
\item[PRBC] 基于PageRank算法的价值函数(PageRank Based Cost Model)
\item[NCC-IM] 非常数价值的影响力最大化(Non-Constant-Cost Influence Maximization)
\item[CF] 代价函数(Cost Function)
\item[BRMDN] 给定预算的最大度邻居算法(Budgeted Random Maximal Degree Neighbor)
\item[XLT4C] 竞争型线性阈值模型拓展(eXtended Linear Threshold for Competition)
% \item[LDH4C] Local Degree Heuristic for Competition
% \item[LG4C] Local Greedy for Competition

\end{denotation}


%%% 正文部分
\mainmatter
% 下面这句用以支持中文
% !Mode:: "TeX:UTF-8"

%%% Local Variables:
%%% mode: latex
%%% TeX-master: t
%%% End:

\chapter{引言}
\label{cha:firstChap01}
\section{研究背景}
互联网快速发展为一种全球化的媒介,人们在互联网里可以超越以往国家的界限去获得信息,分享信息。现如今世界已经由亿万个网页互相连接起来,这亿万的网页充满了各种各样的资源,如web应用,视频站点,图片分享,短文交换站点,博客等等。提供各种服务的在线社会媒体(Online Social Network)也得到了蓬勃的发展,产生了大量的世界级的企业或产品如Facebook,Twitter,WeChat,Snapchat,Instagram,而这些产品都具有以下两个非常明显的特点:
\begin{itemize}
\item 易于使用,用户基数大,便携式设备的发展,使得人们越来越容易参与其中,产生了大量的数据流量,据eBizMBA调查的报告全球最负盛名的Facebook在2016年2月份估计的月访问次数达到了11亿次,而其后的Twitter,Linkedin,Pinterest等都有每月亿级的访问量\footnote{http://www.ebizmba.com/articles/social-networking-websites}。
\item 交互性,用户社会媒体产品可以发布短文、图片、视频,用户之间可以分享、转发、评论等基本社交功能使得各大社会媒体网络上信息交换十分频繁,从而每时每分都在产生着大量的数据。
\end{itemize}


大量的社会媒体平台产生的各种各样的联系以及丰富的数据,很多研究者就开始利用这些资源平台抓取数据开始进行分析研究。人们发现在Facebook,Twitter等社交平台上,消息的传播有着很好的效果,所以加以营销的策略就可以实现特定目的产品、信息、理念在人群中广泛地进行推广\cite{he2012influence}。这种信息的传递的现象被称为信息传播(Influence Diffusion)。随着电子商务的大力发展,商家为了在电子商务平台上出售自己的产品,必须通过投放各种广告来引导消费者到该产品的买卖平台。通过对社会媒体信息的研究和分析,我们可以得知人们在日常生活中所关注的信息,以及这些信息是如何扩散的,并且哪些人能对信息产生更好的传播效应。这样我们就可以有目标地进行广告投放。除此之外,我们还可以利用社会媒体网络去做政策宣传,新理念推广,对不法消息进行辟谣等生活的方方面面。


在信息传播过程中如何使得受众或者接纳该信息的人数最多,此类问题被定义为基于社会媒体网络的影响力最大化,并且该问题在近年来得到了大量研究\cite{he2012influence}\cite{kempe2003maximizing}\cite{chen2011influence}\cite{chen2010scalableKDD}。


\section{研究现状}
%可以分三部分,影响力,最大化,信息传播模型
在分析社会媒体平台影响力最大化时,基于平台的用户关系,可以用图$G(V, E)$来构建整个社会媒体网络,其中定点集合$V$表示OSN(Oline Social Network)上用户或者账号的集合,而边集合$E$表示OSN上用户与用户之间的关系。在考虑影响力传播的过程中,每条边$e \in E$都有其权重,标识用户与用户之间关系的强度或者影响力的大小,而每个顶点$v \in V$则有其价格数值标识选择该顶点所需要付出的代价。影响力最大化则需要在$V$中选择一定的用户数来进行消息传播。


\subsection{顶点估值模型}
在图$G(V, E)$中,顶点估值函数(Cost Function)为$\mathcal{CF}(\cdot)$,那么对于$\forall v \in V$,$v$的价格数值则为$\mathcal{CF}(v)$。目前的研究主要将每个顶点的价值定为常数,最常见的为单位化价格,比如\cite{he2012influence}\cite{kempe2003maximizing}\cite{chen2011influence}\cite{chen2010scalableKDD}\cite{chen2009efficient}将图$G(V, E)$中每个顶点的价格数值都赋值为1,也即$\forall v \in V$,$\mathcal{CF}(v) = 1$。在\cite{leskovec2007cost}中的研究例子1中,J. Leskovec等人用帖子的数量来表示文档的价格;在例子2中,他们只是说明了给每个结点赋了一个非负数值,而并没有详细提到如何给这个值,同样在\cite{han2014balanced}中也没有给出具体的估值函数。然而在\cite{nguyen2013budgeted}中,他们给每个顶点赋一个随机值来表示其价格数值。


\subsection{信息传播模型}
Domingos\cite{domingos2001mining}和Richardson\cite{richardson2002mining}首先利用OSN上用户的联系以马尔科夫场建立模型去进行商品推广。而Kempe\cite{kempe2003maximizing}等人对影响力最大化进行了定义,并证明了得到该问题的最优解是NP难的问题,并在文章中提出了两种信息传播模型,即独立级联模型IC(Independent Cascade)和线性阈值模型LT(Linear Threshold)。这两个模型以及基于IC或者LT改进的模型在后面的研究中被广泛地使用。


\subsection{影响力最大化}
长久以来,最有影响力的点在社会科学领域被广泛地研究\cite{lu2011leaders}\cite{qing2013new}\cite{zhuo2013node}\cite{kitsak2010identification}。社会媒体网络是由用户以及用户与用户之间的关系构成一种图结构,分析网络图用户之间的联系可以为网络营销,政治宣传等活动提供良好的推广策略。基于OSN的影响力最大化,即使采取一种推广策略得在OSN中接受某种信息的用户数量最多。而在一个OSN进行推广的产品、信息、思想等可以具有一个或多个,根据一个平台上传播的信源\footnote{信源指消息传播的来源,在OSN上传播的产品、信息、思想等需要推广的物品。}的多少,我们可以将影响力最大化分为以下两类:
\begin{itemize}
\item 单一信源的影响力最大化,或者无竞争的影响力最大化。即在推广产品信息的过程中,只有一种产品在传播推广,这是目前研究的最多的一个领域。
\item 相对地,多信源的影响力最大化\footnote{一般情况下,多信源指代三种及以上,而在这里也包括两种信源。},或者竞争环境下的影响力最大化。即在推广产品的过程中,有多种功能类似的产品在一个OSN上进行推广,这个相对于单一信源的影响力最大化,需要考虑方面更多,比如IC和LT模型提出时是基于单一信源的,那么研究多信源的影响力最大化,必然需要在此基础上提出新的传播模型或者不基于IC和LT直接提出新的模型,还有在传播的过程中多种产品在某个结点冲突了应该采用哪种策略等等。
\end{itemize}


\subsubsection{无竞争的影响力最大化}
无竞争环境下的影响力最大化一直以来是信息传播领域的一个研究热点。在Kempe\cite{kempe2003maximizing}等人对影响力最大化进行了定义,并得到了一些初步结果之后,大量研究开始出现,旨在解决一方面需要影响传播的效果能得到保证,另一方面则需要尽可能提高算法的效率。正如\cite{kempe2003maximizing}中所得出的结果,得到完全最优解是不能在多项式时间内得到的。那么很多人将注意力集中在改进Kempe提出的贪心算法,这方面做的比较好的有J. Leskovec\cite{leskovec2007cost}发掘并利用了影响力最大化函数具有的子模(Submodularity)特性而提出的CELF,他能在效率上比Kempe提出的Greedy算法快700多倍,后面还有继续在J. Leskovec的研究上继续改进的CELF++\cite{goyal2011celf++}, UBLF\cite{zhou2013ublf},还有Chen\cite{chen2009efficient}等人提出的NewGreedy算法。还有很多研究者提出了很多启发式的算法来提高效率,如随机性的算法Random\cite{kempe2003maximizing}\cite{chen2009efficient}。中心度在社交网络中也是一个研究热点,OSN中若是一个顶点有很多其他顶点连向他,那么一定程度上说明他的重要性,也即很可能是比较有影响力的人,由此Bonacich\cite{bonacich1972factoring}等人就提出了根据顶点度数选取网络图中度数排名前$k$的顶点的启发式算法,Chen\cite{chen2009efficient}等人也根据度数提出了改进的DegreeDiscount,SingleDiscount算法。


\subsubsection{存在竞争的影响力最大化}
Bharathi等人在\cite{bharathi2007competitive}竞争性影响力分析做了初步的分析,并给出了一些结论,贪心算法仍然可以获得很好($(1-\frac{1}{e}) S_{optimal}$)的结果,并将博弈论的知识引入到该问题中进行讨论。Alon\cite{alon2010note}分析了对于网络图在满足什么条件下影响力最大化可以达到纳什均衡(Nash Equilibrium),他给出的结论是当整个网络图$G(V, E)$满足$\mathcal{D}(G) \le 2$时,那么在该网络中进行竞争性影响力最大化过程可以达到纳什均衡状态,其中函数$\mathcal{D}(u, v)$表示图中顶点$u$和顶点$v$的最短距离,$\mathcal{D}(G) \le 2$表示整个图任意两个顶点的距离都小于等于$2$,即$\forall u, v \in V$,$\mathcal{D}(u,v) \le 2$。而Takehara\cite{takehara2012comment}提出了在文献\cite{alon2010note}中的结论条件的不严谨,从而给出了一个更加严格的限制,即在图$G(V, E)$中,$\forall u, v \in V, z \in V\setminus (\mathcal{N}(u)\cup\mathcal{N}(v))$,如果$\mathcal{D}(u,z)=\mathcal{D}(v,z)$,那么竞争条件下的影响力最大化可以达到纳什均衡,其中$\mathcal{N}(\cdot)$为邻居结点集合函数,$\mathcal{N}(u)$为结点$u$的邻居结点集合(包括顶点$u$)。Tzoumas\cite{tzoumas2012game}等人在社交网络中对竞争环境下博弈的分析得出了计算纯纳什均衡(PNE, Pure Nash Equilibrium)策略的复杂度是指数级的。


\section{论文贡献}
本文利用近年来良好的互联网环境所带来的契机,基于快速增长的社会媒体网络,利用社会媒体网络上用户发布的信息,用户与用户之间的联系,发现有影响力的结点,利用这些有影响力的结点我们可以用来完成推广产品,宣传政策,传播新思维等目的。本文对影响力最大化进行了大量的文献调查研究,并分析了各种算法的利弊,然后针对现有算法的不足之处提出了相应解决方案,由此本文的有如下的主要贡献:
\begin{itemize}
\item 本文分别在竞争环境下和非竞争环境中对信息传播进行了研究,并提出相应算法解决现有算法的不足,使得算法在具有良好影响力传播效果的同时,具有较高的执行效率。
\item 本文针对之前对于社会媒体网络中估值函数的问题,提出了基于结点PageRank值的估值模型PRBC(PageRank Based Cost Model),能对结点的价值有更准确和合理的分析。
\item 本文在原有的IC模型之上进行拓展,提出了针对竞争环境下的新的信息传播模型xIC(Extended Independent Cascade)。
\item 本文还分别针对单位价值(Unit Cost Model)和非单位价值(None Unit Cost Model)的顶点估值函数提出了相应的影响力最大化算法。
\item 本文对非竞争环境下的算法进行了大量的实验,实验表明本文提出的算法在时间以及信息的传播效果上有着很好的平衡。另外针对竞争环境下的影响力最大化也进行了大量的实验,结果表明本文提出的算法在信息传播范围有着很好的效果。
\end{itemize}



\section{论文结构}
本文的组织结构为:第一章我们给出了全文的研究影响背景及意义,以及国内外研究状况。第二章描述与本论文相关的一些研究。第三章提出本论文的价格模型PRBC,以及基于PRBC而提出的影响力最大化算法BRMDN。第四章讨论竞争环境下的影响力最大化,并基于IC模型提出一种新的适用于竞争环境下的传播模型xIC,同时给出竞争环境下的算法。而第五章给出竞争环境下影响力最大化的实验结果,并对结果进行分析。最后,第六章中总结论文,并给出了相关研究的展望。

%%% Local Variables: 
%%% mode: latex
%%% TeX-master: t
%%% End: 

\chapter{相关工作}
\label{cha:secondChap02}
\section{复杂网络}
现实世界中有很多的系统,原型,体系都以复杂网络形式存在\cite{boccaletti2006cn},例如生物圈,文献引用网络,OSNs等等。很多大型的在线社会媒体网络(Facebook,Twitter,MySpace,Flickr\cite{mislove2007measurement}) 都具有无标度网络(Scale-free Network)的特性\cite{boccaletti2006cn}。在无标度网络中,网络的结点度数符合幂律分布(Power-law Distribution),也即度数为$k$的结点的可以概率表示为$p(k)=ck^{-\gamma}$\cite{cohen2003scale},并且幂律指数$\gamma \in (2, 3)$。


\section{影响力传播模型}
\label{sec:inf-xtran-models}
网络以图的模型构建之后,需要采取一定的传播模型去对信息进行传播,目前最经典的两个模型为独立级联模型以及线性阈值模型,这两个模型都是由Kempe\cite{kempe2003maximizing}提出的。后面针对竞争环境下的影响力最大化,Xinran He\cite{he2012influence}在研究影响力阻塞传播IBM(Influence Blocking Maximization)的时候基于线性阈值模型提出了竞争性线性阈值模型CLT(Competitive Linear Threshold Model)。Carnes\cite{carnes2007maximizing}在竞争的社会媒体网络中研究影响力最大化时提出了两个传播模型:(a) 基于距离的传播模型(Distance-based Model),(b) 波浪传播模型(Wave Propagation Model)。对于以上传播模型,我们总结如下表\ref{tab:chap2:diffusion-model}:


\begin{table}[htbp]
\centering
\begin{minipage}[t]{0.8\linewidth}
	\caption{传播模型比较}
	\label{tab:chap2:diffusion-model}
	%\begin{tabular}{*{7}{p{.14\textwidth}}}
	\begin{tabular}{*{3}{p{.33\textwidth}}}
		\toprule[1.5pt]
		模型 & 是否适用于竞争环境 & 描述  \\ 
		\midrule[1pt]
		$IC$ & 否 & 详见\ref{sec:IC-model-desc}节 \\
		$LT $ & 否 & 详见\ref{sec:LT-model-desc}节 \\
		$CLT$ & 是 & 详见\ref{sec:CLT-model-desc}节 \\
		$Distant-based$ & 是 & 详见\ref{sec:dist-based-desc}节 \\
		$Wave Propagation$ & 是 & 详见\ref{sec:wave-prop-desc}节 \\
		\bottomrule[1.5pt]
	\end{tabular}
\end{minipage}
\end{table}


\begin{figure}[H]
\centering%
	\subcaptionbox{$t=0$初始状态}
	{\includegraphics[scale=0.66]{./chap2/IC1}}
	\hspace{1mm}%
	\subcaptionbox{$t=1$第一次向邻居传播}
	{\includegraphics[scale=0.66]{./chap2/IC2}}
	\hspace{1mm}%
	\subcaptionbox{$t=2$最终传播状态}
	{\includegraphics[scale=0.66]{./chap2/IC3}}
	\caption{IC传播模型示例}
	\label{fig:IC-inf-diffusion}
\end{figure}


\subsection{独立级联模型}
\label{sec:IC-model-desc}
给定图$G=(V, E)$,以及给定$p(u, v), \forall u, v \in V$表示顶点$u$对顶点$v$的影响权重,还有一个初始选择的结点集合$S$。那么IC模型的传播过程如下:用$S_{t}$表示在传播过程中处于时间$t$时被影响(激活)的结点集合,那么就有$t=0, S_{t}=S$,在$t+1$时刻$\forall w \in S_{t}$,结点$w$依次独立以权重$p(w, z)$去传播给它的邻居结点$z, z \in \mathcal{N}(w) \bigcap (V \setminus \bigcup_{0 \leq i \leq t}S_{i})$,其中$\mathcal{N}(w)$表示$w$的邻居结点集合,而需要注意的是,任何结点只能接受一次激活过程。当某个时刻$t_{end}$,有$S_{t_{end}} = \emptyset$,那么我们称时刻$t_{end}$为终止时刻,而整个影响力传播的结果可以表示为$\bigcup_{0 \leq i \leq t_{end}}S_{i}$,记为$\sigma_{IC}(S)$。


如图\ref{fig:IC-inf-diffusion}所示,其中白色结点表示未被影响过,红色结点表示已经被影响了,灰色结点表示接受过影响传播的过程,但是没有被影响。在初始时刻$t=0,S_{0}=\{A\}$,在时刻$t=1$时结点$A$向其邻居结点$(\mathcal{N}(A)=\{B, C, G, F\})$传播,而此刻结点$C$被影响$(S_{1}=\{C\})$,其他结点$B, G, F$因为已经被激活过一次,则颜色变为灰色表示以后不再接受其他结点的影响。在时刻$t=2$结点$C$影响到了它的邻居结点$D(S_{2}=\{D\})$,而下一步$t=3$则没有结点被影响,即$S_{3}=\emptyset$,整个传播过程终止。最后的结果为$\sigma_{IC}(S)=\bigcup_{0 \leq i \leq 3}S_{i}=\{A, C, D\}$。


\begin{figure}[H]
\centering%
	\subcaptionbox{$t=0$初始状态}
	{\includegraphics[scale=0.63]{./chap2/LT1}}
	\hspace{1mm}%
	\subcaptionbox{$t=1$第一次向邻居传播}
	{\includegraphics[scale=0.63]{./chap2/LT2}}
	\hspace{1mm}%
	\subcaptionbox{$t=2$最终传播状态}
	{\includegraphics[scale=0.63]{./chap2/LT3}}
	\caption{LT传播模型示例}
	\label{fig:LT-inf-diffusion}
\end{figure}


\subsection{线性阈值模型}
\label{sec:LT-model-desc}
给定图$G=(V, E)$,以及给定$\forall u, v \in V, e=(u, v) \in E, p(u, v) \neq 0$表示顶点$u$对顶点$v$的影响权重,而对于$\forall u, v \in V, e=(u, v) \notin E, p(u, v)=0$,并且对于任意结点$w$给定其阈值$\theta_{w}$,还有一个初始选择的结点集合$S$。那么LT模型的传播过程如下:用$S_{t}$表示在传播过程中处于时间$t$时被影响(激活)的结点集合,那么就有$t=0, S_{t}=S$。在时刻$t+1$,对$\forall v \in V\setminus \bigcup_{0 \leq i \leq t}S_{i}$,如果有
\begin{displaymath} 
	{\sum_{u \in \bigcup_{0 \leq i \leq t}S_{i}}p(u,v) \geq \theta_{v}} 
\end{displaymath}
那么结点$v$就被影响了,以上过程不断重复直到某个时刻$t_{end}$,$S_{t_{end}}=\emptyset$,和IC模型一样,LT模型传播过程中每个结点只能接受一次传播过程的影响。最后我们得到的影响传播结果可以表示为$\bigcup_{0 \leq i \leq t_{end}}S_{i}$,记为$\sigma_{LT}(S)$。


如图\ref{fig:LT-inf-diffusion}所示,其中白色结点表示未被影响过,红色结点表示已经被影响了,灰色结点表示接受过影响传播的过程,但是没有被影响。在初始时刻$t=0,S_{0}=\{B\}$,在时刻$t=1$时结点$B$向其邻居结点$(\mathcal{N}(B)=\{A, C, E\})$传播,因为$p(B,A) \geq \theta_{A}, p(B,E) \geq \theta_{E}$,所以结点$A, E$被影响$(S_{1}=\{A, E\})$,其他结点$C$,因为$p(B,C) \leq \theta_{C}$,所以结点$C$没有被影响,而由于$C$已经被传播过一次,所以使其颜色变为灰色表示以后不再接受其他结点的影响。在时刻$t=2$,因为$p(A,D) \geq \theta_{D}$,所以结点$A$影响到了它的邻居结点$D(S_{2}=\{D\})$,同理,$p(A, F) \leq \theta_{F}$,所以$F$颜色也变成灰色。再下一步当$t=3$时则没有结点被影响,即$S_{3}=\emptyset$,整个传播过程终止。最后的结果为$\sigma_{LT}(S)=\bigcup_{0 \leq i \leq 3}S_{i}=\{A, B, D, E\}$。


\begin{figure}[H]
\centering%
	\subcaptionbox{$t=0$初始状态}
	{\includegraphics[scale=0.63]{./chap2/CLT1}}
	\hspace{1mm}%
	\subcaptionbox{$t=1$第一次向邻居传播}
	{\includegraphics[scale=0.63]{./chap2/CLT2}}
	\hspace{1mm}%
	\subcaptionbox{$t=2$最终传播状态}
	{\includegraphics[scale=0.63]{./chap2/CLT3}}
	\caption{CLT传播模型示例}
	\label{fig:CLT-inf-diffusion}
\end{figure}


\subsection{竞争的线性阈值模型}
\label{sec:CLT-model-desc}
Xinran He\cite{he2012influence}在LT的基础上提出了CLT的模型,该模型模拟了两种信源在OSN上的传播,这两种信源的状态标识为$\emph{+activated}$,$\emph{-activated}$。那么对于网络图$G=(V,E)$,对于任意边$e=(u, v) \in E$,都给予两个影响权重$p(u,v)^{+} \in (0, 1)$,$p(u,v)^{-} \in (0, 1)$,同样对于任意边$e=(u,v) \notin E$,那么有$p(u,v)^{+}=0$,$p(u,v)^{-}=0$;而对于图中的任意结点$w \in V$也都给予两个影响阈值$\theta_{w}^{+} \in (0, 1)$,$\theta_{w}^{-} \in (0, 1)$。而对于初始状态有两个集合$P \subset V, N \subset V$,并且$P \bigcap N = \emptyset$,分别标识$\emph{+activated}$,$\emph{-activated}$的信源的初始选择的结点集合。在选择好初始结点集合后,对于任意一个结点$w \in V \setminus (P \bigcup N)$都可以接受两种消息的影响,即每个结点可能成为$+activated$,也可能成为$-activated$,两个集合独立进行传播,其过程类似LT(详见\ref{sec:LT-model-desc})。在时刻$t$的传播情况,下面分三类进行讨论:


\begin{equation}
\label{eq:positive-activated} 
	\begin{aligned}
		{\sum_{u \in \bigcup_{0 \leq i \leq t}P_{i}}p(u,v)^{+} \geq \theta_{v}^{+}}
	\end{aligned}
\end{equation}

\begin{equation} 
\label{eq:negative-activated} 
	\begin{aligned}
		{\sum_{u \in \bigcup_{0 \leq i \leq t}N_{i}}p(u,v)^{-} \geq \theta_{v}^{-}}
	\end{aligned}
\end{equation}


\begin{enumerate}
\item \label{cond:type1} 传播状态满足式子\ref{eq:positive-activated},但是不满足式子\ref{eq:negative-activated},那么此刻结点$v$被标识为$+activated$。
\item \label{cond:type2} 传播状态既不满足式子\ref{eq:positive-activated},也不满足式子\ref{eq:negative-activated},那么此刻结点$v$被标识为已接受过传播但是未被影响,并且以后也不能再被影响。
\item \label{cond:type3} 除上述情况\ref{cond:type1}、情况\ref{cond:type2},则被标识为$-activated$。
注意在结点被标识为$-activated$时有两种可能:
\begin{enumerate}
	\item \label{cond:type3-subtype1} 同时满足式子\ref{eq:positive-activated}和式子\ref{eq:negative-activated}。
	\item \label{cond:type3-subtype2} 不满足式子\ref{eq:positive-activated},但是满足式子\ref{eq:negative-activated}。
\end{enumerate}
在上述情况\ref{cond:type3-subtype1}中将结点标识为$-activated$是基于生活中我们总是更容易被反面消息所影响(negative dominance)而做出的选择。这种选择常被称之为决胜规则(tie-breaking law),解决在竞争环境下某个时刻出现一个结点同时被多种信源影响的情况。
\end{enumerate}


如图\ref{fig:CLT-inf-diffusion}所示,其中白色结点表示未被影响过,红色结点表示被$+activated$所影响,绿色结点表示被$-activated$所影响,灰色结点表示接受过影响传播的过程,但是没有被影响。图中每个顶点由顶点标号(如$A, B, C \in V$),被$+activated$,$-activated$所影响的阈值(如对于顶点$A$,表示为$\{A, +0.12, -0.37\}$三元组),每条边有两个权值(如$e = (B, A) \in E$,有$+0.55$,$-0.38$,分别表示传播$+activated$和$-activated$信源的权值)。其中所有状态中的数值在传播过程中都是正值,数值前面的正负号只表示相对应信源($+activated$,$-activated$)传播权值。下面逐步分析图\ref{fig:CLT-inf-diffusion}的传播过程:
\begin{enumerate}
\item 时刻$t=0$,$P_{0}=\{E\}, N_{0}=\{B\}$
\item 时刻$t=1$,结点$E \in P_{0}$向其邻居结点$(\mathcal{N}(E)=\{A, B, D, F, H\})$传播$+activated$信源,而结点$B \in N_{0}$向其邻居结点$(\mathcal{N}(B)=\{A, G, E, F\})$传播$-activated$信源。
	\begin{enumerate}
	\item 结点$B, E$,初始结点,不再被其他结点影响。
	\item 结点$A$,此时式子\ref{eq:positive-activated}和式子\ref{eq:negative-activated}都成立,所以结点$A$被$-activated$所影响,颜色变为绿色。
	\item 结点$G, D, H$,此时式子\ref{eq:positive-activated}和式子\ref{eq:negative-activated}不成立,故而所以不被影响,颜色变为灰色。
	\item 结点$F$,此时式子\ref{eq:positive-activated}成立,而式子\ref{eq:negative-activated}不成立,所以结点$F$被$+activated$所影响,颜色变为红色。
	\item 所有$P_{0}$,$N_{0}$邻居结点传播完成,所以$P_{1}=\{ F\}$,$N_{1}=\{A\}$。
	\end{enumerate}
\item 时刻$t=2$,$\forall u \in P_{1}$都向其邻居($\mathcal{N}(u)=\{B, E\}$)传播,$\forall v \in N_{1}$都向其邻居($\mathcal{N}(v)=\{B, C, E\}$)传播。
	\begin{enumerate}
	\item 结点$B, E$,都已经被传播过了,所以不再受到传播过程影响。
	\item 结点$C$,此时式子\ref{eq:positive-activated}不成立,但式子\ref{eq:negative-activated}成立,所以被影响为$-activated$,颜色变为绿色。
	\item 所有$P_{1}$,$N_{1}$的邻居结点传播完成,所以$P_{2}=\emptyset$,$N_{2}=\{C\}$。
	\end{enumerate}
\item 时刻$t=3$(图\ref{fig:CLT-inf-diffusion}未有给出,但是其状态和$t=2$状态相同),所有结点都已经接受过一次影响,传播过程终止,即$P_{3}=\emptyset$,$N_{3}=\emptyset$。所以$\sigma_{CLT}(P)=\bigcup_{0 \leq i \leq 3}P_{i}=\{E, F\}$,$\sigma_{CLT}(N)=\bigcup_{0 \leq i \leq 3}N_{i}=\{A, B, C\}$。
\end{enumerate}


\begin{figure}[H]
\centering%
	% \subcaptionbox{$t=0$初始状态}
	{\includegraphics[]{./chap2/Distance-based1}}
	% \hspace{1mm}%
	% \subcaptionbox{$t=1$第一次向邻居传播}
	% {\includegraphics[]{./chap2/Distance-based2}}
	\caption{Distance-based传播模型示例}
	\label{fig:Distance-based-inf-diffusion}
\end{figure}


\subsection{基于距离的传播模型}
\label{sec:dist-based-desc}
Carnes认为在一个网络中结点的位置是和结点的度数一样重要的,就比如生活中个人总是更倾向于模仿身边人的行为\cite{carnes2007maximizing}。在给网络$G=(V,E)$建模时,对于任意边$e=(u,v) \in E$,赋予其一个长度值$d_{uv}$,当没有明确给出边的长度值时,则默认为$d_{uv}=1$,并且给定两个初始结点集合$I_{A}$,$I_{B}$,并令$I=I_{A} \bigcup I_{B}$表示所有的初始选择结点集。定义激活边集$E_{a}$为所有已激活的结点指向未被激活结点的边的集合,那么定义$d_{u}(I, E_{a})$为结点$u$经过激活边集$E_{a}$到已激活结点集$I$的最短距离。如果$u$只允许经过激活边集时,而不能到达$I$,那么$d_{u}(I, E_{a})=\infty$,如果$d_{u}(I, E_{a})<\infty$,定义$v_{u}(I_{A}, d_{u}(I, E_{a}))$和$v_{u}(I_{B}, d_{u}(I, E_{a}))$分别为从$u$通过激活边在距离等于最短距离$d_{u}(I, E_{a})$时到达激活点集$I_{A}$和$I_{B}$的结点数。那么结点$u$被$i \in \{A, B\}$影响的概率可以用如下表达式表示,
\begin{equation}
\label{eq:distant-based-pro}
\begin{aligned} 
	p_{i}(u) = \frac{v_{u}(I_{i}, d_{u}(I, E_{a}))}{v_{u}(I_{A}, d_{u}(I, E_{a})) + v_{u}(I_{B}, d_{u}(I, E_{a}))}
\end{aligned}
\end{equation}
注意上述式子\ref{eq:distant-based-pro}的计算是在激活边集$E_{a}$的范围内的,也即是只有$I$中的结点能影响到$u$,并且是$I$中结点通过边集$E_{a}$达到$u$距离为$d_{u}(I, E_{a})$的结点。那么式子\ref{eq:distant-based-pro}中至少是合法定义的,因为对于$i \in \{A, B\}, v_{u}(I_{i}, d_{u}(I, E_{a}))$至少有一个为正。那么在给定初始结点集$I_{A}$,$I_{B}$时,经过Distance-based模型传播后,被$A$所影响的结果集可以定义为$\rho(I_{A}|I_{B})$,可得如下等式,
\begin{equation}
\label{eq:distant-based-resultA}
\begin{aligned} 
	\rho(I_{A}|I_{B}) = E \left[ \sum_{u \in V} \frac{v_{u}(I_{A}, d_{u}(I, E_{a}))}{v_{u}(I_{A}, d_{u}(I, E_{a})) + v_{u}(I_{B}, d_{u}(I, E_{a}))} \right]
\end{aligned}
\end{equation}
同理可定义$\rho(I_{B}|I_{A})$为,
\begin{equation}
\label{eq:distant-based-resultB}
\begin{aligned} 
	\rho(I_{B}|I_{A}) = E \left[ \sum_{u \in V} \frac{v_{u}(I_{B}, d_{u}(I, E_{a}))}{v_{u}(I_{A}, d_{u}(I, E_{a})) + v_{u}(I_{B}, d_{u}(I, E_{a}))} \right]
\end{aligned}
\end{equation}


在示例图\ref{fig:Distance-based-inf-diffusion}中,蓝色结点标识结点被$A$所影响,红色结点标识结点被$B$影响,所有激活边集$E_{a}$中边的距离都为1,下面逐步对其传播过程进行分析,
\begin{enumerate}
\item 初始状态,$I_{A}=\{x, y\}$,$I_{B}=\{z\}$。
\item 传播过程概率计算
	\begin{enumerate}
	\item 结点$v$,由于$d_{v}(I, E_{a})=1$,而$v_{v}(I_{A}, d_{v}(I, E_{a}))=2$,$v_{v}(I_{B}, d_{v}(I, E_{a}))=0$,根据式子\ref{eq:distant-based-pro},那么$p_{A}(v)=1$,$p_{B}(v)=0$。
	\item 结点$w$,由于$d_{w}(I, E_{a})=1$,而$v_{w}(I_{A}, d_{w}(I, E_{a}))=0$,$v_{w}(I_{B}, d_{w}(I, E_{a}))=1$,根据式子\ref{eq:distant-based-pro},那么$p_{A}(w)=0$,$p_{B}(w)=1$。
	\item 结点$u$,由于$d_{u}(I, E_{a})=2$,而$v_{u}(I_{A}, d_{u}(I, E_{a}))=2$,$v_{u}(I_{B}, d_{u}(I, E_{a}))=1$,根据式子\ref{eq:distant-based-pro},那么$p_{A}(u)=\frac{2}{3}$,$p_{B}(u)=\frac{1}{3}$。
	\end{enumerate}
\item 根据式子\ref{eq:distant-based-resultA}计算,那么可得$\rho(I_{A}|I_{B})=2+1+\frac{2}{3}=\frac{11}{3}$,同理根据式子\ref{eq:distant-based-resultB}计算,那么可得$\rho(I_{A}|I_{B})=1+1+\frac{1}{3}=\frac{7}{3}$。
\end{enumerate}


\begin{figure}[H]
\centering%
	\subcaptionbox{$d=0$初始状态}
	{\includegraphics[scale=0.63]{./chap2/WavePro1}}
	\hspace{1mm}%
	\subcaptionbox{$d=1$向邻居传播}
	{\includegraphics[scale=0.63]{./chap2/WavePro2}}
	\hspace{1mm}%
	\subcaptionbox{$d=2$最终状态}
	{\includegraphics[scale=0.65]{./chap2/WavePro3}}
	\caption{Wave Propagation传播模型示例}
	\label{fig:wavepro-inf-diffusion}
\end{figure}


\subsection{波浪传播模型}
\label{sec:wave-prop-desc}
在\ref{sec:dist-based-desc}节,图\ref{fig:Distance-based-inf-diffusion}中结点$u$虽然只有两个邻居结点(其中一个被$A$以概率1影响,另一个以概率1被$B$影响),但是$u$被$A$影响的概率为$p_{A}(u)=\frac{2}{3}$。这与直观上的观察很难有直接联系,所以Carnes\cite{carnes2007maximizing}提出了波浪传播模型(Wave Propagation Model),为了更好的描述该模型,首先定义任意结点$u$的邻近相关集合$\mathcal{R}(u)$,集合$\mathcal{R}(u)$满足下面两个条件:
\begin{enumerate}
\item $\forall v \in \mathcal{R}(u)$,$v \in \mathcal{N}(u)$。
\item 如果$d_{u}(I, E_{a})=d$,那么$\forall v \in \mathcal{R}(u)$,$d_{v}(I, E_{a})=d-1$。
\end{enumerate}
那么在该模型下,任意没有被影响结点$u$随机选择一个结点$v \in \mathcal{R}(u)$,并接受该选择得到的结点$v$的影响,并且该模型传播从$d=1$开始,然后依次递增距离。注意,在该模型下那么对上图\ref{fig:Distance-based-inf-diffusion}中结点$u$,其受到$A$,$B$的影响的概率都为$\frac{1}{2}$。


定义$P_{A}(u|I_{A}, I_{B}, E_{a})$表示结点$u$被$A$所影响的概率,那么可得,
\begin{equation}
\label{eq:wave-proga-pro}
\begin{aligned}
P_{A}(u|I_{A}, I_{B}, E_{a}) = \frac{\sum_{v \in \mathcal{R}(u)}P_{A}(v|I_{A}, I_{B}, E_{a})}{|\mathcal{R}(u)|}
\end{aligned}
\end{equation}
其中上述等式\ref{eq:wave-proga-pro}的$I_{A}$,$I_{B}$,$E_{a}$的定义如\ref{sec:dist-based-desc}节。并定义最后传播受到$A$影响的结果为$\pi(I_{A}|I_{B})$,那么有,
\begin{equation}
\label{eq:wave-proga-resultA}
\begin{aligned}
\pi(I_{A}|I_{B}) = E \left[ \sum_{u \in V} P_{A}(u|I_{A}, I_{B}, E_{a}) \right]
\end{aligned}
\end{equation}
同理受到$B$影响的结果$\pi(I_{B}|I_{A})$可表示为,
\begin{equation}
\label{eq:wave-proga-resultB}
\begin{aligned}
\pi(I_{B}|I_{A}) = E \left[ \sum_{u \in V} P_{B}(u|I_{A}, I_{B}, E_{a}) \right]
\end{aligned}
\end{equation}


下面逐步对示例图\ref{fig:wavepro-inf-diffusion}进行分析,被$A$所影响的着色为蓝色,被$B$所影响的则着色为红色。其中\underline{图\ref{fig:wavepro-inf-diffusion}}中结点$u$的着色是因为其是概率性的,并不能确定是$A$还是$B$影响。
\begin{enumerate}
\item 初始时$d=0$,则$I_{A}=\{x, y\}$,$I_{B}=\{z\}$。
\item 下一步进行对$d=1$点的传播,也即对结点$v, w$进行影响。
\begin{enumerate}
\item 结点$v$,因为与结点$v$相邻的两个结点$x \in I_{A}, y \in I_{B}$,那么根据式子\ref{eq:wave-proga-pro},则$P_{A}(v|I_{A}, I_{B}, E_{a})=1$,所以结点$v$被$A$影响,着色为蓝色。
\item 结点$w$,因为与结点$w$相邻的一个结点$z \in I_{B}$,那么根据式子\ref{eq:distant-based-pro},则$P_{B}(w|I_{A}, I_{B}, E_{a})=1$,所以结点$w$被$B$影响,着色为红色。
\end{enumerate}
\item $d=1$的结点已经被传播完毕,所以开始传播$d=2$的结点,即结点$u$。而结点$u$的邻居结点$v \in I_{A}$,$w \in I_{B}$,因此根据式子\ref{eq:distant-based-pro},那么此时$P_{A}(w|I_{A}, I_{B}, E_{a})=P_{B}(w|I_{A}, I_{B}, E_{a})=\frac{1}{2}$。因为以概率$\frac{1}{2}$进行传播,所以我们不能确定会被那个结点着色,所以对其进行特殊的着色(见\underline{图\ref{fig:wavepro-inf-diffusion}}结点$u$)。
\item 综上,根据式子\ref{eq:wave-proga-resultA},则有$\pi(I_{A}|I_{B})=2+1+\frac{1}{2}=\frac{7}{2}$,同理根据式子\ref{eq:wave-proga-resultB},可得$\pi(I_{A}|I_{B})=1+1+\frac{1}{2}=\frac{5}{2}$。
\end{enumerate}



\section{问题定义}
Domingos和Richardson\cite{domingos2001mining}提出了可以利用社会媒体网络中的一些人(有影响力的人)去影响其他人去接受或者购买某种产品的想法,Kempe\cite{kempe2003maximizing}等人在此基础上正式提出了影响力最大化问题,并提出了传播模型以及贪心算法去选择那些用影响力的人,下面给出影响力最大化问题定义。

\begin{definition}
\emph{(无竞争的影响力最大化)}
\label{def:noncompetitive-infmax}
给定图$G=(V, E)$,$V$表示网络图中的结点集合,$E$表示网络结点之间联系的集合,并给定一定的预算$\mathcal{B}$,已知常数$K$,那么在给定的传播模型$M$下,影响力最大化问题就是要找出初始集合$S$,使其满足下面式子,
\begin{displaymath}
\sigma_{M}(S) = \argmax_{|S|=K \wedge S \subseteq V} Inf(S), ~ \sum_{v \in S} \mathcal{CF}(v) \leq \mathcal{B}
\end{displaymath}
\end{definition}

\begin{definition}
\emph{(竞争的影响力最大化)}
\label{def:competitive-infmax}
给定图$G=(V, E)$,$V$表示网络图中的结点集合,$E$表示网络结点之间联系的集合,并给定一定的预算$\mathcal{B}$,已知常数$K$,那么在给定的传播模型$M$下,一个已知的初始结点集合$S$,影响力最大化问题就是要找出初始集合$T$,使其满足下面式子,
\begin{displaymath}
\sigma_{M}(T|S) = \argmax_{|T|=K \wedge T \subseteq (V \setminus S)} Inf(T|S), ~ \sum_{v \in T} \mathcal{CF}(v) \leq \mathcal{B}
\end{displaymath}
\end{definition}

其定义中的$Inf(\boldmath\cdot)$表示影响力传播函数(算法),$\mathcal{CF}(\boldmath\cdot)$结点估值函数。并且本文只考虑两种信源的传播,所以只给出了两种信源竞争的影响力最大化定义\ref{def:competitive-infmax}。若是对多方博弈感兴趣,可以参考\cite{bharathi2007competitive}。


\begin{table}[htbp]
\centering
\begin{minipage}[t]{0.8\linewidth}
	\caption{影响力最大化算法}
	\label{tab:chap2:infmax-alg}
	%\begin{tabular}{*{7}{p{.14\textwidth}}}
	\begin{tabular}{*{3}{p{.33\textwidth}}}
		\toprule[1.5pt]
		算法 & 复杂度 & 描述  \\ 
		\midrule[1pt]
		$GenericGreedy$\cite{he2012influence}\cite{kempe2003maximizing}\cite{carnes2007maximizing} & $O(KRmn\tau_{M})$ & 可推广性好,理论能作用于任何传播模型,传播效果好,但是时间复杂度高,目前已经被用在了章节\ref{sec:inf-xtran-models}中的所有模型\\
		$CELF$\cite{leskovec2007cost}/CELF++\cite{goyal2011celf++} UBLF\cite{zhou2013ublf}/NewGreedy\cite{chen2009efficient} & -- & 传播效果比较好,虽然是$GenericGreedy$的改进,但是计算复杂度还是很高,具体参考各文献\\
		$Random$\cite{kempe2003maximizing}\cite{chen2009efficient}  &  $O(K)$ &  速度较快,可以用于任意模型,但是传播效果极差\\
		$MIA$\cite{chen2010scalableKDD} & -- & 利用结点局部树状结构,对其进行阈值剪枝,提高了算法运行效率,运行时间复杂度参考\cite{chen2010scalableKDD} \\
		$LDAG$\cite{chen2010scalableICDM} & -- & 在LT模型下对结点进行剪枝,构建局部子图,进行计算,效果较好,降低运行时间复杂度,具体复杂度去分析参考\cite{chen2010scalableICDM} \\
		$DH$\cite{hu2015rmdn} & $O(nlog(n))$ &  效果没有$Greedy$好,但是时间较好,也可以用于任意传播模型,目前主要用于$IC/LT$\\
		$SD$\cite{chen2009efficient}/$DD$\cite{chen2009efficient} &  $O(Klog(n) + m)$ &  $DH$的改进版本\\
		$RMDN$\cite{hu2015rmdn} & $O(Klog(n))$ &  在运行时间和传播效率上进行一个折中得到的算法\\
		$CLDAG$\cite{he2012influence} & $O(n\tau_{CLDAG} + Klog(n) + K\tau_{CLDAG}$ &  为了研究消息阻塞(可以认为一种竞争的影响力最大化算法),用在了$CLT$模型里\\
		\bottomrule[1.5pt]
	\end{tabular}
\end{minipage}
\end{table}


\section{影响力最大化算法}
根据问题定义\ref{def:noncompetitive-infmax}和定义\ref{def:competitive-infmax},在不同的模型下,目前已经提出了很多的相关算法去解决不同的问题,表\ref{tab:chap2:infmax-alg}给出其中一些算法的概况。


其中上面表\ref{tab:chap2:infmax-alg}中的$R$为迭代次数,$n$为图结点个数,$m$为图边的数量,$K$为要选择的集合的大小,$\tau_{M}$为在模型$M$中执行的时间,$\tau_{CLDAG}$为构建$LDAG^{+}(v)$,$LDAG^{-}(v)$的需要的总时间\cite{he2012influence}。表\ref{tab:chap2:infmax-alg}中MIA、LDAG复杂度没有给出,是因为其构成比较复杂,由于表格篇幅有限且文中有详细描述,所以只给出引文。而
表\ref{tab:chap2:diffusion-model}中CELF、CELF++、UBLF,NewGreedy的复杂度将在章节\ref{sec:chap2-greedy-alg}中给予描述。


根据传播效果以及选择初始结点的策略不同,本文将现有的所有算法分为如下三类,
\begin{enumerate}
\item 贪心算法,详见\ref{sec:chap2-greedy-alg}节。
\item 随机算法,详见\ref{sec:chap2-random-alg}节。
\item 启发式算法,详见\ref{sec:chap2-heuristic-alg}节。
\end{enumerate}



\subsection{基于贪心选择的算法}
\label{sec:chap2-greedy-alg}
D. Kempe\cite{kempe2003maximizing}在提出IC,LT模型的同时,给出了影响力最大化是NP问题,所以必须其转向其他策略来求解,他们在研究中发现影响力最大化算法具有子模(Submodularity)的性质,那么贪心算法得到的结果可以以常数$1-1/e$接近最优解。
\begin{definition}
\emph{(子模性质)}
对于影响力传播函数$f(\boldmath\cdot)$,给定初始结点集$S$,$T$,且$S \subset T \subset V$,那么对于任意结点$v \in V\setminus T$,则有
\begin{enumerate}
\item 函数$f(\boldmath\cdot) \geq 0$,并且函数$f(\boldmath\cdot)$是非递减函数。
\item $f(S \bigcup \{v\}) -  f(S) \geq f(T \bigcup \{v\}) -  f(T)$。
\end{enumerate}
\end{definition}


由于最优解不可计算,而贪心算法能获得很好的效果,所以D. Kempe在文献\cite{kempe2003maximizing}中给出了通用的贪心算法,之后J. Leskovec发现D. Kempe的算法计算复杂度还是很高,所以他利用计算的局部性原理在文献\cite{leskovec2007cost}中提出了CELF算法,使得算法运行效率提高了700多倍。Goyal\cite{goyal2011celf++}利用数据结构堆(Queue)维护当前选择点的边际效益,之前计算的结果最优质,之前最优质情况下的传播效果值,还有当前选择点边际效益更新时的迭代次数四种信息来优化CELF,提出了CELF++算法,并获得了比CELF高35\%-55\%的效率。Zhou\cite{zhou2013ublf}等人发现了贪心选择算法的上界值(Upper Bound),并利用这个值在迭代中改善CELF算法, 提出了UBLF算法,获得了比CELF更好的效率。Chen\cite{chen2009efficient}在对传播过程进行分析之后,将传播过程逐步化,每一步过程中,删除一些不可能被影响的边,进行构建一个新的子图$G^{'}$,然后选择一个新的结点加入初始结点集,这样就可以很快计算出增加的影响力数值。据此他们在基于IC模型上提出了NewGreedy算法
。该算法将原先的复杂度$O(KRmn\mathcal{T}_{M})$提升为$O(KRm\mathcal{T}_{M})$。


Greedy算法由于传播效果比较好,所以后面很多工作在提高算法的运行效率的同时都需用它作为比较的基准,防止由于关注减少时间复杂度而造成传播效果的急剧损失。并且上面提到的这些算法都在IC、LT模型下使用,CLDAG算法开始将贪心选择策略运用到了竞争环境下的影响力最大化问题中。


\subsection{基于随机选择的算法}
\label{sec:chap2-random-alg}
随机选择算法在研究工作中常作为比较的对象,如在\cite{kempe2003maximizing}\cite{chen2009efficient}\cite{hu2015rmdn}都进行了比较。随机选择算法就是在给定图$G=(V, E)$,初始结点集的大小$K$,还有预算$\mathcal{B}$,随机选择$K$个结点$v_{i} \in V, i \in \{0, 1, 2 \dots K-1\}$,使得其满足
\begin{displaymath}
\sum_{0 \leq i \leq K-1}^{i} \mathcal{CF}(v_{i}) \leq \mathcal{B}
\end{displaymath}
这样的算法时间复杂度极低,但是选择的初始结点的影响力传播效果是不确定的,一般情况下十分不理想。不同于贪心选择算法,随机算法不依赖于信息传播模型,而只依赖于图中的结点,所以其通用性是最好的。


\subsection{启发式算法}
\label{sec:chap2-heuristic-alg}
“计算机科学的两大基础目标,就是发现可证明其执行效率良好且可得最佳解或次佳解的算法”\footnote{https://zh.wikipedia.org/wiki/\%E5\%90\%AF\%E5\%8F\%91\%E5\%BC\%8F\%E6\%90\%9C\%E7\%B4\%A2}。启发式算法则是分析求解问题的特点,辅之以平时得到的经验法则,在合理的条件下去给出需要求解问题的一个或者多个解,这个解可能是最优的,也有可能不是最优的,并且算法在平时的使用过程中,常常不可以被确定地证明解的好坏。通常需要采用启发式算法解决的问题一般在确定性算法下是不可多项式时间计算的,也即NP-Hard问题。


影响力最大化问题求最优解已经在文献\cite{kempe2003maximizing}中被证明是NP-Hard的问题了,所以很多人提出了Greedy算法(见章节\ref{sec:chap2-greedy-alg})去获得渐进最优解。同时很多研究\cite{chen2010scalableKDD}\cite{chen2010scalableICDM}\cite{chen2009efficient}\cite{hu2015rmdn}另辟蹊径采用启发式算法。由于影响力最大化问题以图建模,故而目前很多启发式算法都是根据图中结点的度中心性\cite{bonacich1972factoring},结点的距离属性\cite{kimura2006tractable}或者对图进行解构、剪枝获得局部解构,甚至引入随机过程\cite{hu2015rmdn},然后进行计算。


根据不同的启发式策略对表\ref{tab:chap2:infmax-alg}中启发式算法分成以下三类,
\begin{enumerate}
\item 以度中心性为启发式原则的算法,如SD\cite{chen2009efficient},DD\cite{chen2009efficient},DH\cite{hu2015rmdn}。
\item 以解构图或者剪枝边得到子图为启发式原则的算法,如$MIA$\cite{chen2010scalableKDD},$LDAG$\cite{chen2010scalableICDM}。
\item 以随机过程加上度中心性为启发式原则的算法,如RMDN\cite{hu2015rmdn}。
\end{enumerate}




\section{本章小结}
本章介绍了复杂网络的相关研究背景工作,特别是随机网络模型、小世界网络模型和无标度网络模型等;并给出复杂网络的基本数学定义及形式化描述。接下来列出了基于中观层次分析的社区划分研究工作,并给出了目前相关研究的总结。最后,我们简单介绍了整个论文所需要的传播模型,包括传染病传播模型、独立级联模型、线性阈值模型及实验分析的实际社会网络数据集。
% 下面这句用以支持中文
% !Mode:: "TeX:UTF-8"

%%% Local Variables:
%%% mode: latex
%%% TeX-master: t
%%% End:

\chapter{有预算限制的影响力最大化算法}
\label{cha:3thChap03}

\section{引言}
基于Web或者移动设备的在线社交媒体网络蓬勃发展,例如国内的微信,博客,微博,国外的包括Facebook,Twitter等大型社交平台。人们每天都花大量的时间在社交网络上,其发布的短文、评论,图片等信息媒体也越容易被传播从而影响到别人。如何发现有很大影响力的人,并且让他们传播的产品,政治宣传,新思维能尽快地扩散到更广泛的人群中去,这种现象被称为影响力传播(Influence Propagation)。目前有很多研究者在做这方面的研究,但是他们的方法存在下面两方面的问题,
\begin{enumerate}
\item 需要全局了解整个网络的拓扑结构进行计算,这在用户越来越多,关系越来越复杂的现代社交网络中也越来越不容易,同时在运行时间以及传播效果上没有进行很好的平衡。
\item 对于选取网络中结点的代价,没有提出很好的顶点估值模型,一般都是赋予任何结点都是同样的值,这并不符合现实生活中不同影响力的人有不同身价的现象。
\end{enumerate}
本节基于以上两点,首先基于PageRank值并融合顶点度中心性性质,提出了PRBC(PageRank Based Cost)顶点估值模型,然后利用复杂网络中无标度网络的特点提出了BRMDN(Budgeted Random Maximal Degree Neighbor)算法。实验结果表明我们的算法能在时间性能和传播效果上达到比较好的平衡。


\section{相关工作介绍}
\subsection{传播模型}
本章节主要基于Kempe\cite{kempe2003maximizing}的IC、LT模型,具体传播过程可以参考\ref{sec:IC-model-desc}节IC模型示例和\ref{sec:LT-model-desc}节LT模型示例。


\subsection{PRBC模型}
\label{sec:chap3:cost-model}
给定社交网络$G=(V, E)$,需要给每个顶点赋予一个价格,根据现实生活中对于影响力比较大的人给予的身价一般越高,也需要给予图中每个顶点相应不同的价值。L. Page\cite{page1999pagerank}等人提出了PageRank算法对网页根据重要程度进行排名,web网页由超链接连接起来,那么将每个网页都看作一个顶点,链接网页的超链接看作边,那么整个互联网就是一个大的网络图。鉴于社交网络也构建为图模型,所以同样可以利用PageRank算法对图中的顶点进行排名,然后进行一定的映射得出顶点的价值。


具体来说,利用PageRank算法计算结点的PageRank值,然后再利用增长因子和调和系数对原有的PageRank值进一步进行调整,考虑到PageRank算法中一个不太重要的点连接到一个非常重要的点也会被认为是重要的,从而提高了其PageRank值,所以再利用结点度数以及周围邻居中最大度数进行适当的缩放。下面对本文的顶点估值模型PRBC(PageRank Based Cost Model)进行形式化定义。
\begin{definition}
\emph{顶点估值模型}
给定社交网络$G=(V, E)$,一个预先定义好的增长因子$\delta$以及调和系数$\lambda$,那么对于任意结点$u \in V$,定义其在$PRBC$模型下价格数值为$PRBC(u)$,可以形式化表示为,

\begin{equation}
\label{eq:prbc}
PRBC(u) = \frac{\lambda(PR(u) + \delta) \mathcal{D}(u)}{\mathcal{D}(v_{max})},v_{max} = \argmax_{v \in \mathcal{N}(u) \cup \{u\}}\mathcal{D}(v)
\end{equation}

其中$PR(u)$为对整个图利用PageRank算法计算得到的结点$u$的PageRank数值,其物理意义代表了结点的重要度,而$\mathcal{D}(\cdot)$表示为结点度数的函数,式子\ref{eq:prbc}中$\mathcal{D}(u), \mathcal{D}(v)$表示结点$u, v$的度数,$\mathcal{N}(\cdot)$为邻居结点集函数,即$\mathcal{N}(u)$表示结点$u$的邻居结点集。
\end{definition}


\subsection{非常值估值函数下的影响力最大化}
\subsubsection{问题定义}
由于价值估值函数要反映现实生活中的情况,故而采取\ref{sec:chap3:cost-model}节的估值模型,这种情况下的影响力最大化问题可以称之为非常值估值函数下的影响力最大化$NCC-IM$(Non-Constant Cost Influence Maximization),这种比单位化估值函数情况下的问题要复杂一些,因为在预算$\mathcal{B}$的约束下,要找到一个集合$S$,使其满足以下式子\ref{eq:chap3:ncc-im-cond}是非常困难的。
\begin{equation}
\label{eq:chap3:ncc-im-cond}
|S|=K, \sum_{v \in S} \mathcal{CF}(v) = \mathcal{B}
\end{equation}


由于在$NCC-IM$问题中满足式子\ref{eq:chap3:ncc-im-cond}比较难,而且容易陷入无限循环中。所以需要采取一些策略对条件\ref{eq:chap3:ncc-im-cond}进行改造或者对式子\ref{eq:chap3:ncc-im-cond}进行松弛,使得算法容易达到终止条件,而不会因为不满足上述式子而陷入无限循环,下面给出$NCC-IM$问题改造后的定义。


\begin{definition}
\label{def:chap3:ncc-im}
\emph{(NCC-IM)}
给定社交网路图$G=(V, E)$,常数$K$,价值估值函数$\mathcal{CF}(\cdot)$,一定的资源预算$\mathcal{B}$,可接受预算的误差值$\varepsilon$,需要找出一个初始结点集合$S$使其满足如下条件
\begin{displaymath}
\sigma(S) = \argmax_{|S| \leq K \wedge S \subset V} Inf(S), \mathcal{B} - \varepsilon \leq \sum_{v \in S} \mathcal{CF}(v) \leq \mathcal{B} + \varepsilon
\end{displaymath}
其中$\sigma(S)$表示初始集合$S$最终获得影响力的结果,$Inf(\cdot)$是影响力传播函数。
\end{definition}


注意在定义\ref{def:chap3:ncc-im}中将条件松弛为$|S| \leq K$,所以在我们所采用的算法中,都在原来单位化估值函数模型下进行了条件松弛,当满足$\sum_{v \in S}\mathcal{CF}(v) \geq \mathcal{B} - \varepsilon$时,我们会终止初始结点的寻找过程。


\subsubsection{相关算法}
目前的工作主要是在单位化价值估值函数下的影响力最大化,故而对于所有现有的算法,我们都对其进行的相关的调整,在满足原有方法的策略时使其满足定义\ref{def:chap3:ncc-im}中的相关条件限制,主要的算法详细参见下表,

\begin{table}[htbp]
\centering
\begin{minipage}[t]{0.8\linewidth}
	\caption{相关影响力最大化算法}
	\label{tab:chap3:ncc-im-algs}
	\begin{tabular}{*{3}{p{.33\textwidth}}}
		\toprule[1.5pt]
		算法 & 复杂度 & 描述  \\ 
		\midrule[1pt]
		$Greedy$ & $O(KnRm)$ & 详见\ref{sec:chap2-greedy-alg}节 \\
		$Random$ & $O(K)$ & 详见\ref{sec:chap2-random-alg}节 \\
		$DegreeHeuristic$ & $O(nlog(n))$ & 详见\ref{sec:chap2-heuristic-alg}节 \\
		$DegreeDiscount$ & $O(Klog(n) + m)$ & 详见\ref{sec:chap2-heuristic-alg}节 \\
		$SingleDiscount$ & $O(Klog(n) + m)$ & 详见\ref{sec:chap2-heuristic-alg}节 \\
		\bottomrule[1.5pt]
	\end{tabular}
\end{minipage}
\end{table}


注意以上表\ref{tab:chap3:ncc-im-algs}中所有算法都在其原有的基础上进行了改动,使其满足$NCC-IM$的定义。


\section{算法设计与分析}
解决影响力最大化问题是NP-Hard的,所有基于General Greedy算法(见算法\ref{alg:chap3:general-greedy})改进的贪心算法虽然在效率有所提高,但是还是有很高的复杂度。然而拥有极低复杂度的随机算法在效果上没有保证。在文章\cite{barabasi1999emergence}\cite{adamic2000power}\cite{watts1998collective}中,研究者发现很多网络符合无标度网络的特点,也就是大量的网络结点是很稀疏地被连接,而有少部分的结点有则着很稠密的连接关系网。据此,我们无标度网络、小世界网络的特点,设计了我们的算法。

\begin{algorithm}
\caption{贪心算法:计算初始结点集$S$}
\label{alg:chap3:general-greedy}
\begin{algorithmic}
\REQUIRE 图$G=(V,E)$; 初始结点集大小$K$, 算法迭代次数$R$
\ENSURE 得到的初始结点集$S$,并且$S$的元素个数为$K$
\STATE $S \leftarrow \emptyset; i \leftarrow 0;$
\WHILE {$i < K$}
	\FORALL {$v \in V \setminus S$}
		\STATE $s_{v} \leftarrow 0;$
		\FOR {$j = 0 \to R$}
			\STATE $s_{v} \leftarrow s_{v} + \sigma(S \cup \{v\})$
		\ENDFOR
		\STATE $s_{v} \leftarrow s_{v}/R$
	\ENDFOR
	\STATE $S \leftarrow S \cup \argmax_{v \in V \setminus S}(s_{v})$
	\STATE $i \leftarrow i + 1$
\ENDWHILE
\RETURN $S$
\end{algorithmic}
\end{algorithm}


\subsection{BRMDN算法}
受到随机算法以及网络的连接特点,以及新提出的价值估值模型PRBC,我们提出了预算范围内随机最大度邻居算法BRMDN(Budgeted Random Maximal Degree Neighbor)。在算法执行过程中,首先随机选择一个结点,然后根据这个结点的邻居结点集合,对这个局部结合中每个结点按照度数排序,选择满足预算$\mathcal{B}$的限制条件顶点加入初始结点集$S$,直到满足定义\ref{def:chap3:ncc-im}中的限制。


\begin{algorithm}
\caption{MDN:计算结点$u$最大度数的邻居结点}
\label{alg:chap3:mdn-alg}
\begin{algorithmic}
\REQUIRE 图$G=(V,E)$; 结点$u$, 互斥集合$\mathcal{ES}$(Exclusive Set)
\ENSURE 要加入初始结点集的候选结点结点$v_{max}$
\STATE $v_{max} \leftarrow u; v_{degree} \leftarrow \mathcal{D}(u);$
\FORALL {$nbr < \mathcal{N}(u)$}
	\IF {$nbr \notin \mathcal{ES} \wedge v_{degree} < \mathcal{D}(nbr)$}
		\STATE $v_{degree} \leftarrow \mathcal{D}(nbr);$
		\STATE $v_{max} \leftarrow nbr;$
	\ENDIF
\ENDFOR
\RETURN $v_{max}$
\end{algorithmic}
\end{algorithm}


\begin{algorithm}
\caption{BRMDN:计算初始结点集$S$}
\label{alg:chap3:brmnd-alg}
\begin{algorithmic}
\REQUIRE 图$G=(V,E)$; 初始结点集大小$K$, 预算$\mathcal{B}$;预算可接受误差值$\varepsilon$;价值估值函数$\mathcal{CF}(\cdot)$;
\ENSURE 得到的初始结点集$S$;且满足$|S| \leq K \wedge \mathcal{B} - \varepsilon \leq \sum_{v \in S}\mathcal{CF}(v) \leq \mathcal{B} + \varepsilon$;
\STATE $S \leftarrow \emptyset; i \leftarrow 0; totalcost \leftarrow 0;$
\WHILE {$i < K$}
	\STATE 随机选择一个结点 $u \in V \setminus S;$
	\STATE 选择$u_{max} \leftarrow MDN(G, u, S);$
	\IF {$\mathcal{CF}(u_{max}) + totalcost \leq \mathcal{B} + \varepsilon$}
		\STATE $S \leftarrow S \cup \{u_{max}\};$
		\STATE $totalcost \leftarrow \mathcal{CF}(u_{max}) + totalcost;$
		\STATE $i \leftarrow i + 1;$
		\IF {$totalcost \geq \mathcal{B} - \varepsilon$}
			\STATE $ i \leftarrow K + 1;$
		\ENDIF
	\ENDIF
\ENDWHILE
\RETURN $S$
\end{algorithmic}
\end{algorithm}


注意在算法\ref{alg:chap3:brmnd-alg}中利用到算法\ref{alg:chap3:mdn-alg},而在算法\ref{alg:chap3:mdn-alg}中我们可以发现,对于随机选择的一个节点,我们只需要了解其邻居结点的连接状态就可以,而不需要获得整个图的连接状态进行计算,这样就可以很大地降低算法的复杂性,从而提升算法的效率。


\subsection{算法可行性分析}
\label{sec:chap3:feasibility-analysis}
在现实生活中无标度网络是很常见的一种网络状态,所以本文在无标度网络下分析算法\ref{alg:chap3:brmnd-alg}的可行性。在无标度网络中,结点度数为$k$的概率为$p(k)=ck^{-\gamma}$,定义$k_{max}$为网络中结点的最大度数,同理定义$k_{min}$为网络中结点的最小度数,那么根据概率的意义,可以有如下式子\ref{eq:chap3:max-infty}和式子\ref{eq:chap3:min-infty}成立,
\begin{equation}
\label{eq:chap3:max-infty}
\int_{k_{max}}^{+\infty}p(k) dk = \frac{1}{n}
\end{equation}

\begin{equation}
\label{eq:chap3:min-infty}
\int_{k_{min}}^{+\infty}p(k) dk = 1
\end{equation}

求解等式\ref{eq:chap3:max-infty}和等式\ref{eq:chap3:min-infty}可以得到$k_{max} = k_{min}n^{\frac{1}{\gamma-1}}$。对于任意结点$u$,由它连接出去的任意边$e=(u, \cdot) \in E$,那么$u$能以多大的概率连接到一个网络图的中心结点(称为图的hub结点,例如该结点的度数属于整个网络中结点的$Top-K$)。定义$p_{Top-K}$为结点$u$能连接到中心结点的概率,那么可以得到式子\ref{eq:chap3:p-top-k},
\begin{equation}
\label{eq:chap3:p-top-k}
p_{Top-K} = \int_{k_{Top-K}}^{k_{max}}p(k)dk = \frac{k_{max}^{2-\gamma} - k_{Top-K}^{2-\gamma}}{k_{max}^{2-\gamma} - k_{min}^{2-\gamma}}
\end{equation}
如果初始结点结合$S$的大小为$K$,那么算法\ref{alg:chap3:brmnd-alg}至少能获得一个中心结点(hub结点)的概率$p_{hub}$可以表示为
\begin{equation}
\label{eq:chap3:p-hub}
p_{hub} = 1 - (1-p_{Top-K})^{K} - \epsilon
\end{equation}


上面式子\ref{eq:chap3:p-hub}中的$\epsilon$是在预算$\mathcal{B}$控制下的误差,可能使得初始结点的大小没有到$K$从而影响到概率$p_{hub}$。进一步说,假设$\zeta$是图中估值最大的结点的价格数值,也即$\zeta = max \{\mathcal{CF}(u) | u \in V\}$,那么当预算满足条件$\mathcal{B} \geq K\zeta$时,则概率误差$\epsilon = 0$。针对式子\ref{eq:chap3:p-hub}中的参数$K$,如果我们选择的初始集合的大小足够大(例如$K=30$),那么就可以使得$(1-p_{Top-K})^{K} \rightarrow 0$,从而有$p_{hub} \approx 1 - \epsilon$。对于算法BRMDN中的概率误差,我们可以进行多次迭代从而让$\epsilon$进一步降低,也就是使得$p_{hub}$接近于1,这就说明算法能以很大概率选择到比较好的结点,同时不会陷入一种局部最优的情况,即富人俱乐部现象\cite{zhou2004rich},该现象是指结点度数大的结点通常互联在一起,出现度数大的结点扎堆现象。而算法\ref{alg:chap3:brmnd-alg}中的随机选择过程能很好的避免这一情况。


\subsection{时间复杂度分析}
根据算法\ref{alg:chap3:brmnd-alg},在随机选择一个结点后,我们需要遍历其邻居结点从而找到度数最大并且价格数值合适的那个结点,所以我们首先计算图中任意结点的邻居结点数的平均值$\bar{k}$,由于已知无标度网络中的度数与概率的关系,那么可得,
\begin{equation}
\label{eq:chap3:avg-degree-k}
\bar{k} = \sum_{1}^{n} kp(k) = \sum_{1}^{n}kck^{-\gamma} = c\sum_{1}^{n}\frac{1}{k^{\gamma-1}},p(k)=ck^{-\gamma}
\end{equation}


\begin{lemma}
\label{lemma:noexist-gamma-le2}
给定网络图$G=(V, E)$,并且该网络满足无标度网络的性质,那么如果图$G$中没有自环,或者两个顶点之间不存在多条边,那么不存在这样的一个图$G$,使得其满足$1 < \gamma < 2$。
\end{lemma}
\begin{proof}
在章节\ref{sec:chap3:feasibility-analysis}中,我们已经得到$k_{max} = k_{min}n^{\frac{1}{\gamma-1}}$。现在假设存在图$G$满足无标度网络性质,并且有$1 < \gamma < 2$, 那么可以得到$0 < \gamma-1 < 1$,从而$n^{\frac{1}{\gamma-1}} > n$,进一步可以知道$k_{max} = k_{min}n^{\frac{1}{\gamma-1}} > n$,这就意味着一个结点的度数比图中结点的数量还要多,由于引理条件中已知图没有自环,并且两点顶点中不存在多条边,那么可以知道该假设与已知矛盾。证毕。
\end{proof}


根据引理\ref{lemma:noexist-gamma-le2}可知,$\gamma > 2$,那么对于式子\ref{eq:chap3:avg-degree-k},可以作如下变换,
\begin{equation}
\label{eq:chap3:approximate-k}
\bar{k} = c\sum_{1}^{n}\frac{1}{k^{\gamma-1}} \leq c\sum_{1}^{n}\frac{1}{k}=cln(n),n \rightarrow +\infty
\end{equation}


如果初始集的大小表示为$K$,那么算法\ref{alg:chap3:brmnd-alg}的复杂度可以由$K\bar{k}$计算,用式子\ref{eq:chap3:approximate-k}替代$\bar{k}$,可以得到算法的时间复杂度为$O(Klog(n))$。


\section{实验结果与分析}
在现实生活中,考虑到不同的社会媒体网络会有不同的结构,这里我们选择两个不同的数据集,这两个数据集都符合无标度网络的性质,且服从幂律分布,但是拥有不同的$\gamma$值,我们的实验表明算法\ref{alg:chap3:brmnd-alg}在这两个数据集上有着很好的效果,在时间上和传播效果上得到很好的平衡。

\subsection{实验设置}
对两个数据集的性质可以参考如下表格\ref{tab:chap3:datsetTable},

\begin{table}[htbp]
	\centering
	\begin{minipage}[t]{0.8\linewidth}
		\caption{数据集相关参数}
		\label{tab:chap3:datsetTable}
		%\begin{tabular}{*{7}{p{.14\textwidth}}}
		\begin{tabular}{*{6}{p{.16\textwidth}}}
			\toprule[1.5pt]
			Networks & {$n$} & {$m$} & {$\bar{k}$} & {$k_{max}$} & {$\gamma$} \\ 
			\midrule[1pt]
			Blogs & 3982 & 6803 & 3.42 & 189 & 2.453 \\
			Facebook & 4039 & 88234 & 43.69 & 1045 & 2.509 \\
			\bottomrule[1.5pt]
		\end{tabular}
	\end{minipage}
\end{table}


对于表格\ref{tab:chap3:datsetTable}中$n$表示结点个数,$m$表示边的条数,$\bar{k}$表示平均度数,$k_{max}$图中结点的最大度数,两个数据集的具体描述如下,
\begin{itemize}
\item Blogs\cite{hu2013newACN},该数据集拥有4K个结点,6K条边,显然这个数据集构成的图是比较稀疏的,其中$\frac{\#edges}{\#nodes}=1.70$。
\item Facebook\cite{leskovec2012learning},该数据集只是Facebook上部分的数据,拥有4K的结点,但是其变数达到了88K,相对于Blogs数据集来说,这个数据集是连接比较紧密的,其构成的图也是属于比较稠密的图,其中$\frac{\#edges}{\#nodes}=21.84$。
\end{itemize}

为了模拟计算数据集的$\gamma$值我们采用了\cite{hu2015rmdn}当中描述的方法,并得到对表格\ref{tab:chap3:datsetTable}中的数据集进行模型,得到了如下图\ref{fig:blogs-facebook-gamma}的结果,

\begin{figure}[H]
\centering%
	\subcaptionbox{Blogs $\gamma=2.453$}
	{\includegraphics[scale=0.36]{./chap3/Blogs-gamma}}
	\hspace{1mm}%
	\subcaptionbox{Facebook $\gamma$=2.509}
	{\includegraphics[scale=0.36]{./chap3/Facebook-gamma}}
	\caption{数据集模拟$\gamma$值}
	\label{fig:blogs-facebook-gamma}
\end{figure}


我们将IC模型用于BRMDN算法,将BRMDN算法所得到的数据和目前的一些启发式算法进行了对比,我们将要作为对比的算法的描述如下,
\begin{itemize}
\item RandomHeuristic,随机选择$K$个满足条件的结点,速度最快。
\item DegreeHeuristic,这是很基本的一个以度中心性的算法,选取网络图中度数排名前几的满足条件的结点作为初始结点集。
\item SingleDiscount,选择结点之后,对该选择的结点的邻居结点度数减1,是针对DegreeHeuristic算法的改进。
\item DegreeHeuristic,相比于SingleDiscount,该算法在计算要减去的度数的数值时,做到了更精确,从而能更好地模拟选中的结点对于之后结点的影响。
\end{itemize}


贪心算法在理论上能以$1-\frac{1}{e}=0.63$的程度接近最优值,但是其对于大型网络来说,计算时间消耗太长。在我们做实验的过程中,在Blogs数据集上迭代5次(通常情况下我们设置迭代次数为1000次)时,贪心算法\ref{alg:chap3:general-greedy}需要花费20.78个小时才能得出结果,然后同样条件下基于度数的启发式算法只需要0.05秒,这几乎是快了150万倍。所以在本文中我们不把贪心算法作为对比的对象。本文所有实验的环境配置为一台拥有24个Intel(R) Xeon(R) CPU,主频为2.50GHz,内存128个G的服务器,其运行的操作系统为Ubuntu 14.04 LTS。

对于IC模型来说,如果传播概率$p$很大的话,那么对于不同的算法来说影响力的传播相互之间的差值不会很明显,所以本文对于传播概率设置为$p=0.01$。同样为了防止偶然出现的实验偏差,本文对于上面提到的每个算法都进行了$R=1000$次的迭代。同时图中结点的代价函数我们都是利用前面提到的PRBC模型来进行估值,在PRBC模型值的两个参数,我们分别设置为$\lambda=100$,$\delta=0.5$。


\subsection{实验结果及相关分析}
\subsubsection{实验结果}
实验结果表明在IC传播模型下,BRMDN算法能达到和DegreeDiscount相近的传播效果,但是在运行时间上能比DegreeDiscount有更好的表现。具体的实验结果如图\ref{fig:chap3:blogs-seed},图\ref{fig:chap3:facebook-seed},图\ref{fig:chap3:blogs-infl},图\ref{fig:chap3:facebook-infl},图\ref{fig:chap3:running-time}。


\begin{figure}[H]
	\centering%
	\subcaptionbox{Blogs 预算$\mathcal{B}=600$\label{fig:chap3blogs-seed-600}}
	{\includegraphics[scale=0.3]{./chap3/blogs-seed-k600}}
	\hspace{3em}%
	\subcaptionbox{Blogs 预算$\mathcal{B}=900$\label{fig:chap3blogs-seed-900}}
	{\includegraphics[scale=0.3]{./chap3/blogs-seed-k900}}
	\hspace{3em}%
	\subcaptionbox{Blogs 预算$\mathcal{B}=1200$\label{fig:chap3blogs-seed-1200}}
	{\includegraphics[scale=0.3]{./chap3/blogs-seed-k1200}}
	\hspace{3em}%
	\subcaptionbox{Blogs 预算$\mathcal{B}=1500$\label{fig:chap3blogs-seed-1500}}
	{\includegraphics[scale=0.3]{./chap3/blogs-seed-k1500}}
	\caption{Blogs数据集在不同预算下得到的初始结点集大小与预期的结点集大小比较}
	\label{fig:chap3:blogs-seed}
\end{figure}


\begin{figure}[H]
	\centering%
	\subcaptionbox{Facebook 预算$\mathcal{B}=300$\label{fig:chap3facebook-seed-300}}
	{\includegraphics[scale=0.24]{./chap3/facebook-seed-k300}}
	\hspace{3em}%
	\subcaptionbox{Facebook 预算$\mathcal{B}=400$\label{fig:chap3facebook-seed-400}}
	{\includegraphics[scale=0.24]{./chap3/facebook-seed-k400}}
	\hspace{3em}%
	\subcaptionbox{Facebook 预算$\mathcal{B}=500$\label{fig:chap3facebook-seed-500}}
	{\includegraphics[scale=0.24]{./chap3/facebook-seed-k500}}
	\hspace{3em}%
	\subcaptionbox{Facebook 预算$\mathcal{B}=600$\label{fig:chap3facebook-seed-600}}
	{\includegraphics[scale=0.24]{./chap3/facebook-seed-k600}}
	\caption{Facebook数据集在不同预算下得到的初始结点集大小与预期的结点集大小比较}
	\label{fig:chap3:facebook-seed}
\end{figure}


\begin{figure}[H]
	\centering%
	\subcaptionbox{Blogs 预算$\mathcal{B}=600$\label{fig:chap3blogs-infl-600}}
	{\includegraphics[scale=0.24]{./chap3/blogs-infl-600}}
	\hspace{3em}%
	\subcaptionbox{Blogs 预算$\mathcal{B}=900$\label{fig:chap3blogs-infl-900}}
	{\includegraphics[scale=0.24]{./chap3/blogs-infl-900}}
	\hspace{3em}%
	\subcaptionbox{Blogs 预算$\mathcal{B}=1200$\label{fig:chap3blogs-infl-1200}}
	{\includegraphics[scale=0.24]{./chap3/blogs-infl-1200}}
	\hspace{3em}%
	\subcaptionbox{Blogs 预算$\mathcal{B}=1500$\label{fig:chap3blogs-infl-1500}}
	{\includegraphics[scale=0.24]{./chap3/blogs-infl-1500}}
	\caption{Blogs数据集在不同预算下影响力传播效果}
	\label{fig:chap3:blogs-infl}
\end{figure}


\begin{figure}[H]
	\centering%
	\subcaptionbox{Facebook 预算$\mathcal{B}=300$\label{fig:chap3facebook-infl-300}}
	{\includegraphics[scale=0.3]{./chap3/facebook-infl-300}}
	\hspace{3em}%
	\subcaptionbox{Facebook 预算$\mathcal{B}=400$\label{fig:chap3facebook-infl-400}}
	{\includegraphics[scale=0.3]{./chap3/facebook-infl-400}}
	\hspace{3em}%
	\subcaptionbox{Facebook 预算$\mathcal{B}=500$\label{fig:chap3facebook-infl-500}}
	{\includegraphics[scale=0.3]{./chap3/facebook-infl-500}}
	\hspace{3em}%
	\subcaptionbox{Facebook 预算$\mathcal{B}=600$\label{fig:chap3facebook-infl-600}}
	{\includegraphics[scale=0.3]{./chap3/facebook-infl-600}}
	\caption{Facebook数据集在不同预算下影响力传播效果}
	\label{fig:chap3:facebook-infl}
\end{figure}


\begin{figure}[H]
	\centering%
	\subcaptionbox{Blogs\label{fig:chap3blogs-time-k30}}
	{\includegraphics[scale=0.3]{./chap3/blogs-time-k30}}
	\hspace{3em}%
	\subcaptionbox{Facebook \label{fig:chap3facebook-time-k30}}
	{\includegraphics[scale=0.3]{./chap3/facebook-time-k30}}
	\caption{数据集Blogs和Facebook的运行时间}
	\label{fig:chap3:running-time}
\end{figure}

\subsubsection{实验分析}
本节将针对算法的两个方面进行分析,即算法的传播效果以及算法的运行时间。具体分析如下,
\begin{itemize}
\item 影响力传播(Influence Spread),在图\ref{fig:chap3:blogs-infl}和图\ref{fig:chap3:facebook-infl}中我们可以看出DegreeDiscount,DegreeHeuristic,SingleDiscount在传播效果上是最好的,而RandomHeuristic则是最差的,从图\ref{fig:chap3:blogs-infl},图\ref{fig:chap3:facebook-infl}中我们还可以看出BRMDN(算法\ref{alg:chap3:brmnd-alg})在不同的数据集下有着不同的表现,但是都很接近最好的那些算法。我们定义$\nabla_{fig}$为在上面结果图$fig$中算法BRMDN的传播效果与算法DD的传播效果之间的比值
\begin{displaymath}
\nabla_{fig} = 100\% \times \frac{\sigma_{BRMDN}(S)}{\sigma_{DD}(S)}
\end{displaymath}
其具体意义即是算法BRMDN接近最好算法的程度。那么对于图\ref{fig:chap3:blogs-infl}分析可得,$\nabla_{\ref{fig:chap3blogs-infl-600}}=95\%$,$\nabla_{\ref{fig:chap3blogs-infl-900}}=96\%$,$\nabla_{\ref{fig:chap3blogs-infl-1200}}=91\%$,$\nabla_{\ref{fig:chap3blogs-infl-1500}}=85\%$。然而在图\ref{fig:chap3:facebook-infl}中我们可以发现当选择的初始结点的大小$K<15$时,BRMDN算法接近DegreeDiscount算法的程度低于90\%,而当初始结点集大小$K \rightarrow 30$时,算法BRMDN的效果越来越接近DegreeDiscount算法,具体来说,$\nabla_{\ref{fig:chap3facebook-infl-300}}=93\%$,$\nabla_{\ref{fig:chap3facebook-infl-400}}=99\%$,$\nabla_{\ref{fig:chap3facebook-infl-500}}=99\%$,$\nabla_{\ref{fig:chap3facebook-infl-600}}=97\%$。综上图\ref{fig:chap3:blogs-infl},图\ref{fig:chap3:facebook-infl}的结果,算法在影响力传播上有着很好的效果。
\item 运行时间(Running time),从图\ref{fig:chap3:running-time}中可以看出对于任何算法在同一个数据集下,不管预算$\mathcal{B}$值为多少,其运行时间曲线是一天近似的平行于横轴的直线,这意味着运行时间与预算$\mathcal{B}$相关性不大。从另一方面并结合表格\ref{tab:chap3:ncc-im-algs}中算法的时间复杂度,可以知道,RandomHeuristic运行时间最少,BRMDN则略高于RandomHeuristic,DegreeHeuristic排第三,SingleDiscount与DegreeDiscount最慢。从图\ref{fig:chap3blogs-time-k30}可以看出BRMDN比DegreeDiscount大约快了14倍,而从图\ref{fig:chap3facebook-time-k30}可以得到BRMDN比DegreeDiscount大约快了19倍。对于表格\ref{tab:chap3:datsetTable}中的两个数据集来说,他们的结点数$n$比较小,但是根据算法时间复杂度(见表\ref{tab:chap3:ncc-im-algs})来说,如果在结点很多的网络图中,那么BRMDN算法将比DegreeDiscount算法运行时间少更多。
\end{itemize}


注意在图\ref{fig:chap3blogs-infl-900}中我们看到了当初始结点集大小$K>23$时RandomHeuristic的传播效果超过了其他算法。从图\ref{fig:chap3blogs-seed-900}中我们可以发现,当预算$\mathcal{B}$很小时,其他算法找到了价值很高的结点,而RandomHeuristic则选择了较多的小价值结点,从而最后总的传播超过了其他算法。从图\ref{fig:chap3:facebook-infl}中可以发现当预算$\mathcal{B}>400$时,影响力传播效果并没有增加多少,但是从图\ref{fig:chap3facebook-seed-500}中可以得到解释,因为当预算$\mathcal{B}>400$时,在图中的有影响力的结点都已经被选入了初始结点集,而由于我们限制了初始结点集大小$K=30$,所以出现了增加预算而没有增加影响力的传播的现象。


\section{本章小结}
影响力最大化对于产品促销、信息传播、紧急事件疏散等各方面活动都能起着非常重要的作用。在网络中基于常数价格估值函数(单位估值函数),找到$K$个初始传播结点使得在网络中的影响力最大已经被很多人所研究。本文首先提出非常数价格估值函数(PRBC模型)去评估网络中结点的代价,然后基于PRBC模型提出BRMDN算法去进行影响力的传播,并选择了现实生活中的两个常见的数据集进行了大量的实验。通过实验的结果,我们得出下面两个结论:(1)当选定初始结点集$K$的大小时,预算的大小对运行时间的影响不大;(2)BRMDN在达到与DegreeDiscount算法相当的传播效果时,在运行时间上有着一定的优势。综上可知,本文提出的算法能在影响力的传播效果与运行时间上达到很好的平衡。
% 下面这句用以支持中文
% !Mode:: "TeX:UTF-8"

%%% Local Variables:
%%% mode: latex
%%% TeX-master: t
%%% End:

\chapter{竞争型传播模型设计以及影响力最大化算法实现}
\label{cha:4thChap04}
既然利用社会媒体平台可以进行信息宣传,那么产品商就能想到提供给一些有影响力的用户一些自己的产品进行销售,比如在篮球运动鞋上Nike签约了勒布朗,科比为你推广,Adidas签约了哈登,罗斯等人为其推广。这样就产生了竞争,如何在竞争环境下,使得自己的产品得到最广的传播,这就需要一定的策略去选择那些有影响力的用户为产品代言。目前非竞争的影响力分析已经得到了大量的研究,而对于现实生活中更为常见的存在竞争的影响力分析的研究则比较缺乏,本章将针对竞争环境下的影响力分析进行探讨和研究。
% 媒体网络的发展,互联网的普及让更多的人参与到这个世界最大的社区中去。大量的网络用户促使Web服务商,互联网企业开发出了大量的产品,其中提供社会媒体功能的产品就有很多,这些产品每天都有大量的用户去访问,在上面发布文字图片,传导舆论,转发关注信息等等。利用这些平台,以及用户与用户之间的关系,可以进行各种信息的推广。每个人都可以在上面去进行推广自己的产品,所以会造成竞争关系,如何在这样的环境中使自己的消息得到很好的传播是一个很重要的研究问题。

% \section{引言}

% 在社会媒体繁荣发展的今天,社交网络就像是引力极大的黑洞,每天吸引着大量的互联网用户。世界范围内各国用户都有着自己的社交网络偏好,比如在欧美地区Facebook,Twitter,Linkedin等大受欢迎,国内的用户极度依赖微信,微博进行社交活动,俄罗斯用户也有着自己VK网络平台。图\ref{fig:chap4-most-pop-osns}中显示了2016年3月份更新的世界最受欢迎社交网络的排名前5名\footnote{http://www.ebizmba.com/articles/social-networking-websites}。可以看出Facebook,Twitter,Linkedin,Pinterest,Google Plus每个月平均的访问量都达到了亿级以上。每个用户每个月只发布一段文字或者图片信息,那么这个数据量也是十分大的,并且内容也是非常丰富的。良好地利用这些数据以及平台的活跃性,可以在这些平台上着一些具有大量跟随者的人做一些私人的产品推广,可想而知受众基数则是非常大的。或者正对某类产品,可以邀请某些在相关专业领域有着良好口碑的人进行友好评价,也能使得产品得到很好的推广。另一方面,政府有关部门,非盈利组织机构也可以利用这些做宣传。例如利用一些公众人物进行活动的宣传也能得到很好的效果,在图\ref{fig:chap4-lbj-facebook}中NBA球星勒布朗詹姆斯的一个短文,得到了66万用户的关注,5956次的分享转发,7190次的评论。

% \begin{figure}[H]
% 	\centering
% 	\includegraphics[scale=0.6]{./chap4/most-popular-osns}
% 	\caption{最受欢迎的社会媒体网络排行榜(2016年3月)}
% 	\label{fig:chap4-most-pop-osns}
% \end{figure}


% \begin{figure}[H]
% 	\centering
% 	\includegraphics[scale=0.6]{./chap4/lbj-facebook}
% 	\caption{勒布朗詹姆斯在Facebook的一次发文}
% 	\label{fig:chap4-lbj-facebook}
% \end{figure}

% 既然利用社会媒体平台可以进行信息宣传,那么产品商就能想到提供给一些有影响力的用户一些自己的产品进行销售,比如在篮球运动鞋上Nike签约了勒布朗,科比为你推广,Adidas签约了哈登,罗斯等人为其推广。这样就产生了竞争,如何在竞争环境下,使得自己的产品得到最广的传播,这就需要一定的策略去选择那些有影响力的用户为产品代言。

\section{相关工作介绍}
Shishir Bhrathi\cite{bharathi2007competitive}等人在2007年就提出了竞争型影响力最大化问题,作出了一定的研究,给出了一些理论结果。本节我们将阐述本文利用到的几个结论而不进行相应的证明。Tim Carnes\cite{carnes2007maximizing}的工作和本文的工作比较相似,他们基于IC模型提出了两个新的传播模型即Distance-based model和Wave Propagation Model,并将贪心算法运用到了这两个模型之上。Xinran He\cite{he2012influence}等人将竞争型影响力最大化利用到了正面/负面消息的传播中,试图找到一个传播正面消息的集合能最大化地阻塞(Blocking)负面消息的传播。Reiko Takehara\cite{takehara2012comment}等人将竞争型的影响力最大化作为博弈论去研究,试图去发现图满足什么样的条件的情况下能使得博弈能达到纯纳什均衡状态(Pure Nash Equilibria)。Alon\cite{alon2010note}同样适用博弈方法去研究,并对Reiko Takehara先前的结果进行了修正。


\subsection{竞争型影响力最大化问题定义}
\label{sec:chap4:def-for-problem}
竞争型影响力最大化(Competitive Influence Maximization)问题不同于一般的最大化问题主要在于其有多个信源在网络上进行传播,从而使得单一的传播模型不在适用,并且具体的问题也更为复杂,下面先给出Shishir Bhrathi\cite{bharathi2007competitive}等人比较普适的定义,再对其进行细化以适合本文的研究问题。

\begin{definition}
\label{def:chap4-bhrathi-kempe-cim}
\emph{竞争型影响力最大化}
给定网络图$G=(V,E)$,传播模型$M$,顶点的价格函数$\mathcal{CF}(\centerdot)$,以及信源或者玩家数量$b$,还有每个玩家的预算$\mathcal{B}_{i}$,每个玩家都需要在图$G$中选择各自的初始结点集$S_{i}$,使得每个玩家最后传播的影响力最大,并且满足如下式子
\begin{displaymath}
\sum_{u \in S_{i}}\mathcal{CF}(u) \leq \mathcal{B}_{i}, ~i=0,1,2,\cdots,b-1
\end{displaymath}
\end{definition}


下面给出两个Bhrathi\cite{bharathi2007competitive}等人得到的重要结论,
\begin{lemma}
\label{lemma:chap4-functionalities}
假设除了玩家$i$,其他玩家的初始结点集$S_{j}(j \neq i)$已经确定,那么玩家$i$的影响力最大化函数期望$E[|T_{i}|~|S_{0},\cdots ,S_{b-1}]$是关于$S_{i}$的一个单调函数,并且满足子模性质。其中$|T_{i}|$表示在选择初始结点集为$S_{i}$时最后的传播结果。
\end{lemma}

\begin{lemma}
\label{lemma:chap4-opt-approximate}
最后一个玩家$i$选择一个初始结点集$S_{i}$能够高效地获得以$1-\frac{1}{e}$比例接近最优选择得到的$S_{i}^{optimal}$。
\end{lemma}

上述的引理\ref{lemma:chap4-functionalities}和引理\ref{lemma:chap4-opt-approximate}说明最后玩家可以很好地获得比例接近其最优解的结果,本文的研究目标主要是两种信源($b=2$)的传播,所以给定一种信源(比如$A$)的初始结点集合,我们可以找出另一个信源(比如$B$)初始结点集使其能得到很好的传播效果,下面我们定义本文的研究问题,即两种信源竞争的影响力最大化问题。
\begin{definition}
\label{def:chap4-self-on-cim}
给定网络图$G=(V,E)$,传播模型$M$,一种信源的初始结点集$S$,代价函数$\mathcal{CF}(\centerdot)$,另一种信源的预算$\mathcal{B}$,以及其预算误差$\varepsilon$,那么我们需要找出一个初始结点集$T$,是其传播的范围最广,并且满足以下条件,
\begin{displaymath}
\mathcal{B} - \varepsilon \leq \sum_{u \in T}\mathcal{CF}(u) \leq \mathcal{B} + \varepsilon
\end{displaymath}
\end{definition}

\subsection{代价估值函数}
目前已有的研究大部分都选择了单位化的估值函数,然后给定的预算即为初始结点集合的大小,本文的研究将采用非常数的估值函数,具体模型参考章节\ref{sec:chap3:cost-model},本章采用PRBC估值函数。


\subsection{竞争型影响力的传播模型}
在竞争型影响力分析领域目前主要的模型有CLT,Distance-based,Wave Propagation等模型,具体内容可以参考章节\ref{sec:inf-xtran-models}中的描述。其中CLT是基于经典模型LT的,而Distance-based,Wave Propagation是基于经典IC模型的。本章将根据LT模型提出一种更能描述传播过程的一种竞争型影响力传播模型。


\section{模型设计与算法实现}
在章节\ref{sec:inf-xtran-models}描述的两种基于IC模型改进的竞争型传播模型中,Distance-based模型忽略了传播过程中被影响的结点也能影响其他结点的过程,Wave Propagation中为被影响的结点在周围被影响的结点中随机选择一种信源作为被影响的信源,这在现实生活中似乎淡化了个人影响力,而其实这种因素起着非常重要的作用。本文对LT模型进行扩展提出了一种新的传播模型XLT4C(eXtended Linear Threshold for Competition),并基于这个模型提出了相应的算法。


\subsection{XLT4C传播模型}
为了构建模型XLT4C,需要对之前的网络图顶点信息做一些修改,然后设置相应的顶点数据结构构造网络图,之后在该新的网络图中进行传播,下面先给出数据结构设计,然后再给出传播过程。

\subsubsection{数据结构设计}
\label{sec:xlt4c:ds-design}
图$G=(V,E)$由顶点和边构成,下面对顶点和边的数据结构分别进行阐述,相比于LT模型的顶点结构,本模型中的顶点将多增加一些信息,具体增加的信息为结点颜色(color),结点变异信息包括变异因子(mutation factor),变异标识(mutation flag),还有信源访问次数的计数器,由于是两种信源的传播,所以具体包括红色访问计数器(red visit count)以及黑色访问计数器(black visit count),为了便于XLT4C模型的计算,图采用邻接表来存储,所以顶点还需有以该顶点为尾的边的集合(edges)。加上经典IC模型中的阈值(threshold)以及结点本身标识,所以每个结点的可以由一个下面的八元组来标示,
\begin{displaymath}
(labelID, ~color, ~mfactor, ~mflag, ~redcnt, ~blkcnt, ~edges, ~threshold)
\end{displaymath}


图的边信息由三部分构成,源节点(source),目的结点(destination/to),影响权值(weight),即边可以由下面三元组表示,
\begin{displaymath}
(source, ~destination, ~weight)
\end{displaymath}

所以构建图的过程即以顶点的$labelID$为键,映射为该结点的存储信息,即上述的八元组。

\subsubsection{顶点状态及转移过程}
\label{sec:xlt4c:state-xtran}
根据顶点在传播过程的状态变化,可以分为一个初始态,三个过程态,一个完成状态,具体描述如下,
\begin{itemize}
\item Initial state($I$),初始状态结点,还未开始传播。
\item Transient state($T$),过渡状态,比如未被影响的状态要过渡到其他状态,或者已经被影响的状态要变异为其他状态。
\item Adopted state($A$),受到任意信源影响的状态。
\item Mutation state($M$),变异状态,主要是已经被一种信源影响的状态,转变为被另一种信源影响的状态。
\item Done state($D$),完成状态,完成状态是顶点的一种最终状态,并非类似$T, ~A, ~M$四种在传播过程态。
\end{itemize}

\begin{figure}[H]
	\centering%
	\includegraphics[scale=0.4]{./chap4/XLT4C-vertice-status}
	\caption{顶点信息及部分状态表示}
	\label{fig:chap4:vertice-status}
\end{figure}


图\ref{fig:chap4:vertice-status}中给出结点的信息,以及主要的一些状态。下面进一步根据图\ref{fig:chap4:state-xformation}描述一下顶点状态转移过程,具体转移过程如下分析,
\begin{itemize}
\item 状态$I$,那么可以进行以下两种状态转移,
	\begin{enumerate}
	\item $I \rightarrow T$,表示初始结点可以经历某种信源的影响,但是其最后结果是不确定的,具体分析要看状态$T$的转移。
	\item $I$状态保持,表示传播过程不可能传播到该结点,那么该结点可以一直处于初始态。
	\end{enumerate}
\item 状态$T$,该状态可以有以下两种转移过程
	\begin{enumerate}
	\item $T \rightarrow A$,表示被某种信源所影响。
	\item $T$状态保持,此时信源试图去影响该状态,但是顶点并未接受任何信源的影响,只是更改了内部的访问计数器。
	\end{enumerate}
\item 状态$A$,该状态可以有以下三种转移过程,
	\begin{enumerate}
	\item $A \rightarrow D$,表示该结点不再活动,直接转入完成态。
	\item $A \rightarrow T \rightarrow A$,表示该结点可能进行了一次变异,由当前的信源(如接受$X$)变异为另一种信源(如接受$Y$);还有一种可能就是先前是接受$X$,经过状态$T$之后变异失败,那么此时还是处于接受$X$状态。
	\item $A \rightarrow M$,此时表示接受态变异成功时进入变异状态。
	\end{enumerate}
\item 状态$M$,该状态只能向完成态转移,即$M \rightarrow D$。
\item 状态$D$,其为结点的最终状态,不再进行转移。
\end{itemize}

\begin{figure}[H]
	\centering%
	\includegraphics[scale=0.6]{./chap4/XLT4C-state-transformation}
	\caption{顶点状态转移过程}
	\label{fig:chap4:state-xformation}
\end{figure}


\subsubsection{传播模型过程解析}
\label{sec:xlt4c:model-analysis}
本章提出的XLT4C模型是基于LT模型的一些基本传播概念,并在将这些概念移植到了竞争型环境中,所以下面我们简单描述一下该传播模型的整个过程是如何操作的。

首先需要明确如下几个传播规则,
\begin{itemize}
\label{list:rules-for-xtran}
\item 初始结点集中的结点不可变异。
\item 每个结点只被一种信源影响一次。
\item 已经受到影响的结点只能有一次机会向周边的结点继续扩展传播影响。
\item 受到影响的结点可以接受变异的过程,但是一旦经历过一次变异,则此后不再接受变异过程。
\end{itemize}


我们将整个传播流程进行离散化描述,假设存在一个时间变量$t$,那么最开始时有$t=0$,此时两种信源的初始集合为$S_{0}, T_{0}$,同理我们记在$t=i(i>0)$时刻被$S_{i-1}, T_{i-1}$中结点所影响的结点集为$S_{i}, T_{i}$,并且结点$u$的邻居结点集合记为$\mathcal{N}(u)$,并且我们假定集合$S_{i}$的信源为$X$,其访问计数器是$redcnt$,集合$T_{i}$的信源为$Y$,其访问计数器是$blkcnt$,那么整个传播过程的流程如下描述,
\begin{itemize}
\item 时刻$t=0$,初始集合$S_{0}, T_{0}$。
\item 时刻$t=i(i>0)$,对于$\forall s \in S_{i-1}$向其邻居结点$u \in \mathcal{N}(s)$传播,此时根据结点$u$的当前状态,可以出现下面三种情况,
	\begin{enumerate}
	\item 结点$u$处于$I$状态,那么有以下两种情况,
		\begin{enumerate}
		\item $redcnt>0$,此时表示之前已经经历过一次影响过程,那么根据上面的规则,则忽略此次影响;
		\item $redcnt=0$,那么如果此时$u$的邻居$\mathcal{N}(u)$中的已经被$X$影响结点的影响权值之和$weight_{u}^{X}$大于$u$的阈值$threshold_{u}$,那么结点$u$被$X$所影响。
		\end{enumerate}
	\item 结点$u$处于$A$状态,那么也可以分三种情况,
		\begin{enumerate}
		\item 结点$u$已经接受了信源$X$,忽略此次影响过程;
		\item 结点$u$已经接受了信源$Y$并且$redcnt>0$,忽略此次影响过程;
		\item 结点$u$已经接受了信源$Y$并且$redcnt=0$,那么结点$u$以概率$mfactor_{u}$变异为接受信源$X$。一旦接受了变异,就需要在原来的影响集合中删去该结点。
		\end{enumerate}
	\item 结点$u$处于$M,D$状态,那么则忽略此次影响过程。
	\end{enumerate}
	同理对于$\forall t \in T_{i-1}$的过程如上。并且记时刻$t=i$最后新增的影响结果为$S_{i}, T_{i}$。若是某个结点$u$在某个时刻同时可被两种信源影响,那么我们随机接受一种信源的影响,这里我们称之为\textbf{平衡破坏原则}(Rule of Tie-breaking)。
\item 重复上面的过程,知道某个时刻$e$新增的影响结果集均为空,也即$S_{e}=T_{e}=\emptyset$,那么传播过程结束。
\item 最后综合上面每个时刻的传播结果,那么最后信源$X$的传播结果可以表示为$\sigma_{X}=\sum_{i=0}^{e}S_{i}$,同理信源$Y$的结果可以表示为$\sigma_{Y}=\sum_{i=0}^{e}S_{i}$。
\end{itemize}

对于上面的描述我们利用一个示例(图\ref{fig:chap4:xcic-demo})逐步地进行阐述,具体分析如下,
\begin{itemize}
\item 时刻$t=0$,初始结点集分别为$S_{0}=\{C\}$和$T_{0}=\{I\}$。
\item 在第一次过渡态时,结点$\{A, D, E, F, G\}$可能会接受两种信源的影响。
\item 时刻$t=1$,对于上述各个结点分别讨论,
	\begin{enumerate}
	\item 结点$A$,因为$weight_{A}^{X} = 0.37 > threshold_{A}=0.35$,并且$weight_{A}^{Y} = 0.44 > threshold_{A}=0.35$,此时我们利用随机选择的tie-breaking原则,在这里我们选择了接受$X$的影响。
	\item 结点$D$,因为$weight_{D}^{Y} = 0.54 > threshold_{D}=0.23$,所以被$Y$影响。
	\item 结点$E$,因为$weight_{E}^{X} = 0.45 < threshold_{E}=0.62$,所以$E$不被影响,且$redcnt$增加1。
	\item 结点$F$,因为$weight_{F}^{Y} = 0.96 > threshold_{F}=0.71$,并且$weight_{F}^{X} = 0.66 < threshold_{F}=0.71$,所以接受$Y$的影响。
	\item 结点$G$,因为$weight_{G}^{X} = 0.50 > threshold_{G}=0.33$,所以被$X$所影响。
	\end{enumerate}
	此时有$S_{1}=\{A, G\}, T_{1}=\{D, F\}$。
\item 在第二次过渡态是,结点$\{A, B, D, H\}$可能会受到影响,这包括对于$\{A, D\}$来说可能产生的变异。
\item 时刻$t=2$,对于上述各个结点有如下讨论,
	\begin{enumerate}
	\item 结点$A$,由于之前已经接受了信源$X$的影响,而此时可能再次受到结点$D$的影响而产生变异,在示例图\ref{demo-result-1}过程中我们选择了变异失败。
	\item 结点$B$,因为$weight_{B}^{X}=0.12 < threshold_{B}=0.29$,所以不被影响,且$redcnt$增加1。
	\item 结点$D$,由于之前已经接受了信源$Y$的影响,而此时可能再次受到结点$A$的影响而产生变异,在示例过程中我们选择了变异成功,此时标识$mflag_{D}=True$,说明以后不再接受影响或者变异过程了。并且要删除$T_{1}$中的结点$D$,此时$T_{1}$变为$\{F\}$。
	\item 结点$H$,由于$weight_{H}^{Y}=0.13 < threshold_{H}=0.47$,且$weight_{H}^{X}=0.08 < threshold_{H}=0.47$,所以不被影响。
	\end{enumerate}
	此时有$S_{2}=\{D\},T_{2}=\emptyset$。
\item 时刻$t=3$(图中未给出),此时所有结点都已经被影响过一次以上,所以有$S_{3}=T_{3}=\emptyset$,传播过程结束。最终结果有$\sigma_{X}=\sum_{i=0}^{3}S_{i}=\{A, C, D, G\}, \sigma_{Y}=\sum_{i=0}^{3}T_{i}=\{F, I\}$。
\end{itemize}


注意图\ref{fig:chap4:xcic-demo}中最后一张图,即图\ref{demo-result-2},是另一种可能的结果,也即是我们将结点$A$也视为变异成功了。那么此时的结果为$\sigma_{X}=\sum_{i=0}^{3}S_{i}=\{C, D, G\}, \sigma_{Y}=\sum_{i=0}^{3}T_{i}=\{A, F, I\}$。

\begin{figure}[H]
	\centering%
	\subcaptionbox{时刻$t=0$,初始态}
	{\includegraphics[scale=0.5]{./chap4/XLT4C-diffusion0}}
	\hspace{3em}%
	\subcaptionbox{过渡态}
	{\includegraphics[scale=0.5]{./chap4/XLT4C-diffusion1}}
	\hspace{3em}%
	\subcaptionbox{时刻$t=1$,经历第一次传播之后}
	{\includegraphics[scale=0.5]{./chap4/XLT4C-diffusion2}}
	\hspace{3em}%
	\subcaptionbox{过渡态}
	{\includegraphics[scale=0.5]{./chap4/XLT4C-diffusion3}}
	\hspace{3em}%
	\subcaptionbox{\label{demo-result-1} 一种可能的传播完成态}
	{\includegraphics[scale=0.5]{./chap4/XLT4C-diffusion4}}
	\hspace{3em}%
	\subcaptionbox{\label{demo-result-2} 另一种可能的传播完成态}
	{\includegraphics[scale=0.5]{./chap4/XLT4C-diffusion5p}}
	\caption{XLT4C模型传播示例图}
	\label{fig:chap4:xcic-demo}
\end{figure}


\subsubsection{XLT4C模型算法实现}
根据上面章节\ref{sec:xlt4c:ds-design}中描述了数据结构,章节\ref{sec:xlt4c:state-xtran}给出了状态转移过程,而章节\ref{sec:xlt4c:model-analysis}给出了详细的过程分析,下面给出XLT4C模型的算法描述。


\begin{algorithm}[H]
	\caption{$XLT4C-RED-Update(G, S, T, resS, resT)$}
	\label{alg:chap4:inf-spread-red}
	\begin{algorithmic}[1]
	\REQUIRE 图$G=(V,E)$; 结点集$S,T$;传播过程中得到的结果结点集$resS, resT$
	\ENSURE 更新$G, S, T, resS, resT$
		\FOR {$\forall s$ in $S$}
			\STATE 得到$s$的邻居结点集合$\mathcal{N}(s)$;
			\FOR {$\forall u$ in $\mathcal{N}(s)$}
				\IF {结点$u$的$redcnt > 0$ 或者结点$u$的颜色为红色}
					\STATE 将红色访问计数器增加1;
					\STATE continue;
				\ELSIF {结点$u$的颜色为黑色}
					\STATE 以概率$mfactor_{u}$进行变异为红色结点,并设置相应的变异标识$mflag_{u}$;
					\STATE 将结点$u$加入$resS$,并从$T$或者$resT$中将结点$u$删除;
				\ELSE
					\STATE 得到$u$的邻居结点$\mathcal{N}(u)$,记$weight_{u}$为$\mathcal{N}(u)$中结点颜色为红色的结点权值之和,并记$threshold_{u}$为结点$u$的阈值;
					\IF {$weight_{u} > threshold_{u}$}
						\STATE 将结点$u$的颜色记为红色;并将结点$u$加入$resS$;
					\ENDIF
				\ENDIF
				\STATE 将红色访问计数器增加1;
			\ENDFOR
		\ENDFOR
		\RETURN $G, S, T, resS, resT$;
	\end{algorithmic}
\end{algorithm}


\begin{algorithm}[H]
	\caption{$XLT4C-BLACK-Update(G, S, T, resS, resT)$}
	\label{alg:chap4:inf-spread-black}
	\begin{algorithmic}[1]
	\REQUIRE 图$G=(V,E)$; 结点集$S,T$;传播过程中得到的结果结点集$resS, resT$
	\ENSURE 更新$G, S, T, resS, resT$
		\FOR {$\forall t$ in $T$}
			\STATE 得到$t$的邻居结点集合$\mathcal{N}(t)$;
			\FOR {$\forall u$ in $\mathcal{N}(t)$}
				\IF {结点$u$的$blkcnt > 0$ 或者结点$u$的颜色为黑色}
					\STATE 将黑色访问计数器增加1;
					\STATE continue;
				\ELSIF {结点$u$的颜色为红色}
					\STATE 以概率$mfactor_{u}$进行变异为黑色结点,并设置相应的变异标识$mflag_{u}$;
					\STATE 将结点$u$加入$resT$,并从$S$或者$resS$中将结点$u$删除;
				\ELSE
					\STATE 得到$u$的邻居结点$\mathcal{N}(u)$,记$weight_{u}$为$\mathcal{N}(u)$中结点颜色为黑色的结点权值之和,并记$threshold_{u}$为结点$u$的阈值;
					\IF {$weight_{u} > threshold_{u}$}
						\STATE 将结点$u$的颜色记为黑色;并将结点$u$加入$resT$;
					\ENDIF
				\ENDIF
				\STATE 将黑色访问计数器增加1;
			\ENDFOR
		\ENDFOR
		\RETURN $G, S, T, resS, resT$;
	\end{algorithmic}
\end{algorithm}


\begin{algorithm}[H]
	\caption{$XLT4C-Influence-Diffusion(G, S, T, stMutable)$}
	\label{alg:chap4:xlt4c-inf-diffusion}
	\begin{algorithmic}[1]
	\REQUIRE 图$G=(V,E)$; 两个初始结点集$S,T$,而$stMutable$表示节点集合$S,T$中节点能否变异
	\ENSURE 两种初始结点集最后传播的范围$Snum, Tnum$
		\STATE $resS \leftarrow \emptyset, resT \leftarrow \emptyset, Snum \leftarrow 0, Tnum \leftarrow 0$;
		\IF {$stMutable == False$}
			\STATE 将集合$S$中结点颜色标示为红色,集合$T$中结点颜色设置为黑色,且$S,T$中所有结点的变异标示$mflag$设置为$True$;
		\ENDIF
		% \FOR {$\forall s$ in $S$}
		% 	\STATE 得到$s$的邻居结点集合$\mathcal{N}(s)$;
		% 	\FOR {$\forall u$ in $\mathcal{N}(s)$}
		% 		\IF {结点$u$的$redcnt > 0$ 或者结点$u$的颜色为红色}
		% 			\STATE 将红色访问计数器增加1;
		% 			\STATE continue;
		% 		\ELSIF {结点$u$的颜色为黑色}
		% 			\STATE 以概率$mfactor_{u}$进行变异为红色结点,并设置相应的变异标识$mflag_{u}$;
		% 			\STATE 将结点$u$加入$resS$,并从$T$或者$resT$中将结点$u$删除;
		% 		\ELSE
		% 			\STATE 得到$u$的邻居结点$\mathcal{N}(u)$,记$weight_{u}$为$\mathcal{N}(u)$中结点颜色为红色的结点权值之和,并记$threshold_{u}$为结点$u$的阈值;
		% 			\IF {$weight_{u} > threshold_{u}$}
		% 				\STATE 将结点$u$的颜色记为红色;并将结点$u$加入$resS$;
		% 			\ENDIF
		% 		\ENDIF
		% 		\STATE 将红色访问计数器增加1;
		% 	\ENDFOR
		% \ENDFOR

		% \STATE 对$\forall t \in T$同上面的$s \in S$一样处理,其中颜色换为黑色,添加的集合为$resT$;
		\STATE $XLT4C-RED-Update(G, S, T, resS, resT)$;
		\STATE $XLT4C-BLACK-Update(G, S, T, resS, resT)$;

		\IF {$LENGTH(resS) == 0 ~and~ LENGTH(resT) == 0$}
			\STATE $Snum \leftarrow LENGTH(S), Tnum \leftarrow LENGTH(T)$;
		\ELSE
			\STATE $s, t = XLT4C-Influence-Diffusion(G, resS, resT, True)$;
			\STATE $Snum \leftarrow s + LENGTH(S), Tnum \leftarrow t + LENGTH(T)$;
		\ENDIF

		\RETURN $Snum, Tnum$
	\end{algorithmic}
\end{algorithm}


需要注意的是我们在算法\ref{alg:chap4:xlt4c-inf-diffusion}中的第10行进行了递归调用,进行对信息传播的模拟从而使其符合我们的描述。而算法\ref{alg:chap4:xlt4c-inf-diffusion}中第5,6行引用了两个辅助函数分别表示对不同信源信息的传播过程,并且在算法\ref{alg:chap4:inf-spread-red}和算法\ref{alg:chap4:inf-spread-black}中的第9行只有在变异成功(以一定的概率变异成功)的时候才进行操作。

\subsection{竞争型影响力最大算法实现与分析}
根据上述XLT4C的传播模型以及在章节\ref{sec:chap4:def-for-problem}中的定义\ref{def:chap4-self-on-cim},我们需要在已经给定一个初始结点集合(如$S$)的情况下,选择另一个初始结点集合(如$T$)使得$T$中的结点能传播的最广,或者说选择一些结点降低$S$的传播,从而使得$T$有更多可能的受众。在X. He\cite{he2012influence}中他们就是利用积极信息(Positive)去阻塞消极信息(Negative)。在此我们受到其算法的启发,提出下面两种算法,
\begin{itemize}
\item Local Greedy for Competition算法,这个算法的思想是,如果信息需要传播那么其需要依赖其周围的一系列结点去级联传播,那么我们可以在原来集合$S$中的结点周围选择相应比较好的结点使其加入集合$T$作为初始结点集。这里的比较好的衡量准则为利用XLT4C进行传播模拟,哪个结点的传播范围最广则视为最佳候选结点加入。
\item Local Degree Heuristic for Competition算法,该算法不同于Local Greedy for Competition的地方在于选择衡量集合$S$周围比较好的结点标准为选择其度数最大的结点,这也符合度中心性原则\cite{bonacich1972factoring}。
\end{itemize}

在以下章节中我们将算法Local Greedy for Competition和Local Degree Heuristic for Competition分别简称为$LG4C$和$LDH4C$。

\subsubsection{算法实现}
根据算法$LG4C$和$LDH4C$的思路,我们在此给出算法详细过程。
\begin{algorithm}[H]
	\caption{$LG4C(G, S, \mathcal{CF}, \mathcal{B})$}
	\label{alg:chap4:lg4c-proc}
	\begin{algorithmic}[1]
	\REQUIRE 图$G=(V,E)$; 初始结点集$S$, 顶点估值函数$\mathcal{CF}$,预算$\mathcal{B}$, 预算误差$\varepsilon$;
	\ENSURE 初始结点集$T$,使得$T$中的结点的影响在图中能得到最大的传播
		\STATE $T \leftarrow \emptyset, budgetsum \leftarrow 0$;
		\FOR {$\forall s$ in $S$}
			\STATE 得到$s$的邻居结点集合$\mathcal{N}(s)$;
			\STATE $maxinf \leftarrow -\infty,targetnode \leftarrow -1$;
			\FOR {$\forall u$ in $\mathcal{N}(s)$}
				\IF {$u \in S ~or~ u \in T ~or~ \mathcal{CF}(u) + budgetsum > \mathcal{B} + \varepsilon$}
					\STATE $continue$;
				\ENDIF
				\STATE $s_0, t_0 = XLTH4C-Influence-Diffusion(G, S, T, False)$;
				\STATE $s_1, t_1 = XLTH4C-Influence-Diffusion(G, S, T \cup \{u\}, False)$;
				\IF {$maxinf < t_1 - t_0$}
					\STATE $maxinf \leftarrow t_1 - t_0$;
					\STATE $targetnode \leftarrow u$;
				\ENDIF
			\ENDFOR
			\IF {$targetnode == -1$}
				\STATE $continue$;
			\ENDIF
			\STATE $T \leftarrow T \cup \{targetnode\}$;
			\STATE $budgetsum \leftarrow budgetsum + \mathcal{CF}(targetnode)$;
			\IF {$budgetsum > \mathcal{B} - \varepsilon$}
				\STATE $break$;
			\ENDIF
		\ENDFOR
		\RETURN $T$;
	\end{algorithmic}
\end{algorithm}


\begin{algorithm}[H]
	\caption{$LDH4C(G, S, \mathcal{CF}, \mathcal{B})$}
	\label{alg:chap4:ldh4c-proc}
	\begin{algorithmic}[1]
	\REQUIRE 图$G=(V,E)$; 初始结点集$S$, 顶点估值函数$\mathcal{CF}$,预算$\mathcal{B}$, 预算误差$\varepsilon$;
	\ENSURE 初始结点集$T$,使得$T$中的结点的影响在图中能得到最大的传播
		\STATE $T \leftarrow \emptyset, budgetsum \leftarrow 0$;
		\FOR {$\forall s$ in $S$}
			\STATE 得到$s$的邻居结点集合$\mathcal{N}(s)$;
			\STATE $maxdegree \leftarrow -\infty,targetnode \leftarrow -1$;
			\FOR {$\forall u$ in $\mathcal{N}(s)$}
				\IF {$u \in S ~or~ u \in T ~or~ \mathcal{CF}(u) + budgetsum > \mathcal{B} + \varepsilon$}
					\STATE $continue$;
				\ENDIF
				\STATE $d \leftarrow LENGTH(\mathcal{N}(u))$;
				\IF {$maxinf < d$}
					\STATE $maxinf \leftarrow d$;
					\STATE $targetnode \leftarrow u$;
				\ENDIF
			\ENDFOR
			\IF {$targetnode == -1$}
				\STATE $continue$;
			\ENDIF
			\STATE $T \leftarrow T \cup \{targetnode\}$;
			\STATE $budgetsum \leftarrow budgetsum + \mathcal{CF}(targetnode)$;
			\IF {$budgetsum > \mathcal{B} - \varepsilon$}
				\STATE $break$;
			\ENDIF
		\ENDFOR
		\RETURN $T$;
	\end{algorithmic}
\end{algorithm}

\subsubsection{算法复杂性分析}
根据算法$LG4C$(算法\ref{alg:chap4:lg4c-proc})和算法$LDH4C$(算法\ref{alg:chap4:ldh4c-proc})的过程可以看出算法的时间复杂度主要在两次循环选取目标结点的过程中。对于外层循环其主要是依赖于初始结点集$S$的大小,在内层循环中则都依赖集合$S$中结点的邻居结点数。但是具体与内层循环来看,算法$LG4C$和$LDH4C$有所不同,在算法$LG4C$中第9,10行用到了传播模型XLT4C的算法\ref{alg:chap4:xlt4c-inf-diffusion},所以这也是主要的时间开销。在算法$LDH4C$中第9行需要计算某个结点的邻居结点集,然后选取最优结点。我们假设算法\ref{alg:chap4:xlt4c-inf-diffusion}的时间复杂度为$\tau_{xlt4c}$,计算邻居结点集大小的时间为$\tau_{n}$,图中结点的平均度数为$\bar{k}$,集合$S$的大小为$K$,那么算法$LG4C$的运行时间复杂度为$O(\bar{k}K\tau_{xlt4c})$,算法$LDH4C$的运行时间复杂度为$O(\bar{k}K\tau_{n})$。

\section{实验设置与结果分析}
\subsection{数据集}
\label{sec:chap4-exp-datasets}
为了实验的充分性并考虑到社交网络的多样性,我们选择了几个不同的社交网络图,这些图是真实爬取的数据,由于真实的社交网络平台太大,所以所采用的这些数据集中,部分的数据只是真实网络中的一个子集。我们选取的这些数据集之前也被应用在了其他研究成果上,其具体描述如下:
\begin{itemize}
\item USAir97\cite{batagelj2009pajek},美国97年航空交通网络的一部分子图。拥有332个结点和4252条边。
\item Blogs\cite{xie2006social},微博网络图的一部分数据。有3982个结点以及6803条边。
\item BA\_weight,利用BA模型生成的无标度网络,有3000个结点以及8991条边,此数据集的权值(weight)并没有用在本实验中,为了和数据集USAir97和Blogs一致,我们算法中是随机生成的权值。
\end{itemize}

\subsection{环境参数设置}
\label{sec:chap4-exp-setup}
在顶点估值模型PRBC中的两个参数$\lambda,\delta$,我们分别设为$\lambda=100,\delta=0.5$。算法的运行环境为配置了24个Intel(R) Xeon(R)的CPU,主频2.5GHz,内存128GB的运行Ubuntu 14.04LTS的服务器。
对于实验过程中的其中一方(称为$X$)的初始结点,我们始终是以Greedy算法选出的,而另一个方(称为$Y$)选择的初始结点集则需要在最后的传播结果上尽量与$X$相接近。而对于不同数据集的预算$\mathcal{B}$则是根据选取Greedy算法选取的30个结点的代价之和作一定的偏移。为了算法的精确性,我们对于每个算法都进行了$R=1000$次的重复计算,然后选取综合所有结果选取平均值。

\subsection{实验结果}
\label{sec:chap4-exp-results}
由于我们每次都是两种信源之间竞争性地进行传播,并且其中一个信源(如$X$)总是利用贪心算法获得的初始结点集,另一个信源(如$Y$)则根据不同的算法选择初始结点集合,那么单纯地比较传播的数量很难衡量一个算法的好坏,本文利用一以下比例指标$\mathcal{R}_{alg}$来衡量算法的好坏。设某算法$alg$选择的初始结点集传播的数量为$\#alg$,而贪心算法选出的初始结点集传播的数量为$\#greedy$,那么数值$\mathcal{R}_{alg}$的定义如下,
\begin{displaymath}
\mathcal{R}_{alg}=\frac{\#alg}{\#greedy+\#alg}
\end{displaymath}
数值$\mathcal{R}_{alg}$表示算法$alg$能获得整个受传播影响结点中被$Y$影响的比例,而因为在同等条件下信源$X$采用Greedy算法,那么数值$\mathcal{R}_{alg}$一般总是小于$\frac{1}{2}$\footnote{因为Influence Maxmization问题是NP难问题,所以在目前算法中,我们一般认为Greedy算法是可计算算法里最优的(但这点不是绝对的,所以$\mathcal{R}_{alg} > \frac{1}{2}$是可能的,比如下面的图\ref{fig:chap4-usair97-b200})。},当$\mathcal{R}_{alg} \rightarrow \frac{1}{2}$时,竞争的效果越好。在本文的环境下,因为总是假定信源$X$选择了比较好的初始结点集(用Greedy算法选择),然后才用其他算法选择初始结点集,所以要达到比较好的竞争平衡效果,那么$\mathcal{R}_{alg}$越高则表示算法平衡效果越好。

根据章节\ref{sec:chap4-exp-datasets}的实验数据集及章节\ref{sec:chap4-exp-setup}给出的设置条件,我们给出算法的运行结果(见图\ref{fig:chap4-usair97-result},图\ref{fig:chap4-blogs-result},图\ref{fig:chap4-ba-result}),并注意在图中我们的横坐标都在实验数据上缩小了5倍,也即我们实验过程中所采用的初始数据集大小为5,10,15,20,25,30,为了使得结果图更为直观,所以我们对坐标进行了适当的缩小。算法DH4C(Degree Heuristic for Competition)是直接对所有结点排序,排除已经被选中的结点,然后选择剩下度数较高的且代价在预算范围内的结点集,而算法LDH4C是基于DH4C的思路,然后根据对方已经选择的结点进行局部化针对性选择满足条件的结点作为初始结点集。


由于不同算法之间的运行时间在数量级上相差比较大,在图中不好表达,所以我们采用表格的形式进行描述,我们选取了实验中每个数据集最大预算以及选取初始结点集的限制为最大的情况下进行对比(见表\ref{tab:chap4-algs-time})。

\begin{table}[htbp]
	\centering
	\begin{minipage}[t]{0.8\linewidth}
		\caption{算法运行时间(秒)对比}
		\label{tab:chap4-algs-time}
		\begin{tabular}{*{7}{p{.12\textwidth}}}
			\toprule[1.5pt]
			Networks & {预算} & {结点集大小} & {Greedy} & {DH4C} & {LDH4C} & {LG4C} \\ 
			\midrule[1pt]
			USAir97 & 300 & 30 & 2656.336 & 0.00065 & 0.00036 & 42.0492 \\
			Blogs & 600 & 30 & 114851.6 & 0.00537 & 0.00679 & 516.659 \\
			BA\_weight & 400 & 30 & 108265.7 & 0.00493 & 0.00056 & 511.642 \\
			\bottomrule[1.5pt]
		\end{tabular}
	\end{minipage}
\end{table}

\begin{figure}[H]
	\centering%
	\subcaptionbox{\label{fig:chap4-usair97-insufficient}数据集USAir97预算$\mathcal{B}=150$}
	{\includegraphics[width=3.5cm]{./chap4/results/USAir97-insufficient-budget}}
	\hspace{3em}
	\subcaptionbox{\label{fig:chap4-blogs-insufficient}数据集USAir97预算$\mathcal{B}=300$}
	{\includegraphics[width=3.5cm]{./chap4/results/blogs-insufficient-budget}}
	\hspace{3em}
	\subcaptionbox{\label{fig:chap4-ba-insufficient}数据集BA\_weight\\预算$\mathcal{B}=250$}
	{\includegraphics[width=3.5cm]{./chap4/results/BA-insufficient-budget}}
	\caption{预算不足时预期初始结点集大小与实际选择初始结点集大小对比}
	\label{fig:chap4-insufficient-budget}
\end{figure}

\begin{figure}[H]
	\centering%
	\subcaptionbox{\label{fig:chap4-usair97-b150}数据集USAir97,预算限制为$\mathcal{B}=150$}
	{\includegraphics[width=5.5cm]{./chap4/results/USAir97-B150}}
	\hspace{3em}
	\subcaptionbox{\label{fig:chap4-usair97-b200}数据集USAir97,预算限制为$\mathcal{B}=200$}
	{\includegraphics[width=5.5cm]{./chap4/results/USAir97-B200}}
	\hspace{3em}
	\subcaptionbox{\label{fig:chap4-usair97-b250}数据集USAir97,预算限制为$\mathcal{B}=250$}
	{\includegraphics[width=5.5cm]{./chap4/results/USAir97-B250}}
	\hspace{3em}
	\subcaptionbox{\label{fig:chap4-usair97-b300}数据集USAir97,预算限制为$\mathcal{B}=300$}
	{\includegraphics[width=5.5cm]{./chap4/results/USAir97-B300}}
	\caption{XLT4C模型下数据集USAir97在不同预算限制下的$\mathcal{R}_{alg}$值}
	\label{fig:chap4-usair97-result}
\end{figure}

\begin{figure}[H]
	\centering%
	\subcaptionbox{\label{fig:chap4-blogs-b300}数据集Blogs,预算限制为$\mathcal{B}=300$}
	{\includegraphics[width=4.5cm]{./chap4/results/Blogs-B300}}
	\hspace{3em}
	\subcaptionbox{\label{fig:chap4-blogs-b400}数据集Blogs,预算限制为$\mathcal{B}=400$}
	{\includegraphics[width=4.5cm]{./chap4/results/Blogs-B400}}
	\hspace{3em}
	\subcaptionbox{\label{fig:chap4-blogs-b500}数据集Blogs,预算限制为$\mathcal{B}=500$}
	{\includegraphics[width=4.5cm]{./chap4/results/Blogs-B500}}
	\hspace{3em}
	\subcaptionbox{\label{fig:chap4-blogs-b600}数据集Blogs,预算限制为$\mathcal{B}=600$}
	{\includegraphics[width=4.5cm]{./chap4/results/Blogs-B600}}
	\caption{XLT4C模型下数据集Blogs在不同预算限制下的$\mathcal{R}_{alg}$值}
	\label{fig:chap4-blogs-result}
\end{figure}

\begin{figure}[H]
	\centering%
	\subcaptionbox{\label{fig:chap4-ba-b250}数据集BA\_weight,预算限制为$\mathcal{B}=250$}
	{\includegraphics[width=4.5cm]{./chap4/results/BA-B250}}
	\hspace{3em}
	\subcaptionbox{\label{fig:chap4-ba-b300}数据集BA\_weight,预算限制为$\mathcal{B}=300$}
	{\includegraphics[width=4.5cm]{./chap4/results/BA-B300}}
	\hspace{3em}
	\subcaptionbox{\label{fig:chap4-ba-b350}数据集BA\_weight,预算限制为$\mathcal{B}=350$}
	{\includegraphics[width=4.5cm]{./chap4/results/BA-B350}}
	\hspace{3em}
	\subcaptionbox{\label{fig:chap4-ba-b400}数据集BA\_weight,预算限制为$\mathcal{B}=400$}
	{\includegraphics[width=4.5cm]{./chap4/results/BA-B400}}
	\caption{XLT4C模型下数据集BA\_weight在不同预算限制下的$\mathcal{R}_{alg}$值}
	\label{fig:chap4-ba-result}
\end{figure}

\subsection{实验分析}
我们将结合各算法的运行效率以及竞争的效果对各个算法进行分析,根据表格\ref{tab:chap4-algs-time}中,我们可以看出Greedy算法时间复杂度最高,而DH4C,LDH4C的复杂度最低,LG4C则介于他们之间。特别需要注意的是Greedy算法相对于DH4C或者LDH4C算法在运行时间上相差7-8个数量级,而LG4C则与DH4C,LDH4C相差5-6个数量级。但是随着网络图不断增加,Greedy耗时增加的越来越快,然而LG4C的耗时增加相对于Greedy则没有那么快,如USAir97中$\frac{T_{Greedy}}{T_{LG4C}}=\frac{2656.336}{42.0492}=63.17$,而在Blogs和BA\_weight中分别为$\frac{T_{Greedy}}{T_{LG4C}}=\frac{114851.6}{516.659}=222.29$和$\frac{T_{Greedy}}{T_{LG4C}}=\frac{108265.7}{511.642}=211.6$。


在有预算限制的情况下,有时我们所选取的结点数是要低于我们所限定的数值$K$的,从图\ref{fig:chap4-insufficient-budget}中可以看出,在三个数据集中,当预算不足时,实际的初始结点集大小并不能达到我们所期望的大小,这点对于竞争的效果会有一定的影响,因为结点集的数量变少,必然使得竞争一方的传播效果减弱。具体分析算法在竞争的效果上,我们从图\ref{fig:chap4-usair97-result},图\ref{fig:chap4-blogs-result},图\ref{fig:chap4-ba-result}中可以知道LG4C的效果要优于DH4C,而DH4C则优于LDH4C。对于我们定义的$\mathcal{R}_{alg}$值,在相等条件下的某个具体点,算法DH4C,LDH4C,LG4C的$\mathcal{R}_{alg}$值的大小,从这点上看算法LG4C的结果基本上都占优。分析算法LG4C的折线图,我们可以发现其一般都是先随着初始结点集的大小递增到某个峰值,然后慢慢往下降。这个现象的原因是开始时竞争双方的初始结点集都很小,此时由于竞争方的结点阻塞Greedy算法的传播效果不充分,所有使得Greedy算法占的优势更为明显;随着结点集的大小增加,那么竞争方对于阻塞的效果慢慢体现出来,这样就使得$\mathcal{R}_{alg}$值开始增加;最后当结点集大小达到一定的值时,Greedy算法由于其贪心最优的特性,开始有着稳定的主导优势(Dominate Advantage)。

综合上面两方面的讨论,算法LG4C的表现很好,一方面相对于效果很好的Greedy算法不需要消耗太长的计算时间,另一方面相对于运行时间少的DH4C,LDH4C算法在竞争的效果上能做的更好。


\section{本章小结}
本章将前面章节的内容进行了拓展,从没有竞争的影响力分析过渡到有竞争的影响力分析。我们对已有的方法进行了仔细的调研,发现现有算法在顶点价值估计上与现实不符,并且已有的传播模型都存在的一定的问题。基于这些发现的问题,本章主要贡献可以归结如下:(1)利用新的估值模型PRBC的顶点估值函数对网络中点进行代价评估;(2)基于经典的LT模型提出了XLT4C的传播模型,并对XLT4C模型进行了详细的描述,然后给出了相应的模型算法;(3)基于新提出的XLT4C模型,我们提出了启发式算法LDH4C以及LG4C,并在不同的数据集上进行实验,证实我们提出算法的有效性。
% 下面这句用以支持中文
% !Mode:: "TeX:UTF-8"

%%% Local Variables:
%%% mode: latex
%%% TeX-master: t
%%% End:

\chapter{ISSD信息传播结构多样化模型}
\label{cha:5thChap05}

\section{引言}
近年来,信息传播是复杂网络传播动力学研究的热点之一,特别是随着社交网络的爆炸式发展,越来越多的人开始利用微博、微信、人人网等来传播信息、分享趣事和观点,这为研究信息传播提供了前所未有的实际数据。信息传播对于新思想、新技术、新产品等带来了无限的商机与机遇,同时也对社会稳定造成了极大的危害,甚至引发社会动荡,这给谣言控制、舆情监控、信息引导提出了新的挑战。因此,研究信息传播具有极其重要的理论与现实实际意义。

目前信息传播研究,从宏观角度分析,主要基于传染病传播的研究框架模型,参照第~\ref{SIModel}节,考虑了信息传播的部分特征,进行了建模分析,但是忽略了传播者的个体差异,及信息传播与传染病传播完全不同的独有特性;从微观角度分析如独立级联模型(IC),具体参照第~\ref{chap2:ICModel}节,和线性阈值模型(LT),具体参照第~\ref{chap2:LTModel}节,但是没考虑信息传播者的外部环境影响因素和随时间变化等因素。虽然目前研究人员开始广泛聚焦信息传播研究,但是信息传播是个及其复杂的过程,新媒体下信息传播的模型是什么?其传播特征规律是什么?如何刻画信息传播随时间动态演化的过程?等等,依然是一个个待深入研究的重要内容。本章重点从微观结合中观的层次角度进行分析,探索信息传播的模型与规律特点。


\section{相关工作}
目前,越来越多的用户应用社交网络传播信息、分享信息。如腾讯微信、新浪微博等成为人们获取信息的重要渠道来源、交互信息的重要平台和共享信息的载体。信息在社交网络上传播是一个相当复杂的过程,除了涉及到个人用户的认知心理、个人喜好、个人知识背景等个体属性因素,还会受所在的社交圈子、朋友之间的关系程度,以及整个社会大环境对相关话题的关注程度等,这些外部环境因素的多方面影响。在已有的相关研究中,信息传播常建立在传染病模型基础上进行宏观分析。文献\cite{huang2011preventing}利用SIS的信息传播模型,研究了谣言的传播过程,文献\cite{zhao2012sihr}提出了SIHR模型,也分析了谣言的规律。文献\cite{zhangyancao2011}利用传染病动力学理论分析了信息传播的SIR模型。Anderson等人\cite{anderson1991infectious}提出了SEIR模型,在经典的SIR模型中增加了潜伏节点E(Exposed)。文献\cite{lu2011small}也利用传染病模型分析了信息在规则网络、小世界网络中的传播规律,并归纳了信息传播的一些特点,这些信息传播的研究给我们提供了很好的参考依据。但上述基于传染病传播模型的信息传播方法,只是从整体角度分析了传播规律,没有考虑到每个人用户(节点)的个体差异,及信息随时间变化的特有规律,因为信息传播与传染病传播是有本质不同的。Ugander等人\cite{ugander2012structural}利用实际Facebook真实数据,实际分析了社交网络中用户接受信息具有的结构多样性的特点。如用户从两个不同圈子听到一条信息其可信程度要远远高于从一个圈子内听到两次。其过程分析如图~\ref{fig:DSpnas2012}所示。这篇文章使用相对转化率(Relative conversion rate)代表个体被说服注册Facebook的可能性大小。首次观测了邻居的数目和邻居连通子图对相对转化率的影响,文章详细列出了2、3、4个邻居的用户的相对转化率的情况。从下面的图中可以看出,固定邻居数量,在连通子图的数目相同时,不同连接情况下的转化率相差不大,但是当连通子图数目增大时,转化率会有相当大的提升。
\begin{figure}[H] % use float package if you want it here
	\centering
	\includegraphics[scale=0.9]{./chap5/SDpnas2012}
	\caption{Facebook中结构多样性对用户转化分析情况\cite{ugander2012structural}}
	\label{fig:DSpnas2012}
\end{figure}

%\subsection{overlap}
%\subsubsection{community}
%\subsubsection{community}
%\subsection{unoverlap}
%\subsubsection{clique}
%\subsubsection{k-trussess}
%\subsubsection{connect component}
\subsection{信息传播时间变化模型}
%信息衰减函数(newton cool)
信息传播随着时间的变化非常地快,我们参照参考文献\cite{yang2011patterns}中对社交网络Twitter一些热点词及标签内容随时间的变化进行分析,如图~\ref{fig:chap05WSDM11Wang}如示。新信息在产生后迅速传播成为热点信息,然后随着时间快速衰减,成为过时信息。
\begin{figure}[H] 
	\centering
	\includegraphics[scale=0.34]{./chap5/WSDM11Wang}
	\caption{Twitter中某些信息被关注提及随时间的变化规律}
	\label{fig:chap05WSDM11Wang}
\end{figure}
我们从 Alex Bentley等人\cite{bentley2011ll}书中可以知道人们接受一些行为是如何由一小部分热衷创新、学习的人发起,到后来怎么被大众模仿传播开来的。其中有两个重要模式:个人学习模式、模仿学习(社会学习)模式,不同模式下的接受率不同,如图~\ref{fig:chap05IwillhaveWhate}所示,不同的学习模式接受率的增幅变化是不同,这与我们对信息传播受内因、外因共同影响的思路是一致的。
\begin{figure}[H] 
	\centering
	\includegraphics[scale=0.6]{./chap5/chap05IwillhaveWhate}
	\caption{社会信息在个人学习、模仿学习模式下的接受规律}
	\label{fig:chap05IwillhaveWhate}
\end{figure}
%http://www.alex-bentley.com/books
我们也对关键词“环境污染”、“雾霾”及“caijing”应用Google趋势进行统计分析。从图~\ref{fig:chap05Haze1}可以看出“环境污染”2005年大家高度关注,随着时间关注度逐渐下降。由于2015年2月柴静关于雾霾的新闻报告《穹顶之下》又一次爆发成为人们关注的焦点,如图~\ref{fig:chap05Haze2}所示。所以我们识为信息传播不但与用户节点的邻居节点结构多样化的内部环境相关,而且与整个网络外部信息大环境相关,同时我们可以看到人们对信息的关注度随时间变化非常迅速。
\begin{figure}[H] 
	\centering
	\includegraphics[scale=1.0]{./chap5/Haze1}
	\caption{关键词在2004年-2015年3月被关注度随时间的变化趋势}
	\label{fig:chap05Haze1}
\end{figure}
\begin{figure}[H] 
	\centering
	\includegraphics[scale=1.0]{./chap5/Haze2}
	\caption{关键词在2015年2月-2015年3月被关注度随时间的变化趋势}
	\label{fig:chap05Haze2}
\end{figure}

根据以上相关分析,我们可以知道信息传播具有活性,其活跃程度会随外界整个信息大环境影响,且变化很快。我们参照物理学中的"牛顿冷却定律"(Newton's Law of Cooling)\cite{winterton1999newton,incropera2011fundamentals}对信息传播变化进行了相应的建模。牛顿冷却定律是:物体温度变化与外界的温差成正比,如公式:
\begin{equation} 
\label{chap05newtoncool}
T'(t) = -\alpha (T(t)-H) ,\alpha > 0
\end{equation}
其中$\alpha$ 负表示当物体自身温度高于外界温度,相反外界温度大于物体温度,物体温度会升高。如图~\ref{fig:chap05newtonTemp}所示。
\begin{figure}[H] 
	\centering
	\includegraphics[scale=1]{./chap5/newtonTemp}
	\caption{牛顿冷却定律中物体温度随外界温度变化情况}
	\label{fig:chap05newtonTemp}
\end{figure}

我们对应建立信息随时间变化的数学模型--信息传播时间变化模型,用户(个体)对信息感兴趣程度与用户周围邻居知道信息所占比例成正比。可以表示为:
%比例参照2002 watts
\begin{equation} 
\label{chap05infocool}
I'(u,t) = -\rho (I(u,t)-H(u))
\end{equation}
其中$H(u)$为用户$u$对某类信息感兴趣程度,根据实际个体情况设定初始阈值,$I(u,t)$为用户$u$在时间$t$时对某个信息的感兴趣程度;$\rho $ 为兴趣变化参数,如果用户对某信息的兴趣度大于外界信息热度,$\rho \geq 0$,否则相反,人们会对某个信息随时间的变化,进行信息过滤衰减。人们对信息的兴趣随时间变化过程如图~\ref{fig:chap05infoNewton}所示,从图~\ref{fig:chap05infoNewton}上变化的简单曲线可以理解新产品、新技术、新思想的传播趋势走向。虽然这些变化曲线并不能解释一切传播现象,但是我们能够根据一些信息传播的特点作出具体、有用的假设,再进行采样校正,进一步细化,达到传播趋势分析精确化。
\begin{figure}[H] % use float package if you want it here
	\centering
	\includegraphics[scale=2.0]{./chap5/infoNewton}
	\caption{用户节点对某个信息感兴趣随时间变化规律}
	\label{fig:chap05infoNewton}
\end{figure}

对~\ref{chap05infocool}解不定积分推导过程如下:
\begin{equation} 
\label{chap05infocooldirivation}
\begin{split}
&\int \frac{I'(u,t)}{I(u,t)-H(u)} = \int (-\rho)dt \\
&\rightarrow \log |I(u,t)-H(u)| = -\rho t + c \\
&\rightarrow |I(u,t)-H(u)| = C \exp ^{-\rho t }
\end{split}
\end{equation}

\subsection{信息传播结构多样性}
信息传播过程中,用户接受信息受其所在外部环境的影响,即用户的自我网络环境(Egocentric network,简称为Ego网络)\cite{marsden2002egocentric,fisher2005using,weng2014topic}结构多样性对接受信息的影响。Ego网络是整个关系网络的子图,包括一个中心节点和其邻居节点,边包括中心节点与邻居节点,及其邻居节点之间的连线。如图~\ref{fig:chap05SD1}所示,以节点32中心的Ego网络。

\begin{figure}[H]
	\centering%
	\subcaptionbox{节点32的Ego图\label{fig:chap05SD1}}
	%标题的长度,超过则会换行,如下一个小图。
	{\includegraphics[scale=0.36]{./chap5/SD1}}
	%\hspace{3em}%
	\subcaptionbox{节点32邻居知道某个信息的环境\label{fig:chap05SD2}}
	{\includegraphics[scale=0.36]{./chap5/SD2}}
	
	\caption{以Zachary俱乐部网络为例,节点32,接受不同圈子信息的多样性结构}
	\label{fig:chap05egonetwork}
\end{figure}
我们以节点[32]作为信息的接收点,如图~\ref{fig:chap05SD2}所示,假设你同时从你研究生同学节点[25,26]分别得到一信息,其可信度与你分别从高中同学[1]和大学同学[34]得到一消息是完全不同的,你对后一种情况下信息的可信性和感兴趣程度将倍增,你转发传播该信息的概率将会累积增加,参考了文献\cite{ugander2012structural}。

从信息接收者的角度来看,其是否会推动信息的传播由其自身的内部因素决定的同时,也与其所处的外部环境密切相关。我们所提出的信息传播模型同时也考虑了这两方面的因素。我们根据传播信息节点与未知信息节点关系紧密程度,及信息传播结构的多样性,从以下几个方面对进行了分析、考虑和划分,并给出了实验分析结果及比较。
\begin{enumerate}[(1)]
	\item 从重叠社区划分角度分析,每个节点可以属于多个社区(团体)。重点参考、分析了Clique划分方法\cite{palla2005uncovering}。
	\item 对重叠社区的Clique方法限制进行了放宽,从K-trussess角度分析\cite{cohen2008trusses,csermely2013structure,redmond2012mining,huang2014querying,rossi2014fast}。
	\item 从无重叠社区划分(Community)\cite{newman2004finding}角度分析,每个节点只属于唯一的一个社区。因为不同的算法划分社区的模块优化值不同,社区节点之间的紧密程度也不尽相同。
	\item 从连通子图角度分析,每个一节点所属的Ego网络如果节点的邻居节点之没有连通,我们就认为他们属于不同的社区(或圈子)。
\end{enumerate}

%同时我们的模型也可成为重要热点人物发现,热点团体发现的研究基础。这为信息的研究提供了新的思路和方法。

\section{ISSD模型与算法}
\subsection{信息传播假设}
我们根据信息传播的特性做出以下假设:
\begin{assumption}
\label{chap5:assumption1}
	信息传播具有记忆性、累积性。
\end{assumption}

这与疾病传播是不同,个体在没完全接受之前,信息会留在记忆中,只是会因为对信息的兴趣程度,所留下的印象程度有所不同。

\begin{assumption}
\label{chap5:assumption2}
	信息传播具有活性。%人们接受心理不同
\end{assumption}

信息活跃程度会随外界、周围信息环境变化,会随时间快速增加然后迅速衰减。A.M.Treisman\cite{treisman1980feature}的过虑器理论模型认为,在多通道信息加工过程中,信息得到处理的程度不同,某些信息通过过滤器时会被衰减,实验证明衰减模型能对人的行为做出很好的预测。

\begin{assumption}
\label{chap5:assumption3}
	信息传播具有社会加强性。%信息的累积性,及多样性的增强效应
\end{assumption}

特别是由于所处的环境具有结构多样性,其加强性表现的更加突出。譬如当你高中圈子的一个同学说一信息时你可能没太在意,可是高中同学圈子越来越多的同学说该信息,你开始留意了;但是你大学同学的圈子也在讨论这一件事,那么你这个信息的可信性和感兴趣程度将倍增,你相信该信息的概率将会累积增加。

\begin{assumption}
	\label{chap5:assumption4}
	信息接受程度不同。%连权重
\end{assumption}

信息传播中由于网络成员之间关系不同,对不同联系人员所传播信息内容信任程度和接受程度不同。如在我们现实生活朋友圈子中,我们对铁哥们(闺蜜),亲密朋友及认识朋友传来的信息信任程度是完全不同的。
\begin{assumption}
	\label{chap5:assumption5}
	信息传播中传播个体存在差异。
\end{assumption}

信息传播中不同传播者,由于每个个体自身年龄、文化程度、知识背景、工作性质、生活环境等不同,对信息内容的兴趣程度不同存在差异。%个体阈值不同
\subsection{信息传播机理}
结合相关研究,我们把社会网络中个体(节点)对信息的了解程度分为未知信息状态(\underline{\textbf{U}}nknown),表示没有接收到信息;得知信息状态(\underline{\textbf{K}}nown),接收到信息但是不准备传播信息;传播信息状态(\underline{\textbf{A}}pproved),发送信息、传播信息;信息疲惫态(\underline{\textbf{E}}xhausted),不再传播已传播的信息,只是接收但是放弃传播信息等四种状态类型。社会网络中,一个用户产生或发布一条信息会传播到与其连接的邻居用户,然后自己在下一个时间阶段对此信息失去兴趣,不再重复传播,变为信息疲惫者;邻居用户得知这条信息后根据个人的喜好、兴趣程度等因素或者成为得知信息状态或者成为传播信息状态,若得知信息状态用户受信息累积而传播此信息,即其信息状态由得知信息状态变为传播信息状态;在信息传播过程中如果遇到信息疲惫状态用户,信息将不被传播。传播过程个体用户状态转变过程如图~\ref{fig:SDIMgraph}所示。
\begin{figure}[H] % use float package if you want it here
	\centering
	\includegraphics[scale=0.7]{./chap5/SDIMgraph}
	\caption{信息传播节点状态转变示意图}
	\label{fig:SDIMgraph}
\end{figure}

%信息在U、K、A、E四种状态之间的传播规则描述如下:
%\begin{enumerate}[(1)]
%\item 对于确认信息节点会以$p_1$的概率影响其邻居节点,使其邻居节点转变为,$p_1$的值为节点之间的影响力、所处网络结构等因素量化所得。
%\item 对于确认信息节点会以$p_1$的概率影响其邻居节点,使其邻居节点转变为
%\end{enumerate}

我们以著名的Zachary\cite{zachary1977information}空手道俱乐部成员关系网络为例(如图~\ref{fig:ZacharyS1}所示),给出信息传播过程中每个节点的信息状态随时间的演变过程,设定初始传播信息节点为[1,34],如图~\ref{fig:chap05ZacharySpreading}所示。
\begin{figure}[H] % use float package if you want it here
	\centering
	\includegraphics[scale=0.8]{./chap5/ZacharyS1}
	\caption{Zachary 俱乐部人员网络关系图}
	\label{fig:ZacharyS1}
\end{figure}
%ISSDgraph
\begin{figure}[H]
\centering%
	\subcaptionbox{t=1\label{fig:ZacharyS2}}
	%标题的长度,超过则会换行,如下一个小图。
	{\includegraphics[scale=0.36]{./chap5/ZacharyS2}}
	\hspace{1mm}%
	\subcaptionbox{t=2\label{fig:ZacharyS3}}
	{\includegraphics[scale=0.36]{./chap5/ZacharyS3}}
	
	\subcaptionbox{t=3\label{fig:ZacharyS4}}
	%标题的长度,超过则会换行,如下一个小图。
	{\includegraphics[scale=0.36]{./chap5/ZacharyS4}}
	%\hspace{3em}%
	\subcaptionbox{t=4\label{fig:ZacharyS5}}
	{\includegraphics[scale=0.36]{./chap5/ZacharyS5}}
	
	\subcaptionbox{t=5\label{fig:ZacharyS6}}
	%标题的长度,超过则会换行,如下一个小图。
	{\includegraphics[scale=0.36]{./chap5/ZacharyS6}}
	%\hspace{2em}%
	\subcaptionbox{t=6\label{fig:ZacharyS7}}
	{\includegraphics[scale=0.36]{./chap5/ZacharyS7}}
	\caption{以Zachary网络为例,选定初始信息传播种子节点为[1、34],随着时间信息传播演化过程}
	\label{fig:chap05ZacharySpreading}
\end{figure}

%\section{ISSD模型与算法}

\subsection{基本概念定义}
信息传播的社会网络可以描述为由节点及节点之间的边组成的复杂网络结构,$G(V,E,A)$,其中$V$是所有结节的集合,节点的个数为$n=|V|$,节点的属性可以是个体自身特征,如用户的爱好、所属社区、年龄、性别等。$E$为所有边的集合,为$E\subseteq V\times V$,$e_{uv}\in E$且$(u,v \in V)$,边的个数为$m=|E|$,边的属可以为节点之间的影响权重、紧密程度等。$A$为图的属性,可以是图的名称代号、网络结构特点等。无重叠社区划分(Community)$(c_1,c_2,\dots,c_n|c_i \in C)$,且$c_i\cap c_j = \phi$。有重叠社区划分(K-clique)$(k_1,k_2,\dots,k_n|k_i \in K)$,且$|k_i\cap k_j| \geq 0$。 有重叠社区划分(K-trusses)$(tr_1,tr_2,\dots,tr_n|tr_i \in Tr)$,且$|tr_i\cap tr_j| \geq 0$。网络连通子图(Connected Component)$(cc_1,cc_2\dots,cc_n|cc_i \in Cc)$,且$|cc_i\cap cc_j| \geq 0$。

信息随时间传播过程行为形式化描述为:$I(V,E,T)$,某个信息$(i_1,\dots,i_n|i_j \in I)$,传播时间$(t_1,t_2,\dots,t_n|t_i \in T)$。例如$i_j(u,t_1)$,表示为节点$u$在时间$t_1$产生了信息$i_j$。$i_j(e_{uv},t_2)$,表示为节点$u$在$t_2$时间将信息$i_j$传播给了邻居节点$v$,具体过程如图~\ref{fig:chap05VETgraph}所示。
\begin{figure}[H] 
	\centering
	\includegraphics[scale=1.0]{./chap5/chap05VETgraph}
	\caption{信息随时间传播过程形式化描述样例}
	\label{fig:chap05VETgraph}
\end{figure}
\subsection{ISSD模型}
信息传播过程具有复杂多样性,及不确定性。这里我们重点考虑信息传播过程所具有的5个假设特性,提出了信息传播结构多样性模型(ISSD,Infomation Spreading Structural Diversity Model)。
以前研究模型和方法主要从传播节点触发、激活其它节点的角度分析问题,而我们从被触发或将要被激活的节点这一独特的角度分析和解决问题。我们认为未知信息节点(\textbf{U})被激活(或感染)的概率不但与传播信息节点(\textbf{A})之间的关系程度有关,即在网络关系图中表示 为节点之间边的权重。我们还特别考虑了接收信息节点所处的外部环境这个外部因素,如其所在的社区结构多样性特征等;同时我们也着重考虑了信息传播时间变化因素。总体与哲学范畴中“事物的发展是内因、外因共同起作用的结果”思想相符。这也与第~\ref{cha:3thChap03}章节中最具影响力节点发现思路也是一致的。

ISSD模型可以描述为
:在复杂网络$G=(V,E,A)$中,定义$N(u)$为节点$u$的邻居节点集合,$X$为某种社区划分方法,$(x_1,\dots,x_n|x_i \in X)$,$X$可以为社区划分中的C、K、Tr、Cc等。未被激活(未知信息)的节点$u$,不但受其激活邻居节点$S_a$(传播信息节点)中$v$所产生的影响力$b_{uv}$影响,同时受其Ego网络中激活邻居节点所在的结构多样性变化的社区环境影响,参阅假设,如信息传播的社会加强性,信息在“影响积累”的同时也会随周围信息的冷热度进行变化,但是一个节点$u$的受其所有激活邻居节点对$u$的影响力总和小于等于1,即:
\begin{equation} 
{\sum_{x_i=1}^{x_i=k}   {\sum_{v \in N(u),v\in S_a}} b_{uv} \leq 1} 
\end{equation}
我们假设每个节点$u$有一个激活特定阈值$\theta_u\in\left[0,1\right]$,如果${\sum_{x_i=1}^{x_i=k}   {\sum_{v \in N(u),v\in S_a}} b_{uv} \geq \theta_u}$,则节点$u$由未知状态(U)被激活为传播信息状态(A)。在ISSD模型中,当某个圈子(或社区)的一个激活节点$v$尝试激活它的未激活邻居$u$没有成功时,那么节点$v$对节点$u$的影响力被积累起来,信息从不同圈子来,可信度、影响力是不同的,这样对后面其它已激活的邻居节点对$u$的激活是有贡献的。直到所有节点被激活或没有新的节点再被激活,信息传播的整个过程结束。这就是ISSD模型中信息传播结构多样性的特性,这与IC模型、LT模型及传染病模型等所不同的。

下面我们加入信息传播时间变化因素。具有时间特征的ISSD模型可描述为,每个节点$u$都有确认传播信息的阈值$\theta_u \in [0,1]$。对于节点$u$会受其邻居中已传播信息状态(也叫激活状态)节点$v\in N(u), v\in S_a$的影响,设其中$g(u,t)$为节点$u$在时间$t$时,受其邻居节点中确认信息状态节点的结构多样化影响程度函数;$f(u,t)$为节点$u$在时间$t$时的信息累积程度函数。且${\sum_{x_i=1}^{x_i=k}   {\sum_{v \in N(u),v\in S_a}} b_{uv} \le \theta_u}$,其中$X \subseteq ({C\cup K\cup Tr\cup Cc}) $。那么在$t + 1$时,当满足条件:
\begin{equation}
\label{chap05infocoolsumtheta}
\begin{split}
& f(u,t_1){\sum_{v \in N(u),v\in S_a}} g(u,t_1)b_{uv} + f(u,t_2){\sum_{v \in N(u),v\in S_a}} g(u,t_2)b_{uv} + \cdots \\
& + f(u,t_{t+1}){\sum_{v \in N(u),v\in S_a}} g(u,t_{t+1})b_{uv}  \geq \theta _u
\end{split}
\end{equation}
时,节点$u$将变为激活状态。模型中参数在不同类型的信息传播网络中是不同的,节点对不同种类的信息传播初始阈值也不相同,参数的变化可以根据实际情况来确定。下面给出了模型的相应算法。

\subsection{ISSD算法}
为了对ISSD算法表述方便,表~\ref{tab:ISSD_table}列出了相关重要变量参数。
\begin{table}[htbp]
	\centering
	\begin{minipage}[t]{0.9\linewidth}
		\caption{ISSD算法参数描述说明}
		\label{tab:ISSD_table}
		%\begin{tabular}{*{7}{p{.14\textwidth}}}
		\begin{tabular}{p{2.1cm}p{9cm}}
			\toprule[1.5pt]
			\heiti 变量 &  \heiti 说明描述 \\  
			\midrule[1pt]
			$S$& 初始传播的种子节点 \\
			$S_a$& 传播信息状态的节点 \\
			$S_k$& 得知信息状态的节点,对信息进行了累积记忆 \\
			$S_e$& 信息疲惫状态的节点,节点不再传播已传播的信息 \\
			$newS_a$& 新的被激活传播信息状态的节点 \\
			$time$& 信息传播的时间 \\
			$neighbor(u)$& 与节点$u$连接的邻居节点 \\
			$x$& 节点归属哪个社区,可以为无重叠社区(Community)、重叠社区(K-clique)、连通子图和K-trusses子图等社区划分结果 \\
			$f(x,t)$& 信息传播随时间变化的衰减(增强)函数 \\
			$g(x,t)$& 信息传播结构多样性函数 \\
			\bottomrule[1.5pt]
		\end{tabular}
	\end{minipage}
\end{table}

信息传播结构多样化传播算法描述如算法4所示:
\begin{algorithm}
	\caption{ISSD(G,S)}%Infomation Spreading Structural Diversity Model
	\label{alg:chap5:ISSDalgorithm}
	\begin{algorithmic}[1]
		\STATE $S_a=newS_a=S; S_e \leftarrow \Phi; S_k \leftarrow \Phi; time = 1$
		\WHILE {$|S|\geq 0$}
		\STATE $S_k = neighbor(S)$;$newS_a \leftarrow \Phi$
		\FOR {each  vertex $v \in S_k\setminus S$ }
		\FOR {each vertex $u \in S $}
		\IF {$\sum_{v \in x}(f(g(x,t),t)e_{uv}[weight]) \geq \theta_u$ }
		\STATE $newS_a = newS_a \cup \{v\}$
		\ENDIF
		\ENDFOR
		\ENDFOR
		\STATE time +=  1
		\STATE $S_e = S_e \cup S_a $;$S_a = S_a \cup newS_a$;$S_k=S_k \cup \setminus S$
		\STATE $S=newS_a$;
		\ENDWHILE
		\RETURN $S_e,S_a,S_k$
	\end{algorithmic}
\end{algorithm}

\section{实验结果与分析}
\subsection{ISSD时间影响}
我们通过在实际关系网络实验分析,发现了信息传播过程与我们以往认知不同的现象,一般认为,度数大的个体节点是推动信息传播的主要因素和关键因素,初始传播源节点集合的影响力越多,传播范围越大。但我们对节点按度数大小进行排序,逐步增大度数Top-k初始传播节点集合,由于某些节点的加入时机不同,反而阻碍了信息的传播,使传播范围反而明显减小。

这里以《悲惨世界》人物关系网络\cite{knuth1993stanford}为例进行说明。如图~\ref{fig:chap05lesmis}所示,数据源\cite{knuth1993stanford}是D. E. Knuth 整理了小说的人物关系网络,节点表示小说中的角色,边表示两个角色同时出现在一幕或多幕中,网络关系如图~\ref{fig:chap050lesmis}所示。
\begin{figure}[H] 
	\centering
	\includegraphics[scale=0.6]{./chap5/0lesmis}
	\caption{悲惨世界人物关系网络节点表示}
	\label{fig:chap050lesmis}
\end{figure}

网络中主要人物如表~\ref{tab:lesmisNodeMap}所示。
\begin{table}[!htb]
	\begin{minipage}[t]{1\linewidth}
		\centering
		\caption{悲惨世界人物关系网络中节点度数前10名}
		\label{tab:lesmisNodeMap}
		\begin{tabular}{p{.2\textwidth}<{\centering}p{0.2\textwidth}<{\centering}p{0.2\textwidth}<{\centering}p{0.1\textwidth}<{\centering}p{0.1\textwidth}<{\centering}}
			\toprule[1.5pt]
			{\heiti 人物英文名} & {\heiti 人物中文名} & {\heiti 角色} & {\heiti 编号} & {\heiti 度数} \\\midrule[1pt]
			Jean Valjean & 冉阿让 & 主人公 & 11 & 36 \\ 
			Gavroche & 加夫罗契 & 革命青年 & 48 & 22 \\ 
			Marius & 马吕斯 & 爱潘妮的弟弟 & 55 & 19 \\ 
			Javert & 贾维 & 探长 & 27 & 17 \\ 
			Thenardier & 德纳 & 酒馆的老板 & 25 & 16 \\ 
			Fantine & 芳汀 & 女工 & 23 & 15 \\ 
			Enjolras & 恩佐拉 & 革命青年 & 58 & 15 \\ 
			Courfeyrac & 古费拉克 & 学生 & 62 & 13 \\ %学生酱油党里稍微出挑一点的是Courfeyrac
			Bossuet & 博须埃 & 主教 & 64 & 13 \\ 
			Bahorel & 巴阿雷 & 学生 & 63 & 13 \\
			\bottomrule[1.5pt]
		\end{tabular}
	\end{minipage}
\end{table}
%\begin{table}[htb]
%	\centering
%	\caption{悲惨世界人物关系网络中节点度数前10名}
%	%\label{ISSD_table}
%	\label{tab:lesmisNodeMap}
%	\begin{tabular}{|c|c|c|c|c|} \hline
%		\heiti 人物英文名 & \heiti 人物中文名 & \heiti 角色 &  \heiti 编号 &  \heiti 度数 \\ \hline
%		Jean Valjean & 冉阿让 & 主人公 & 11 & 36 \\ \hline
%		Gavroche & 加夫罗契 & 革命青年 & 48 & 22 \\ \hline
%		Marius & 马吕斯 & 爱潘妮的弟弟 & 55 & 19 \\ \hline
%		Javert & 贾维 & 探长 & 27 & 17 \\ \hline
%		Thenardier & 德纳 & 酒馆的老板 & 25 & 16 \\ \hline
%		Fantine & 芳汀 & 女工 & 23 & 15 \\ \hline
%		Enjolras & 恩佐拉 & 革命青年 & 58 & 15 \\ \hline
%		Courfeyrac & 古费拉克 & 学生 & 62 & 13 \\ \hline%学生酱油党里稍微出挑一点的是Courfeyrac
%		Bossuet & 博须埃 & 主教 & 64 & 13 \\ \hline
%		Bahorel & 巴阿雷 & 学生 & 63 & 13 \\ \hline
%	\end{tabular}
%\end{table}

为了保证实验现象的普遍性,我们对悲惨世界网络中节点随机设定在[0,1]之间初始阈值,且整体初值设定整体符合指数正态分布\cite{limpert2001log,holgate1989lognormal}。对于节点之间的相互影响力随机设定,且满足在[0,1]的正态分布\cite{lodzimierz1995normal,patel1996handbook,amari2007methods}。我们选择度数最大的节点[11],作为初始传播源节点,随着时间各节点信息状态的演化结果,如图~\ref{fig:chap05node1lesmis}。具体传播演化过程为: \\
$  t=1,[11] \rightarrow t=2,[10, 15, 27] \rightarrow t=3,[70, 29, 24] \\ 
\rightarrow t=4,[68, 69, 23, 25, 58] \rightarrow t=5,[21, 41, 57, 65, 71, 75] \\ 
\rightarrow t=6,[16, 17, 19, 48, 55, 59, 62, 63, 76] \rightarrow t=7,[18, 20, 22, 60, 61, 64, 66]$

\begin{figure}[H]
	\centering%
	\subcaptionbox{t=1\label{fig:chap0511lesmis}}
	%标题的长度,超过则会换行,如下一个小图。
	{\includegraphics[width=7.3cm]{./chap5/11lesmis}}
	%\hspace{3em}%
	\subcaptionbox{t=2\label{fig:chap0512lesmis}}
	{\includegraphics[width=7.3cm]{./chap5/12lesmis}}

\end{figure}
\addtocounter{figure}{-1}       %先欺骗LaTeX图形计数器
\begin{figure}[H]
	\addtocounter{figure}{1}
	\centering%
	\subcaptionbox{t=3\label{fig:chap0513lesmis}}
	{\includegraphics[width=7.3cm]{./chap5/13lesmis}}
	%\hspace{3em}%
	\subcaptionbox{t=4\label{fig:chap0514lesmis}}
	{\includegraphics[width=7.3cm]{./chap5/14lesmis}}
	\subcaptionbox{t=5\label{fig:chap0515lesmis}}
	%标题的长度,超过则会换行,如下一个小图。
	{\includegraphics[width=7.3cm]{./chap5/15lesmis}}
	%\hspace{3em}%
	\subcaptionbox{t=6\label{fig:chap0516lesmis}}
	{\includegraphics[width=7.3cm]{./chap5/16lesmis}}
	\subcaptionbox{t=7\label{fig:chap0517lesmis}}
	{\includegraphics[width=7.3cm]{./chap5/17lesmis}}
	\caption{节点11作为初始传播源节点传播过程}
	\label{fig:chap05node1lesmis}
\end{figure}
如图~\ref{fig:chap05node4lesmis}所示,我们选择连接度数最大的节点[11,48,55,27],作为初始传播源节点,随着时间各节点信息状态的演化结果。具体传播演化过程为: \\
$  t=1,[11, 48, 55, 27] \rightarrow t=2,[10, 15, 57, 65] \\
\rightarrow t=3,[58, 59, 63] \rightarrow t=4,[64, 66, 76, 60, 61, 62] 
$
\begin{figure}[H]
	
	\centering%
	\subcaptionbox{t=1\label{fig:chap0541lesmis}}
	{\includegraphics[width=7.3cm]{./chap5/41lesmis}}
	%\hspace{3em}%
	\subcaptionbox{t=2\label{fig:chap0542lesmis}}
	{\includegraphics[width=7.3cm]{./chap5/42lesmis}}
\end{figure}
\addtocounter{figure}{-1}       %先欺骗LaTeX图形计数器
\begin{figure}[H]
	\addtocounter{figure}{1}
	\centering%
	\subcaptionbox{t=3\label{fig:chap0543lesmis}}
	{\includegraphics[width=7.3cm]{./chap5/43lesmis}}
	%\hspace{3em}%
	\subcaptionbox{t=4\label{fig:chap0544lesmis}}
	{\includegraphics[width=7.3cm]{./chap5/44lesmis}}
	\caption{节点11,48,55,27作为初始传播源节点传播过程}
	\label{fig:chap05node4lesmis}
\end{figure}

从图~\ref{fig:chap05node4lesmis},图~\ref{fig:chap05node1lesmis}我们可以看到与传统思路相反的结果,即一般认为,越多的影响力节点加入,越能推动整个传播过程。以下我们分析了产生与传统想法相反现象的原因:
\begin{enumerate}[(1)]
	\item 一些关键节点过早地成为信息疲惫状态,中断了信息传播的持续加强性;
	\item 信息传播对邻居节点的信息累积没有影响其信息状态,其后随时间快速衰减;
\end{enumerate}

我们知道信息传播是个复杂的传播演变过程,只是简单地应用传统知识是无法解释信息传播的原理的。另一方面信息传播不但和参与传播的节点相关,而且和参与者加入信息传播的时机密切相关。这也为今后研究提出了新的思考角度与挑战。

\subsection{ISSD结构多样性影响}
目前研究信息传播大多从传播(或已感染)节点去触发、激活其它节点的角度分析问题,而本文我们从被传播或将被激活节点这一独特的角度分析和解决问题,重点分析了被传播节点的Ego环境结构多样化。

目前普遍认为社交网络中度数大,传播速度是最大的,以度中心化指标来衡量信息传播最大化。如新浪微博中利用大V用户(粉丝多的)作为市场营销的注入点,希望通过他们能产生口碑效应(Word of mouth),\cite{trusov2009effects,chevalier2006effect}最终能信息最大化传播。前面章节也发现利用度中心化作为影响力最大化传播方法也存在问题,如“富人俱乐部”现象。例如我们选择Netscience合著网络\cite{boccaletti2006complex}中节点度最大的前10、20个节点形成关系子图,从图~\ref{fig:chap05netscienceTopKDegree}中我们可以看到排名前10中的节点[20,23,26,41]是个K-clique\cite{palla2005uncovering}是4的全连通图,他们是一个研究组的成员,通常他们学术研究内容或范围应该相近。通常会认为,如果其一名学者知道某种理论或技术,其他人员一般也会知道的,所以选择他们其中的任何一个成员传播源节点比较合理。
\begin{figure}[H]
	\centering%
	\subcaptionbox{节点度排名前10的节点关系\label{fig:chap05netscienceTop10Degree}}
	%标题的长度,超过则会换行,如下一个小图。
	{\includegraphics[width=7.3cm]{./chap5/netscienceTop10Degree}}
	%\hspace{3em}%
	\subcaptionbox{节点度排名前20的节点关系\label{fig:chap05netscienceTop20Degree}}
	{\includegraphics[width=7.3cm]{./chap5/netscienceTop20Degree}}
	\caption{合著网络中度排名Top-10(20)节点关系图}
	\label{fig:chap05netscienceTopKDegree}
\end{figure}
同时我们知道许多社会热点事件,大多不是由大V用户发起和主导的,在一定程度上这与他们参与热情及人们对他们信任程度等因素相关。我们知道,往往一些热衷于某些话题的草根用户,他们发起一些话题,然后爆发成为社会焦点或热点问题。

%Ego网络是整个上网络的子图,包括一个中心节点(ego)和其邻居节点,边包括中心节点与邻居节点,及其邻居节点之间的连线。如图~\ref{fig:chap05SD1}所示,以节点32中心的ego网络。
%
%\begin{figure}[H]
%	\centering%
%	\subcaptionbox{节点32的ego图\label{fig:chap05SD1}}
%	%标题的长度,超过则会换行,如下一个小图。
%	{\includegraphics[scale=0.36]{./chap5/SD1}}
%	%\hspace{3em}%
%	\subcaptionbox{节点32邻居知道某个信息的环境\label{fig:chap05SD2}}
%	{\includegraphics[scale=0.36]{./chap5/SD2}}
%	
%	\caption{以Zachary俱乐部网络为例,节点32,接受不同圈子信息的多样性结构}
%	\label{fig:chap05egonetwork}
%\end{figure}
%我们以节点[32]作为信息的接收点,如图~\ref{fig:chap05SD2}所示,你同时从你研究生同学节点[25,26]分别得到一信息,其可信度与你分别从高中同学[1]和大学同学[34]得到一消息是完全不同的,你对后一种情况下信息的可信性和感兴趣程度将倍增,你转发传播该信息的概率将会累积增加。

下面实验,我们即考虑节点自身属性,特别考虑了信息传播过程中节点所处结构多样性的外部环境因素。应用现实实际数据集Netsceince网络、Email网络、Blogs网络,数据具体描述参考表~\ref{dataset-statistics},Netscience网络和Email网络的节点度的幂律分布情况,如图~\ref{fig:chap05netsciencePowerlaw}、~\ref{fig:chap05emailPowerlaw}所示。Blogs网络的节点度的幂律分布情况,见第四章节图~\ref{fig:chap4BlogsPowerlaw}所示。
\begin{figure}[H]
	\centering%
	\subcaptionbox{Netscience数据\cite{boccaletti2006complex},其中节点数为379,边数为914\label{fig:chap05netsciencePowerlaw}}
	%标题的长度,超过则会换行,如下一个小图。
	{\includegraphics[width=7.3cm]{./chap5/netsciencePowerlaw}}
	%\hspace{3em}%
	\subcaptionbox{Email数据\cite{guimera2003self},其中节点数为1133,边数为5451\label{fig:chap05emailPowerlaw}}
	{\includegraphics[width=7.3cm]{./chap5/emailPowerlaw}}
	
	\caption{Netscience、Email网络的度分布情况}
\end{figure}


%\begin{figure}[H]
%	\centering
%	\includegraphics[scale=0.5]{./chap5/blogsPowerlaw}
%	\caption{Blogs数据\cite{xie2006social},其中节点数为3982,边数为6803}
%	\label{fig:chap05blogsPowerlaw}
%\end{figure}

我们采用影响力最大化的典型算法DegreeHeuristic算法\cite{kempe2003maximizing},对ISSD模型与IC模型、LT模型进行了分析比较,在IC模型实验模拟传播过程中,每次选取传播种子节点$1 \leq k \leq 30$作为传播源的种子集合,传播概率取$p=0.01$。三种模型传播范围取10000次迭代求均值,横轴为根据各算法中传播影响最大节点进行排序所得的集合大小,纵轴为在选取相应传播种子节点集合条件下的传播范围大小。

\begin{figure}[H]
	\centering%
	\subcaptionbox{Netscience数据\cite{boccaletti2006complex},其中节点数为379,边数为914\label{fig:chap05netscienceDiffModelCampare}}
	%标题的长度,超过则会换行,如下一个小图。
	{\includegraphics[width=7.3cm]{./chap5/netscienceDiffModelCampare}}
	%\hspace{3em}%
	\subcaptionbox{Email数据\cite{guimera2003self},其中节点数为1133,边数为5451\label{fig:chap05emailDiffModelCampare}}
	{\includegraphics[width=7.3cm]{./chap5/emailDiffModelCampare}}
	
	\caption{ISSD模型与IC、LT模型运行DegreeHeuristic算法结果比较}
\end{figure}

%\begin{figure}[H]
%	\centering
%	\includegraphics[scale=0.5]{./chap5/netscienceDiffModelCampare}
%	\caption{Netscience数据\cite{xie2006social},其中节点数为379,边数为914}
%	\label{fig:chap05netscienceDiffModelCampare}
%\end{figure}
%
%\begin{figure}[H]
%	\centering
%	\includegraphics[scale=0.5]{./chap5/emailDiffModelCampare}
%	\caption{Email数据\cite{xie2006social},其中节点数为1133,边数为5451}
%	\label{fig:chap05emailDiffModelCampare}
%\end{figure}
从图~\ref{fig:chap05netscienceDiffModelCampare}与图~\ref{fig:chap05emailDiffModelCampare},我们可以看到DegreeHeuristic算法在IC模型与LT模型运行结果相似,随着种子节点的加入,影响力逐步增加,曲线变化趋势相对较稳定。而算法运行结果在ISSD模型中出现波动,随着种子节点的增加,影响力范围反而减少现象,这与我们前面在悲惨世界网络中看到的现象相符。利用已有的IC模型和LT模型对复杂的信息传播过程进行分析是不完全的,而我们所提出ISSD模型特别针对信息传播考虑了节点自身因素、外部环境特点及时间因素等,对于分析信息传播更加具有针对性,为信息传播这个热点和难点问题提供了新模型,为探索信息传播规律打下了基础,希望能对各研究领域给予启发。

\section{影响力最大化算法}

\subsection{KClique Heuristic算法}
现实社会中信息总是在各类社交圈中传播,人们总是由于成长阶段、工作学习、兴趣爱好、生活地域等原因,形成一个又一个关系紧密的社交圈。这与我们社交网络形成的网络关系基本相符,你会有班级同学圈,工作同事圈,娱乐伙伴圈等等社交圈,你认识的这些人,他们之间也很可能都相互认识,也可能不认识。从你的角度来看,你是联系这些人的Ego节点。,我们认为最具有影响力的节点是能最大化联系整个社交网络,跨越多个圈子的人,类似于整个复杂网络中的结构洞(Structural holes)\cite{walker1997social,ahuja2000collaboration,burt2009structural},对于这些圈子中有重叠社区划分是研究信息传播的关键基础。
有重叠社区划分最早是由G.Palla等人2005年在《Nature》中提出K-clique社区划分算法CMP\cite{palla2005uncovering},他们定义社区内的任意节点应与多数节点连接,即一个社区可以表示为多个子图的并集,子图要求其内部节点之间必须有联系。在数学上定义一个完全连接的全连接图为K-clique,其中$K$为节点的数目,邻接K-clique是指两个K-clique至少共享$k-1$个节点。简单举例描述如图~\ref{fig:chap05CliqueExample}所示。
\begin{figure}[H]
	\centering%
	\subcaptionbox{K-clique网络关系例图\label{fig:chap05CliqueExample1}}
	%标题的长度,超过则会换行,如下一个小图。
	{\includegraphics[width=7.3cm]{./chap5/CliqueExample1}}
	%\hspace{3em}%
	\subcaptionbox{K-clique网络关系例图中K=3的部分社区划分\label{fig:chap05CliqueExample2}}
	{\includegraphics[width=7.3cm]{./chap5/CliqueExample2}}
	
	\caption{K-clique社区网络划分图例}
	\label{fig:chap05CliqueExample}
\end{figure}
我们提出的KClique Heuristic算法的基本思路是:首先对整个社交网络进行K-clique划分,然后找出Top-K个在所有有重叠划分的社区中出现次数最多的节点集合。其中K是控制社区内部紧密程度的参数。具体描述如算法~\ref{KCliqueHeuristic}所示:
\begin{algorithm}
	\caption{KCliqueHeuristic(G,s)}
	\label{KCliqueHeuristic}
	\begin{algorithmic}[1]
		\STATE initialize $S = \phi$,$dict =\{ \} $
		\STATE KC = KCliqueCommunity(G,k)
		\FOR {each vertex $v \in V$}
		\FOR {each clique $kc_i \in KC$}
		\IF {$v \in kc_i$}
		\STATE dict[v] += 1
		\ENDIF
		\ENDFOR
		\ENDFOR
		\STATE  $S$ = argmaxTop(dict[v])[1:s]
		\RETURN $S$
	\end{algorithmic}
\end{algorithm}

应用第~\ref{cha:4thChap04}章所提到的相关算法,与我们提出的KClique Heuristic算法,在IC模型、LT模型及ISSD模型中进行了影响力最大化实验比较。在IC模型模拟传播过程中,我们主要分析了传播概率较小的情况,节点传播概率p=0.01时传播过程的运行效果及时间。

\begin{figure}[H]
	\centering%
	\subcaptionbox{Blogs网络\label{fig:BlogsDiifAlgIISSDclique}}
	%标题的长度,超过则会换行,如下一个小图。
	{\includegraphics[width=7.3cm]{./chap5/BlogsDiifAlgIISSDclique}}
	%\hspace{3em}%
	\subcaptionbox{Netscience网络\label{fig:netscienceDiifAlgISSDclique}}
	{\includegraphics[width=7.3cm]{./chap5/netscienceDiifAlgISSDclique}}
	
	\caption{ISSD模型下不同算法运行结果分析比较}
\end{figure}

\begin{figure}[H]
	\centering%
	\subcaptionbox{Blogs网络\label{fig:BlogsDiifAlgICclique}}
	%标题的长度,超过则会换行,如下一个小图。
	{\includegraphics[width=7.3cm]{./chap5/BlogsDiifAlgICclique}}
	%\hspace{3em}%
	\subcaptionbox{Netscience网络\label{fig:NetscienceDiifAlgICclique}}
	{\includegraphics[width=7.3cm]{./chap5/NetscienceDiifAlgICclique}}
	
	\caption{IC模型下不同算法运行结果分析比较}
\end{figure}

\begin{figure}[H]
	\centering%
	\subcaptionbox{Blogs网络\label{fig:BlogsDiifAlgLTclique}}
	%标题的长度,超过则会换行,如下一个小图。
	{\includegraphics[width=7.3cm]{./chap5/BlogsDiifAlgLTclique}}
	%\hspace{3em}%
	\subcaptionbox{Netscience网络\label{fig:NetscienceDiifAlgLTclique}}
	{\includegraphics[width=7.3cm]{./chap5/NetscienceDiifAlgLTclique}}
	
	\caption{LT模型下不同算法运行结果分析比较}
\end{figure}

从图~\ref{fig:BlogsDiifAlgIISSDclique}到图~\ref{fig:netscienceDiifAlgISSDclique},可以看出我们所提出的KClique Heuristic算法,在ISSD模型中运行效果要好于其它算法,且在IC模型与LT模型中运行效果也与其它算法相似。通过实验,可以看出我们的模型同样可以应用于其他类型的复杂网络中,同时算法适应性较好。与算法设想:最具有影响力的节点是那些能最大化联系整个社交网络社区的节点基本一致。

\subsection{Community Leader Heuristic算法}%说明算法communityDegree,
%Lesmis diff community alg graph
常言道“物以类聚,人以群分”,对于无重叠社区划分所遵循的常理是社区内节点联系紧密,而社区之间联系松散。通常人们更加相信关系紧密的圈内信息,而不是其它圈外其他来源的信息。这里以《悲惨世界》人物关系网络为例进行说明。如图~\ref{fig:chap05lesmis}所示,主要人物是,主人公Jean Valjean(冉阿让),探长Javert(贾维),神父Bishop Myriel(米里哀),Eponine(爱潘妮),女工Fantine(芳汀)及其女儿Cosette(珂赛特)。如何仅仅从节点关系度来划分,这与作者小说想描述的人物重点是有出入的,如Eponine(爱潘妮)的弟弟Gavroche(加夫罗契)就比Myriel(米里哀)联系人要多,也就是度数要大。即所以传统以节点度数作为影响力最大化的方法是存在一定的缺陷的。
\begin{figure}[H]
	\centering
	\includegraphics[scale=0.45]{./chap5/lesmis-info}
	\caption{Infomaps算法\cite{rosvall2008maps}社区划分悲惨世界人物关系网络图}
	\label{fig:chap05lesmis}
\end{figure}
另一方面,目前分析信息传播过程都以个体或节点为单位粒度进行分析,而现实社交网络中注册用户一般都在$10^8$以上,用户关系网络复杂。如果特别是针对整个用户进行敏感信息监管几乎是不可能的。我们知道论坛、微博中普遍存在少数活跃用户推动大多数信息的传播,就是少数用户与多数用户有交互联系,20/80原理同样存在于社交网络上的信息传播规律。
针对此类问题我们提出了社区意见领袖启发式(Community Leader Heuristic)CLH模型,
从每一个社区中选择意见领袖\cite{katz1957two}作为传播节点,模型示意图如图~\ref{fig:chap05communtyLeaderToNode}所示。每个社区内的意见领袖的可信度要高于社区处人员,意见领袖的选择有利于信息的传播。
\begin{figure}[H]
	\centering
	\includegraphics[scale=0.45]{./chap5/chap05communtyLeaderToNode}
	\caption{社区意见领袖启发式算法示意图}
	\label{fig:chap05communtyLeaderToNode}
\end{figure}
我们可以根据不同的需求,对网络进行划分成不同层次的网络社区。通过CLH模型分析图~\ref{fig:chap05communtyLeaderToNode},可以得到社区意见领袖节点[4,7,9,14],我们也可以进一步计算划分[4,7,9,14],得出新粒度的意见领袖为节点[9]。从相反的角度分析,可以在大社团中再次划分成多个小的社团,从每个小社团中发现其意见领袖,以小粒度分析信息传播过程。如图~\ref{fig:Chap05blogsCommunityToNode}所示,对Blogs网络中$n=3982$个节点,$m=6803$条边以无重叠社区划分的新粒度网络关系图。这样我们可以从中观层次了解整个网络关系,了解信息如何在社区(圈子)内传播,而不至于陷入庞大、不可视的复杂关系中,从而可为信息监控提供依据。
\begin{figure}[H]
	\centering
	\includegraphics[scale=0.50]{./chap5/Chap05blogsCommunityToNode}
	\caption{Blogs网络根据CLH形成的新网络关系图}
	\label{fig:Chap05blogsCommunityToNode}
\end{figure}
因此可以根据实际需要划分成不同粒度的信息传播模型,或信息监控对象。这个常理与我们现实生活中的行政区域化分及行政区域的行政领导的模式相似,目前很多广告也是根据不同的地区,选择不同的区域名人做广告。

我们应用CLH模型到影响力最大化问题中,社区意见领袖启发式(Community Leader Heuristic)CLH算法如下:
\begin{algorithm}
	\caption{communtiyLeaderHeuristic(G,k)}
	\label{communityIM}
	\begin{algorithmic}[1]
		\STATE initialize $S = \phi$
		\STATE $C=communtiyDetect(G)$
		\FOR {each community $c_i \in C$}
		\STATE $subGraph_{i}$ = $k * c_i/n$
		\STATE $subC_i = communtiyDetect(subGraph_{i})$
		\FOR {each subCommunity $c_j \in subC_i $}
		\STATE select $s_{leader}$ = argmaxLeader($c_j$)
		\STATE $S = S \cup s_{leader} $ 
		\ENDFOR
		\ENDFOR
		\RETURN $S$
	\end{algorithmic}
\end{algorithm}

我们应用章节~\ref{cha:4thChap04}所提出的相关算法,与我们提出的KClique Heuristic算法、CLH算法,在~\ref{cha:secondChap02}中所提到的IC模型、LT模型及我们所提出的ISSD模型,对我们所提出的算法及模型进行了最大影响力最大化算法分析比较。在IC模型模拟传播过程中,我们主要分析了传播概率较小的情况,节点传播概率p=0.01(如果节点的传播能力很强,很难区分单个个体的重要性)时传播过程的运行效果及时间。
数据集我们采用Blogs网络和email网络数据。
\begin{figure}[H]
	\centering%
	\subcaptionbox{Blogs网络\label{fig:BlogsDiifAlgIISSDclt}}
	%标题的长度,超过则会换行,如下一个小图。
	{\includegraphics[width=7.3cm]{./chap5/BlogsDiifAlgIISSDclt}}
	%\hspace{3em}%
	\subcaptionbox{Netscience网络\label{fig:netscienceDiifAlgISSD}}
	{\includegraphics[width=7.3cm]{./chap5/netscienceDiifAlgISSD}}
	
	\caption{ISSD模型下不同算法运行结果分析比较}
\end{figure}

\begin{figure}[H]
	\centering%
	\subcaptionbox{Blogs网络\label{fig:BlogsDiifAlgICLT}}
	%标题的长度,超过则会换行,如下一个小图。
	{\includegraphics[width=7.3cm]{./chap5/BlogsDiifAlgICLT}}
	%\hspace{3em}%
	\subcaptionbox{Email网络\label{fig:NetscienceDiifAlgICLT}}
	{\includegraphics[width=7.3cm]{./chap5/NetscienceDiifAlgICLT}}
	
	\caption{IC模型下不同算法运行结果分析比较}
\end{figure}

\begin{figure}[H]
	\centering%
	\subcaptionbox{Blogs网络\label{fig:BlogsDiifAlgLTCLT}}
	%标题的长度,超过则会换行,如下一个小图。
	{\includegraphics[width=7.3cm]{./chap5/BlogsDiifAlgLTCLT}}
	%\hspace{3em}%
	\subcaptionbox{Netscience网络\label{fig:NetscienceDiifAlgLTCLT}}
	{\includegraphics[width=7.3cm]{./chap5/NetscienceDiifAlgLTCLT}}
	
	\caption{LT模型下不同算法运行结果分析比较}
\end{figure}

从图~\ref{fig:BlogsDiifAlgIISSDclt}到图~\ref{fig:netscienceDiifAlgISSD},可以看出我们所提出的CLT算法、KClique Heuristic算法,在ISSD模型中运行效果要好于已有算法,且在IC模型与LT模型中运行效果也与经典算法相似。这与我们的算法思路:每个社区内的意见领袖更有利于信息传播相符。而且CLT算法可以根据实际需要从不同的粒度进行信息传播,能更好适用于信息监控。

\section{本章小结}
在本章中,我们深入研究了信息传播过程,分析了信息传播随时间变化特点,并参照“牛顿冷却定律”提出了适应信息传播的时间变化模型。同时考虑了接收信息个体自身内部属性及所在Ego网络中传播信息状态这一外部属性,按照其与传播信息邻居的关系紧密程度从连通性,无重叠社区,K-truesses子图,有重叠社区Kclique子图等4种结构多样性特点进行了分析。我们给出了符合信息传播的5个假设;给出了信息传播机理,即信息传播过程中未知信息状态(\underline{\textbf{U}}nknown)、得知信息状态(\underline{\textbf{K}}nown)、传播信息状态(\underline{\textbf{A}}pproved)和信息疲惫态(\underline{\textbf{E}}xhausted)等4种信息传播状态类型;提出了ISSD信息传播结构多样化模型,并对整个变化过程的进行了形式化定义和描述。实验分析了信息传播过程中时间影响、结构多样化影响特点。最后,我们结合前面影响力最大化方法,提出了符合信息传播的两个最大化算法,KClique Heuristic算法和CLH算法。实验证明两个算法不但在ISSD模型中效果突出,而且同样也适用于IC模型与LT模型,运行效果与经典算法相似,说明了算法的通用性和适用性。总之,ISSD模型的提出使人们分析信息传播过程及阶段状态有了清晰的认识,并可以进行量化,这方便了信息传播过程的理论研究。希望能在疾病预防、网络安全、市场营销、广告投放、舆情分析、信息监控等实际热点领域中得以应用,期望能给相关研究人员给予启发。
%同时我们的模型也可成为重要热点人物发现,热点团体发现的研究基础。这为信息的研究提供了新的思路和方法。
% 下面这句用以支持中文
% !Mode:: "TeX:UTF-8"

%%% Local Variables:
%%% mode: latex
%%% TeX-master: t
%%% End:

\chapter{总结与展望}
\label{cha:5thChap}

\section{论文总结}
互联网技术日异月新,发展非常之快,目前基于社交网络的应用得到了很好的开发。人们在生活中逐渐地与社交网络联系在一起,可以分享,传播,转发各种各样的关于旅游,关于食品,关于政治等等一系列的事情。互联网深入生活给我们带来了很多的便利,我们可以在任何地方得知世界上正在发生的任何新闻。

从大数据分析中获得有用信息是一个十分活跃的研究热点,而现在网络中社交网络平台海量的用户发布海量的信息可以为我们进行大数据研究提供海量的数据基础。影响力分析在近十年来一直被广泛研究,从传播模型,分析算法,社交网络图等等各方面都有很多学者进行新的研究或者对现有的成果进行改进。下面分以下几个方面对本文进行总结:
\begin{itemize}
\item 我们对社交网络平台进行了相应的调研,对影响力最大化相关课题也进行了深入的了解,掌握了目前主要的研究方法和经典的模型,分析讨论了目前研究存在的问题,并根据问题提出解决方案。
\item 通过分析,我们了解到目前一个很重要的但是没有得到很好的解决研究问题是对网络图中结点进行价值估计,现有的研究中基本是以单位价值来进行研究的,虽然也有其他价值估计的方法(如随机赋值),但都缺乏说服力。并且造成构建顶点估值模型困难的以个原因是很难获得整个网络中的所有结点现实生活中的实际价值,从而从理论上去分析创建合理并可行的算法模型。本文在这一方面主要基于PageRank算法的思路,认为一个重要的结点必然其价值越高,所以提出了PRBC(PageRank Based Cost Model)价值估计模型,并利用PRBC估值模型提出了有预算的影响力最大化算法BRMDN。
\item 进一步我们发现目前研究主要还是集中在单一信源的影响力最大化分析,很多的研究者在此方面都作了大量的工作,也提出了很多比较经典的传播模型(如IC,LT等模型),影响力最大化算法(如Greedy, DegreeDiscount,CELF等算法),但是只有部分的研究涉及多信源的影响力最大化分析。所以本文另一方面的贡献在于提出了竞争环境下的影响力分析一种可行的传播模型XLT4C(eXtended Linear Threshold for Competition),以及基于该模型的算法LG4C,LDH4C。
\item 对于我们提出的模型以及算法,我们在真实的数据集上进行了大量实验,实验结果表明我们所提出的传播模型,顶点估值模型,影响力最大化分析算法都获得了很好的效果。
\end{itemize}

\section{未来研究}
首先,社会是个互相竞争的舞台,然后我们在存在竞争的影响力最大化分析上做的研究还非常有限,其次在缺乏具体的价值数据情况下对顶点估值模型进行准确化也是一个比较有挑战的问题。

所以基于以上的客观事实,并且研究的时间有限很多问题还没有进行优化,所以我们未来可以在本文的基础上对竞争型影响力分析上提出更好的传播模型,并用数学方法严谨地给出相应成立的条件,以及希望能给出一些比较重要的结论或者一些经验性结果。同样我们可以多从网络中发掘有用关于价值估算的数据,并对其进行科学的分析,然后可以得出一些有实际利用价值的模型和方案。总的来说,未来研究就主要集中于两点:(1)进一步研究竞争环境下的影响力最大化;(2)对网络价值估算的模型进行优化。

%%% 其它部分
\backmatter

%% 本科生要这几个索引,研究生不要。选择性留下。
%\makeatletter
%\ifthu@bachelor
%  % 插图索引
%  \listoffigures
%  % 表格索引
%  \listoftables
%  % 公式索引
%  %\listofequations
%\fi
%\makeatother


% 参考文献
\bibliographystyle{thubib}
\bibliography{ref/refs}


% 致谢
%%% Local Variables:
%%% mode: latex
%%% TeX-master: "../main"
%%% End:

\begin{ack}
  工作十年后重返学校攻读博士学位,这需要一种勇气。而对于能来清华园学习深造,是多少我等学子的梦想,兴奋之余面对的是更大的压力与挑战,求学之路任重道远。
  
  5年来,当我终于在漫长的博士论文选题、构思、反复的修改后要为论文划上句号时。我衷心感谢导师邢春晓研究员对本人的精心指导。您思路活跃敏捷,学术严谨,工作敬业,精力充沛,您的言传身教将使我终生受益。衷心感谢张勇副研究员的精心指导,与您无数次的讨论使我理清思路,发现不足,您对我在学术上的进步给出了极大的帮助。
  
  非常感谢清华大学信息技术研究院Web软件与技术中心的全体老师和同学们,你们用爱心、热情与活力铸造了一个和谐团结的大家庭,使我度过了非常快乐和美好的5年时光,一想到这些心里就暖暖的,充满了正能量,再次感谢你们。
  
  最后,我要感谢我的家人,你们自始至终对我绝对的支持和帮助,感谢你们对我的包容和理解,这种无私的情怀甚至无法用语言来表述,我只能心存感激并继续前行。

  一个简短的致谢不足以涵盖在过去的5年中所有帮助、激励过我的家人及朋友,衷心的祝愿您们好人一生平安!
\end{ack}


% 附录
%\begin{appendix}
%%%% Local Variables: 
%%% mode: latex
%%% TeX-master: "../main"
%%% End: 

\chapter{外文资料原文}
\label{cha:engorg}
As one of the most widely used techniques in operations research, {\em
  mathematical programming} is defined as a means of maximizing a quantity known
as {\em objective function}, subject to a set of constraints represented by
equations and inequalities. Some known subtopics of mathematical programming are
linear programming, nonlinear programming, multiobjective programming, goal
programming, dynamic programming, and multilevel programming$^{[1]}$.

It is impossible to cover in a single chapter every concept of mathematical
programming. This chapter introduces only the basic concepts and techniques of
mathematical programming such that readers gain an understanding of them
throughout the book$^{[2,3]}$.


\section{Single-Objective Programming}
The general form of single-objective programming (SOP) is written
as follows,
\begin{equation}\tag*{(123)} % 如果附录中的公式不想让它出现在公式索引中,那就请
                             % 用 \tag*{xxxx}
\left\{\begin{array}{l}
\max \,\,f(x)\\[0.1 cm]
\mbox{subject to:} \\ [0.1 cm]
\qquad g_j(x)\le 0,\quad j=1,2,\cdots,p
\end{array}\right.
\end{equation}
which maximizes a real-valued function $f$ of
$x=(x_1,x_2,\cdots,x_n)$ subject to a set of constraints.

\newtheorem{mpdef}{Definition}[chapter]
\begin{mpdef}
In SOP, we call $x$ a decision vector, and
$x_1,x_2,\cdots,x_n$ decision variables. The function
$f$ is called the objective function. The set
\begin{equation}\tag*{(456)} % 这里同理,其它不再一一指定。
S=\left\{x\in\Re^n\bigm|g_j(x)\le 0,\,j=1,2,\cdots,p\right\}
\end{equation}
is called the feasible set. An element $x$ in $S$ is called a
feasible solution.
\end{mpdef}

\newtheorem{mpdefop}[mpdef]{Definition}
\begin{mpdefop}
A feasible solution $x^*$ is called the optimal
solution of SOP if and only if
\begin{equation}
f(x^*)\ge f(x)
\end{equation}
for any feasible solution $x$.
\end{mpdefop}

One of the outstanding contributions to mathematical programming was known as
the Kuhn-Tucker conditions\ref{eq:ktc}. In order to introduce them, let us give
some definitions. An inequality constraint $g_j(x)\le 0$ is said to be active at
a point $x^*$ if $g_j(x^*)=0$. A point $x^*$ satisfying $g_j(x^*)\le 0$ is said
to be regular if the gradient vectors $\nabla g_j(x)$ of all active constraints
are linearly independent.

Let $x^*$ be a regular point of the constraints of SOP and assume that all the
functions $f(x)$ and $g_j(x),j=1,2,\cdots,p$ are differentiable. If $x^*$ is a
local optimal solution, then there exist Lagrange multipliers
$\lambda_j,j=1,2,\cdots,p$ such that the following Kuhn-Tucker conditions hold,
\begin{equation}
\label{eq:ktc}
\left\{\begin{array}{l}
    \nabla f(x^*)-\sum\limits_{j=1}^p\lambda_j\nabla g_j(x^*)=0\\[0.3cm]
    \lambda_jg_j(x^*)=0,\quad j=1,2,\cdots,p\\[0.2cm]
    \lambda_j\ge 0,\quad j=1,2,\cdots,p.
\end{array}\right.
\end{equation}
If all the functions $f(x)$ and $g_j(x),j=1,2,\cdots,p$ are convex and
differentiable, and the point $x^*$ satisfies the Kuhn-Tucker conditions
(\ref{eq:ktc}), then it has been proved that the point $x^*$ is a global optimal
solution of SOP.

\subsection{Linear Programming} 
\label{sec:lp}

If the functions $f(x),g_j(x),j=1,2,\cdots,p$ are all linear, then SOP is called
a {\em linear programming}.

The feasible set of linear is always convex. A point $x$ is called an extreme
point of convex set $S$ if $x\in S$ and $x$ cannot be expressed as a convex
combination of two points in $S$. It has been shown that the optimal solution to
linear programming corresponds to an extreme point of its feasible set provided
that the feasible set $S$ is bounded. This fact is the basis of the {\em simplex
  algorithm} which was developed by Dantzig as a very efficient method for
solving linear programming.
\begin{table}[ht]
\centering
  \centering
  \caption*{Table~1\hskip1em This is an example for manually numbered table, which
    would not appear in the list of tables}
  \label{tab:badtabular2}
  \begin{tabular}[c]{|c|m{0.8in}|c|c|c|c|c|}\hline
    \multicolumn{2}{|c|}{Network Topology} & \# of nodes & 
    \multicolumn{3}{c|}{\# of clients} & Server \\\hline
    GT-ITM & Waxman Transit-Stub & 600 &
    \multirow{2}{2em}{2\%}& 
    \multirow{2}{2em}{10\%}& 
    \multirow{2}{2em}{50\%}& 
    \multirow{2}{1.2in}{Max. Connectivity}\\\cline{1-3}
    \multicolumn{2}{|c|}{Inet-2.1} & 6000 & & & &\\\hline
    \multirow{2}{1in}{Xue} & Rui  & Ni &\multicolumn{4}{c|}{\multirow{2}*{\thuthesis}}\\\cline{2-3}
    & \multicolumn{2}{c|}{ABCDEF} &\multicolumn{4}{c|}{} \\\hline
\end{tabular}  
\end{table}

Roughly speaking, the simplex algorithm examines only the extreme points of the
feasible set, rather than all feasible points. At first, the simplex algorithm
selects an extreme point as the initial point. The successive extreme point is
selected so as to improve the objective function value. The procedure is
repeated until no improvement in objective function value can be made. The last
extreme point is the optimal solution.

\subsection{Nonlinear Programming}

If at least one of the functions $f(x),g_j(x),j=1,2,\cdots,p$ is nonlinear, then
SOP is called a {\em nonlinear programming}.

A large number of classical optimization methods have been developed to treat
special-structural nonlinear programming based on the mathematical theory
concerned with analyzing the structure of problems.
\begin{figure}[h]
  \centering
  \includegraphics[clip]{thu-lib-logo}
  \caption*{Figure~1\hskip1em This is an example for manually numbered figure,
    which would not appear in the list of figures}
  \label{tab:badfigure2}    
\end{figure}

Now we consider a nonlinear programming which is confronted solely with
maximizing a real-valued function with domain $\Re^n$.  Whether derivatives are
available or not, the usual strategy is first to select a point in $\Re^n$ which
is thought to be the most likely place where the maximum exists. If there is no
information available on which to base such a selection, a point is chosen at
random. From this first point an attempt is made to construct a sequence of
points, each of which yields an improved objective function value over its
predecessor. The next point to be added to the sequence is chosen by analyzing
the behavior of the function at the previous points. This construction continues
until some termination criterion is met. Methods based upon this strategy are
called {\em ascent methods}, which can be classified as {\em direct methods},
{\em gradient methods}, and {\em Hessian methods} according to the information
about the behavior of objective function $f$. Direct methods require only that
the function can be evaluated at each point. Gradient methods require the
evaluation of first derivatives of $f$. Hessian methods require the evaluation
of second derivatives. In fact, there is no superior method for all
problems. The efficiency of a method is very much dependent upon the objective
function.

\subsection{Integer Programming}

{\em Integer programming} is a special mathematical programming in which all of
the variables are assumed to be only integer values. When there are not only
integer variables but also conventional continuous variables, we call it {\em
  mixed integer programming}. If all the variables are assumed either 0 or 1,
then the problem is termed a {\em zero-one programming}. Although integer
programming can be solved by an {\em exhaustive enumeration} theoretically, it
is impractical to solve realistically sized integer programming problems. The
most successful algorithm so far found to solve integer programming is called
the {\em branch-and-bound enumeration} developed by Balas (1965) and Dakin
(1965). The other technique to integer programming is the {\em cutting plane
  method} developed by Gomory (1959).

\hfill\textit{Uncertain Programming\/}\quad(\textsl{BaoDing Liu, 2006.2})

\section*{References}
\noindent{\itshape NOTE: these references are only for demonstration, they are
  not real citations in the original text.}

\begin{enumerate}[{$[$}1{$]$}]
\item Donald E. Knuth. The \TeX book. Addison-Wesley, 1984. ISBN: 0-201-13448-9
\item Paul W. Abrahams, Karl Berry and Kathryn A. Hargreaves. \TeX\ for the
  Impatient. Addison-Wesley, 1990. ISBN: 0-201-51375-7
\item David Salomon. The advanced \TeX book.  New York : Springer, 1995. ISBN:0-387-94556-3
\end{enumerate}

\chapter{外文资料的调研阅读报告或书面翻译}
\section{单目标规划}
北冥有鱼,其名为鲲。鲲之大,不知其几千里也。化而为鸟,其名为鹏。鹏之背,不知其几
千里也。怒而飞,其翼若垂天之云。是鸟也,海运则将徙于南冥。南冥者,天池也。 
\begin{equation}\tag*{(123)}
 p(y|\mathbf{x}) = \frac{p(\mathbf{x},y)}{p(\mathbf{x})}=
\frac{p(\mathbf{x}|y)p(y)}{p(\mathbf{x})}
\end{equation}

吾生也有涯,而知也无涯。以有涯随无涯,殆已!已而为知者,殆而已矣!为善无近名,为
恶无近刑,缘督以为经,可以保身,可以全生,可以养亲,可以尽年。

\subsection{线性规划}
庖丁为文惠君解牛,手之所触,肩之所倚,足之所履,膝之所倚,砉然响然,奏刀騞然,莫
不中音,合于桑林之舞,乃中经首之会。
\begin{table}[ht]
\centering
  \centering
  \caption*{表~1\hskip1em 这是手动编号但不出现在索引中的一个表格例子}
  \label{tab:badtabular3}
  \begin{tabular}[c]{|c|m{0.8in}|c|c|c|c|c|}\hline
    \multicolumn{2}{|c|}{Network Topology} & \# of nodes & 
    \multicolumn{3}{c|}{\# of clients} & Server \\\hline
    GT-ITM & Waxman Transit-Stub & 600 &
    \multirow{2}{2em}{2\%}& 
    \multirow{2}{2em}{10\%}& 
    \multirow{2}{2em}{50\%}& 
    \multirow{2}{1.2in}{Max. Connectivity}\\\cline{1-3}
    \multicolumn{2}{|c|}{Inet-2.1} & 6000 & & & &\\\hline
    \multirow{2}{1in}{Xue} & Rui  & Ni &\multicolumn{4}{c|}{\multirow{2}*{\thuthesis}}\\\cline{2-3}
    & \multicolumn{2}{c|}{ABCDEF} &\multicolumn{4}{c|}{} \\\hline
\end{tabular}  
\end{table}

文惠君曰:“嘻,善哉!技盖至此乎?”庖丁释刀对曰:“臣之所好者道也,进乎技矣。始臣之
解牛之时,所见无非全牛者;三年之后,未尝见全牛也;方今之时,臣以神遇而不以目视,
官知止而神欲行。依乎天理,批大郤,导大窾,因其固然。技经肯綮之未尝,而况大坬乎!
良庖岁更刀,割也;族庖月更刀,折也;今臣之刀十九年矣,所解数千牛矣,而刀刃若新发
于硎。彼节者有间而刀刃者无厚,以无厚入有间,恢恢乎其于游刃必有余地矣。是以十九年
而刀刃若新发于硎。虽然,每至于族,吾见其难为,怵然为戒,视为止,行为迟,动刀甚微,
謋然已解,如土委地。提刀而立,为之而四顾,为之踌躇满志,善刀而藏之。”

文惠君曰:“善哉!吾闻庖丁之言,得养生焉。”


\subsection{非线性规划}
孔子与柳下季为友,柳下季之弟名曰盗跖。盗跖从卒九千人,横行天下,侵暴诸侯。穴室枢
户,驱人牛马,取人妇女。贪得忘亲,不顾父母兄弟,不祭先祖。所过之邑,大国守城,小
国入保,万民苦之。孔子谓柳下季曰:“夫为人父者,必能诏其子;为人兄者,必能教其弟。
若父不能诏其子,兄不能教其弟,则无贵父子兄弟之亲矣。今先生,世之才士也,弟为盗
跖,为天下害,而弗能教也,丘窃为先生羞之。丘请为先生往说之。”
\begin{figure}[h]
  \centering
  \includegraphics{hello}
  \caption*{图~1\hskip1em 这是手动编号但不出现索引中的图片的例子}
  \label{tab:badfigure3}    
\end{figure}

柳下季曰:“先生言为人父者必能诏其子,为人兄者必能教其弟,若子不听父之诏,弟不受
兄之教,虽今先生之辩,将奈之何哉?且跖之为人也,心如涌泉,意如飘风,强足以距敌,
辩足以饰非。顺其心则喜,逆其心则怒,易辱人以言。先生必无往。”

孔子不听,颜回为驭,子贡为右,往见盗跖。

\subsection{整数规划}
盗跖乃方休卒徒大山之阳,脍人肝而餔之。孔子下车而前,见谒者曰:“鲁人孔丘,闻将军
高义,敬再拜谒者。”谒者入通。盗跖闻之大怒,目如明星,发上指冠,曰:“此夫鲁国之
巧伪人孔丘非邪?为我告之:尔作言造语,妄称文、武,冠枝木之冠,带死牛之胁,多辞缪
说,不耕而食,不织而衣,摇唇鼓舌,擅生是非,以迷天下之主,使天下学士不反其本,妄
作孝弟,而侥幸于封侯富贵者也。子之罪大极重,疾走归!不然,我将以子肝益昼餔之膳。”


\chapter{其它附录}
前面两个附录主要是给本科生做例子。其它附录的内容可以放到这里,当然如果你愿意,可
以把这部分也放到独立的文件中,然后将其 \verb|\input| 到主文件中。
%\end{appendix}

% 个人简历
\begin{resume}

  \resumeitem{个人简历}

  1990年8月28日出生于江西省上饶市。
  
  2009年9月考入中南大学软件工程专业,2013年7月本科毕业并获得工学学士学位。
  
  2013年9月考入清华大学计算机科学与技术系攻读工学硕士学位至今。

  \resumeitem{发表的学术论文} % 发表的和录用的合在一起

  \begin{enumerate}[{[}1{]}]
  \item Xinhui Xu, Yong Zhang, Qingcheng Hu, Chao Li, Chunxiao Xing. A Balanced Method for Budgeted Influence Maximization. The 27th International Conference on Software Engineering and Knowledge Engineering (SEKE2015).
  
  \end{enumerate}
 
  %\resumeitem{获奖情况} % 有就写,没有就删除
  
  %\begin{enumerate}[{[}1{]}]
  %	\item 2013年度 清华大学信息技术研究院二等奖学金(一等奖空缺) 
  %	\item 2013年度 清华大学综合三等奖学金
  %\end{enumerate}

  %\resumeitem{参与项目} % 有就写,没有就删除
  %\begin{enumerate}[{[}1{]}]
  %\item 973计划“面向复杂应用环境的数据存储系统理论与技术基础研究” 课题编号:2011CB302302
  %\item 863计划“心血管疾病大数据平台的构建和应用研究”  课题编号:SS2015AA020102
  %\item 863计划“支持数据驱动型应用的跨域共享与服务支撑平台” 课题编号:2009AA01Z143
  %\item 863计划“海底观测网岸基控制运行与数据管理系统” 课题编号:2012AA09A408
  %\end{enumerate}
\end{resume}

\end{document}
