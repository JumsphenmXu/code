
%%% Local Variables:
%%% mode: latex
%%% TeX-master: t
%%% End:
\secretlevel{绝密} \secretyear{2100}

\ctitle{竞争条件下基于社交网络的影响力最大化算法研究及其实现}
% 根据自己的情况选,不用这样复杂
\makeatletter
\ifthu@bachelor\relax\else
  \ifthu@doctor
    \cdegree{工学博士}
  \else
    \ifthu@master
      \cdegree{工学硕士}
    \fi
  \fi
\fi
\makeatother


\cdepartment[计算机]{计算机科学与技术系}
\cmajor{计算机科学与技术}
\cauthor{许信辉} 
\csupervisor{张勇副研究员}


\etitle{Competitive Influence Maximization over Social Network} 

\edegree{Master of Science} 
\emajor{Computer Science and Technology} 
\eauthor{XU Xinhui} 
\esupervisor{Associate Professor ZHANG Yong} 

\begin{cabstract}
互联网技术的日渐成熟,社交网络的蓬勃发展,使得越来越多的人开始加入互联网发布消息,订阅新闻,转发推文,评论热点问题。大量增长的用户带来了海量有用的数据,而研究分析这些数据可以为我们的决策提供十分有价值线索。影响力分析很早就开始为人们所注意并将其利用于产品推广,广告投放,舆论导向,政策宣传等目的,但是目前主要研究都集中于没有竞争的影响力分析,本文将依据双方针对相似资源的竞争来研究影响力的传播模型及相关的算法。本文主要的工作包括以下几个部分:
\begin{itemize}
\item 提出了一种新的网络结点代价估值模型PRBC(PageRank Based Cost Model)。
\item 扩展了经典的LT模型,提出了XLT4C(eXtended Linear Threshold for Competition Model)传播模型用于存在竞争的影响力分析领域。然后根据XLT4C模型提出了LDH4C(Local Degree Heuristic for Competition),LG4C(Local Greedy for Competition)算法。
\item 利用不同的数据集做了大量的实验对本文提出的估值模型,传播模型,影响力最大化算法进行分析验证。
\end{itemize}
\end{cabstract}

\ckeywords{社交网络,代价估值模型,竞争的影响力传播模型,影响力最大化算法}

\begin{eabstract} 
With Internet technology rapidly growing mature and social networks acquiring their great success, more and more people are now joining into the Internet to publish information, subscribe news, forward  tweets, and comment hot topics. The growing number of netizens bring us massive useful data which can be helpful to help us to make right decisions if we study them carefully. Influence analysis has been noticed for long time that it can be used for products promotion, ads distribution, public opinion transformation, policies advocation et al., but the as far as now, the attention has most been paid on the non-competitive influence analysis, and in this paper we will simulate the situation which two groups compete for their influence spread. The main contributions of this paper can be sumarized as follows:
\begin{itemize}
\item We put forward a new cost evaluation model called PageRank Based Cost Model(PRBC).
\item We extend the classical LT model to come up with eXtended Linear Threshold for Competition(XLT4C) which can be used in the competitive influence maxmization field. And then according to XLT4C, we propose two alogrithms to solve the competitive influence maxmization problem which we named Local Degree Heurisitc for Competition(LDH4C) and Local Greedy for Competition(LG4C).
\item We conducts a lot of experiments on different data sets to analysis the performance of our methods and prove the effectiveness of our methods. 
\end{itemize} 
\end{eabstract}

\ekeywords{Social Network, Cost Evaluation Model, Competitive Influence Spread Model, Influence Maximization Algorithm}
