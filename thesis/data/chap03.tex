% 下面这句用以支持中文
% !Mode:: "TeX:UTF-8"

%%% Local Variables:
%%% mode: latex
%%% TeX-master: t
%%% End:

\chapter{有预算限制的影响力最大化算法}
\label{cha:3thChap03}

\section{引言}
基于web或者移动设备的在线社交媒体网络蓬勃发展,例如国内的微信,博客,微博,国外的包括Facebook,Twitter等大型社交平台。人们每天都花大量的时间在社交网络上,其发布的短文、评论,图片等信息媒体也越容易被传播从而影响到别人。如何发现有很大影响力的人,并且让他们传播的产品,政治宣传,新思维能尽快地扩散到更广泛的人群中去,这种现象被称为影响力传播(Influence Propagation)。目前有很多研究者在做这方面的研究,但是他们的方法存在下面两方面的问题,
\begin{enumerate}
\item 需要全局了解整个网络的拓扑结构进行计算,这在用户越来越多,关系越来越复杂的现代社交网络中也越来越不容易,同时在运行时间以及传播效果上没有进行很好的平衡。
\item 对于选取网络中结点的代价,没有提出很好的顶点估值模型,一般都是赋予任何结点都是同样的值,这并不符合现实生活中不同影响力的人有不同身价的现象。
\end{enumerate}
本节基于以上两点,首先基于PageRank值并融合顶点度中心性性质,提出了PRBC(PageRank Based Cost)顶点估值模型,然后利用复杂网络中无标度网络的特点提出了BRMDN(Budgeted Random Maximal Degree Neighbor)算法。实验结果表明我们的算法能在时间性能和传播效果上达到比较好的平衡。


\section{相关工作介绍}
\subsection{传播模型}
本章节主要基于Kempe\cite{kempe2003maximizing}的IC、LT模型,具体传播过程可以参考\ref{sec:IC-model-desc}节IC模型示例和\ref{sec:LT-model-desc}节LT模型示例。


\subsection{顶点估值模型}
\label{sec:chap3:cost-model}
给定社交网络$G=(V, E)$,需要给每个顶点赋予一个价格,根据现实生活中对于影响力比较大的人给予的身价一般越高,也需要给予图中每个顶点相应不同的价值。L. Page\cite{page1999pagerank}等人提出了PageRank算法对网页根据重要程度进行排名,web网页由超链接连接起来,那么将每个网页都看作一个顶点,链接网页的超链接看作边,那么整个互联网就是一个大的网络图。鉴于社交网络也构建为图模型,所以同样可以利用PageRank算法对图中的顶点进行排名,然后进行一定的映射得出顶点的价值。


具体来说,利用PageRank算法计算结点的PageRank值,然后再利用增长因子和调和系数对原有的PageRank值进一步进行调整,考虑到PageRank算法中一个不太重要的点连接到一个非常重要的点也会被认为是重要的,从而提高了其PageRank值,所以再利用结点度数以及周围邻居中最大度数进行适当的缩放。下面对本文的顶点估值模型PRBC(PageRank Based Cost Model)进行形式化定义。
\begin{definition}
\emph{顶点估值模型}
给定社交网络$G=(V, E)$,一个预先定义好的增长因子$\delta$以及调和系数$\lambda$,那么对于任意结点$u \in V$,定义其在$PRBC$模型下价格数值为$PRBC(u)$,可以形式化表示为,

\begin{equation}
\label{eq:prbc}
PRBC(u) = \frac{\lambda(PR(u) + \delta) \mathcal{D}(u)}{\mathcal{D}(v_{max})},v_{max} = \argmax_{v \in \mathcal{N}(u) \cup \{u\}}\mathcal{D}(v)
\end{equation}

其中$PR(u)$为对整个图利用PageRank算法计算得到的结点$u$的PageRank数值,其物理意义代表了结点的重要度,而$\mathcal{D}(\cdot)$表示为结点度数的函数,式子\ref{eq:prbc}中$\mathcal{D}(u), \mathcal{D}(v)$表示结点$u, v$的度数,$\mathcal{N}(\cdot)$为邻居结点集函数,即$\mathcal{N}(u)$表示结点$u$的邻居结点集。
\end{definition}


\subsection{非常值估值函数下的影响力最大化}
\subsubsection{问题定义}
由于价值估值函数要反映现实生活中的情况,故而采取\ref{sec:chap3:cost-model}节的估值模型,这种情况下的影响力最大化问题可以称之为非常值估值函数下的影响力最大化$NCC-IM$(Non-Constant Cost Influence Maximization),这种比单位化估值函数情况下的问题要复杂一些,因为在预算$\mathcal{B}$的约束下,要找到一个集合$S$,使其满足以下式子\ref{eq:chap3:ncc-im-cond}是非常困难的。
\begin{equation}
\label{eq:chap3:ncc-im-cond}
|S|=K, \sum_{v \in S} \mathcal{CF}(v) = \mathcal{B}
\end{equation}


由于在$NCC-IM$问题中满足式子\ref{eq:chap3:ncc-im-cond}比较难,而且容易陷入无限循环中。所以需要采取一些策略对条件\ref{eq:chap3:ncc-im-cond}进行改造或者对式子\ref{eq:chap3:ncc-im-cond}进行松弛,使得算法容易达到终止条件,而不会因为不满足上述式子而陷入无限循环,下面给出$NCC-IM$问题改造后的定义。


\begin{definition}
\label{def:chap3:ncc-im}
\emph{(NCC-IM)}
给定社交网路图$G=(V, E)$,常数$K$,价值估值函数$\mathcal{CF}(\cdot)$,一定的资源预算$\mathcal{B}$,可接受预算的误差值$\varepsilon$,需要找出一个初始结点集合$S$使其满足如下条件
\begin{displaymath}
\sigma(S) = \argmax_{|S| \leq K \wedge S \subset V} Inf(S), \mathcal{B} - \varepsilon \leq \sum_{v \in S} \mathcal{CF}(v) \leq \mathcal{B} + \varepsilon
\end{displaymath}
其中$\sigma(S)$表示初始集合$S$最终获得影响力的结果,$Inf(\cdot)$是影响力传播函数。
\end{definition}


注意在定义\ref{def:chap3:ncc-im}中将条件松弛为$|S| \leq K$,所以在我们所采用的算法中,都在原来单位化估值函数模型下进行了条件松弛,当满足$\sum_{v \in S}\mathcal{CF}(v) \geq \mathcal{B} - \varepsilon$时,我们会终止初始结点的寻找过程。


\subsubsection{相关算法}
目前的工作主要是在单位化价值估值函数下的影响力最大化,故而对于所有现有的算法,我们都对其进行的相关的调整,在满足原有方法的策略时使其满足定义\ref{def:chap3:ncc-im}中的相关条件限制,主要的算法详细参见下表,

\begin{table}[htbp]
\centering
\begin{minipage}[t]{0.8\linewidth}
	\caption{相关影响力最大化算法}
	\label{tab:chap3:ncc-im-algs}
	\begin{tabular}{*{3}{p{.33\textwidth}}}
		\toprule[1.5pt]
		算法 & 复杂度 & 描述  \\ 
		\midrule[1pt]
		$Greedy$ & $O(KnRm)$ & 详见\ref{sec:chap2-greedy-alg}节 \\
		$Random$ & $O(K)$ & 详见\ref{sec:chap2-random-alg}节 \\
		$DegreeHeuristic$ & $O(nlog(n))$ & 详见\ref{sec:chap2-heuristic-alg}节 \\
		$DegreeDiscount$ & $O(Klog(n) + m)$ & 详见\ref{sec:chap2-heuristic-alg}节 \\
		$SingleDiscount$ & $O(Klog(n) + m)$ & 详见\ref{sec:chap2-heuristic-alg}节 \\
		\bottomrule[1.5pt]
	\end{tabular}
\end{minipage}
\end{table}


注意以上表\ref{tab:chap3:ncc-im-algs}中所有算法都在其原有的基础上进行了改动,使其满足$NCC-IM$的定义。


\section{算法设计与分析}
解决影响力最大化问题是NP-Hard的,所有基于General Greedy算法(见算法\ref{alg:chap3:general-greedy})改进的贪心算法虽然在效率有所提高,但是还是有很高的复杂度。然而拥有极低复杂度的随机算法在效果上没有保证。在文章\cite{barabasi1999emergence}\cite{adamic2000power}\cite{watts1998collective}中,研究者发现很多网络符合无标度网络的特点,也就是大量的网络结点是很稀疏地被连接,而有少部分的结点有则着很稠密的连接关系网。据此,我们无标度网络、小世界网络的特点,设计了我们的算法。

\begin{algorithm}
\caption{贪心算法:计算初始结点集$S$}
\label{alg:chap3:general-greedy}
\begin{algorithmic}
\REQUIRE 图$G=(V,E)$; 初始结点集大小$K$, 算法迭代次数$R$
\ENSURE 得到的初始结点集$S$,并且$S$的元素个数为$K$
\STATE $S \leftarrow \emptyset; i \leftarrow 0;$
\WHILE {$i < K$}
	\FORALL {$v \in V \setminus S$}
		\STATE $s_{v} \leftarrow 0;$
		\FOR {$j = 0 \to R$}
			\STATE $s_{v} \leftarrow s_{v} + \sigma(S \cup \{v\})$
		\ENDFOR
		\STATE $s_{v} \leftarrow s_{v}/R$
	\ENDFOR
	\STATE $S \leftarrow S \cup \argmax_{v \in V \setminus S}(s_{v})$
	\STATE $i \leftarrow i + 1$
\ENDWHILE
\RETURN $S$
\end{algorithmic}
\end{algorithm}


\subsection{BRMDN算法}
受到随机算法以及网络的连接特点,以及新提出的价值估值模型PRBC,我们提出了预算范围内随机最大度邻居算法BRMDN(Budgeted Random Maximal Degree Neighbor)。在算法执行过程中,首先随机选择一个结点,然后根据这个结点的邻居结点集合,对这个局部结合中每个结点按照度数排序,选择满足预算$\mathcal{B}$的限制条件顶点加入初始结点集$S$,直到满足定义\ref{def:chap3:ncc-im}中的限制。


\begin{algorithm}
\caption{MDN:计算结点$u$最大度数的邻居结点}
\label{alg:chap3:mdn-alg}
\begin{algorithmic}
\REQUIRE 图$G=(V,E)$; 结点$u$, 互斥集合$\mathcal{ES}$(Exclusive Set)
\ENSURE 要加入初始结点集的候选结点结点$v_{max}$
\STATE $v_{max} \leftarrow u; v_{degree} \leftarrow \mathcal{D}(u);$
\FORALL {$nbr < \mathcal{N}(u)$}
	\IF {$nbr \notin \mathcal{ES} \wedge v_{degree} < \mathcal{D}(nbr)$}
		\STATE $v_{degree} \leftarrow \mathcal{D}(nbr);$
		\STATE $v_{max} \leftarrow nbr;$
	\ENDIF
\ENDFOR
\RETURN $v_{max}$
\end{algorithmic}
\end{algorithm}


\begin{algorithm}
\caption{BRMDN:计算初始结点集$S$}
\label{alg:chap3:brmnd-alg}
\begin{algorithmic}
\REQUIRE 图$G=(V,E)$; 初始结点集大小$K$, 预算$\mathcal{B}$;预算可接受误差值$\varepsilon$;价值估值函数$\mathcal{CF}(\cdot)$;
\ENSURE 得到的初始结点集$S$;且满足$|S| \leq K \wedge \mathcal{B} - \varepsilon \leq \sum_{v \in S}\mathcal{CF}(v) \leq \mathcal{B} + \varepsilon$;
\STATE $S \leftarrow \emptyset; i \leftarrow 0; totalcost \leftarrow 0;$
\WHILE {$i < K$}
	\STATE 随机选择一个结点 $u \in V \setminus S;$
	\STATE 选择$u_{max} \leftarrow MDN(G, u, S);$
	\IF {$\mathcal{CF}(u_{max}) + totalcost \leq \mathcal{B} + \varepsilon$}
		\STATE $S \leftarrow S \cup \{u_{max}\};$
		\STATE $totalcost \leftarrow \mathcal{CF}(u_{max}) + totalcost;$
		\STATE $i \leftarrow i + 1;$
		\IF {$totalcost \geq \mathcal{B} - \varepsilon$}
			\STATE $ i \leftarrow K + 1;$
		\ENDIF
	\ENDIF
\ENDWHILE
\RETURN $S$
\end{algorithmic}
\end{algorithm}


注意在算法\ref{alg:chap3:brmnd-alg}中利用到算法\ref{alg:chap3:mdn-alg},而在算法\ref{alg:chap3:mdn-alg}中我们可以发现,对于随机选择的一个节点,我们只需要了解其邻居结点的连接状态就可以,而不需要获得整个图的连接状态进行计算,这样就可以很大地降低算法的复杂性,从而提升算法的效率。


\subsection{算法可行性分析}
\label{sec:chap3:feasibility-analysis}
在现实生活中无标度网络是很常见的一种网络状态,所以本文在无标度网络下分析算法\ref{alg:chap3:brmnd-alg}的可行性。在无标度网络中,结点度数为$k$的概率为$p(k)=ck^{-\gamma}$,定义$k_{max}$为网络中结点的最大度数,同理定义$k_{min}$为网络中结点的最小度数,那么根据概率的意义,可以有如下式子\ref{eq:chap3:max-infty}和式子\ref{eq:chap3:min-infty}成立,
\begin{equation}
\label{eq:chap3:max-infty}
\int_{k_{max}}^{+\infty}p(k) dk = \frac{1}{n}
\end{equation}

\begin{equation}
\label{eq:chap3:min-infty}
\int_{k_{min}}^{+\infty}p(k) dk = 1
\end{equation}

求解等式\ref{eq:chap3:max-infty}和等式\ref{eq:chap3:min-infty}可以得到$k_{max} = k_{min}n^{\frac{1}{\gamma-1}}$。对于任意结点$u$,由它连接出去的任意边$e=(u, \cdot) \in E$,那么$u$能以多大的概率连接到一个网络图的中心结点(称为图的hub结点,例如该结点的度数属于整个网络中结点的$Top-K$)。定义$p_{Top-K}$为结点$u$能连接到中心结点的概率,那么可以得到式子\ref{eq:chap3:p-top-k},
\begin{equation}
\label{eq:chap3:p-top-k}
p_{Top-K} = \int_{k_{Top-K}}^{k_{max}}p(k)dk = \frac{k_{max}^{2-\gamma} - k_{Top-K}^{2-\gamma}}{k_{max}^{2-\gamma} - k_{min}^{2-\gamma}}
\end{equation}
如果初始结点结合$S$的大小为$K$,那么算法\ref{alg:chap3:brmnd-alg}至少能获得一个中心结点(hub结点)的概率$p_{hub}$可以表示为
\begin{equation}
\label{eq:chap3:p-hub}
p_{hub} = 1 - (1-p_{Top-K})^{K} - \epsilon
\end{equation}


上面式子\ref{eq:chap3:p-hub}中的$\epsilon$是在预算$\mathcal{B}$控制下的误差,可能使得初始结点的大小没有到$K$从而影响到概率$p_{hub}$。进一步说,假设$\zeta$是图中估值最大的结点的价格数值,也即$\zeta = max \{\mathcal{CF}(u) | u \in V\}$,那么当预算满足条件$\mathcal{B} \geq K\zeta$时,则概率误差$\epsilon = 0$。针对式子\ref{eq:chap3:p-hub}中的参数$K$,如果我们选择的初始集合的大小足够大(例如$K=30$),那么就可以使得$(1-p_{Top-K})^{K} \leftarrow 0$,从而有$p_{hub} \approx 1 - \epsilon$。对于算法BRMDN中的概率误差,我们可以进行多次迭代从而让$\epsilon$进一步降低,也就是使得$p_{hub}$接近于1,这就说明算法能以很大概率选择到比较好的结点,同时不会陷入一种局部最优的情况,即富人俱乐部现象\cite{zhou2004rich},该现象是指结点度数大的结点通常互联在一起,出现度数大的结点扎堆现象。而算法\ref{alg:chap3:brmnd-alg}中的随机选择过程能很好的避免这一情况。


\subsection{时间复杂度分析}
根据算法\ref{alg:chap3:brmnd-alg},在随机选择一个结点后,我们需要遍历其邻居结点从而找到度数最大并且价格数值合适的那个结点,所以我们首先计算图中任意结点的邻居结点数的平均值$\bar{k}$,由于已知无标度网络中的度数与概率的关系,那么可得,
\begin{equation}
\bar{k} = \sum_{1}^{n} kp(k) = \sum_{1}^{n}kck^{-\gamma} = c\sum_{1}^{n}\frac{1}{k^{\gamma-1}},p(k)=ck^{-\gamma}
\end{equation}


\begin{lemma}
\label{lemma:noexist-gamma-le2}
给定网络图$G=(V, E)$,并且该网络满足无标度网络的性质,那么如果图$G$中没有自环,或者两个顶点之间不存在多条边,那么不存在这样的一个图$G$,使得其满足$1 < \gamma < 2$。
\end{lemma}
\begin{proof}
在章节\ref{sec:chap3:feasibility-analysis}中,我们已经得到$k_{max} = k_{min}n^{\frac{1}{\gamma-1}}$。现在假设存在图$G$满足无标度网络性质,并且有$1 < \gamma < 2$, 那么可以得到$0 < \gamma-1 < 1$,从而$n^{\frac{1}{\gamma-1}} > n$,进一步可以知道$k_{max} = k_{min}n^{\frac{1}{\gamma-1}} > n$,这就意味着一个结点的度数比图中结点的数量还要多,由于引理条件中已知图没有自环,并且两点顶点中不存在多条边,那么可以知道该假设与已知矛盾。证毕。
\end{proof}

\section{实验结果与分析}
\subsection{实验仿真模型}
我们采用SIR模型\cite{kermack1927contribution,diekmann1990definition,hethcote2000mathematics,bernoulli2004attempt}来模拟仿真传播过程,并将其结果与我们的模型实验结果作比较。
SIR模型中有三种状态,易感染状态S,感染状态I,免疫状态R。当个体处于感染状态时,以$\delta$的概率感染处于易感染状态邻居个体,感染状态的节点以$\gamma$的概率恢复为免疫状态。具体内容可以参照~\ref{cha:secondChap02}章节介绍。
如果$\delta$值较大,节点传播能力很强,节点将很快感染整个网络,从而很难区分单个个体的重要性;较小的$\delta$能在有限时间内更好的显示出感染范围。我们中我们设定较小的$\delta =0.04$。
\subsection{实验数据及环境}
考虑到不同的社会网络类型所代表不同的网络拓扑结构特性,我们选取了4个真实社会网络数据集进行分析比较,表~\ref{tab:chap3:datsetTable}给出各个网络的属性特征。我们的模型也可应用于其他类型的复杂网络。

本文所有实验运行的硬件环境是:处理器Intel® Core™ i5 CPU M430 @2。27GHz,内存(RAM)3GB。

\begin{table}[htbp]
%	\centering
	\begin{minipage}[t]{0.8\linewidth}
		\caption{现实社会网络的属性统计情况}
		\label{tab:chap3:datsetTable}
		%\begin{tabular}{*{7}{p{.14\textwidth}}}
		\begin{tabular}{*{8}{p{.11\textwidth}}}
			\toprule[1.5pt]
			Networks & {$n$} & {$m$} & {$<k>$} & {$max(k)$} & {$d$} & {$max(C_{ks})$} & {$C$} \\ 
			\midrule[1pt]
			Blogs & 3982 & 6803 & 3.42 & 189 & 6.227 & 7 & 47 \\
			Netscience & 379 & 914 & 4.82 & 34 & 6.061 & 8 & 19 \\
			Router & 5022 & 6258 & 2.49 & 106 & 6.393 & 7 & 75 \\
			Email & 1133 & 5451 & 9.62 & 71 & 3.716 & 11 & 10 \\
			\bottomrule[1.5pt]
		\end{tabular}
	\end{minipage}
\end{table}
其中$n$是节点数,$m$为边数,$<k>$表示网络中平均度数据,$max(k)$为节点中最大度数,$d$为节点之间最短路径的平均数,$max(C_{ks})$ 为网络中最大的K-shell,$C$为整个复杂网络被划分的社区数。
\subsection{实验效果}
我们应用SIR模型,对我们所提出的KSC指标与度、紧密性、介数和K-shell中心化指标进行了分析比较。实验模拟传播过程中,每次只选取网络中的一个节点作为初始传播节点,设定较短传播时间($t=10$),对每个节点进行1000次重复实验取均值,最终感染与恢复的节点总数定义为影响力$F(t)$。
\begin{figure}[H] 
	\centering
	\includegraphics[scale=1]{./chap3/chap03alphabeta}
	\caption{KSC模型中内部影响因子$\alpha$与外部影响因子$\beta$取不同对传播影响力指标的影响分析}
	\label{fig:chap03alphabeta}
\end{figure}
图~\ref{fig:chap03alphabeta}表示了在KSC模型中选取不同的内部影响因子$\alpha$与外部影响因子$\beta (\alpha +\beta =1)$对结果的影响程度。哲学范畴中事物的运动和变化,是由它本身固有的内部矛盾引起的,又是同它所处的一定的外在条件相联系的,内因和外因在事物发展中的地位和作用是不同的。这与模型中内部影响因子$\alpha$与外部影响因子$\beta$对传播影响假设完全一致。如在图~\ref{fig:chap03alphabeta}-c中局部放大了最具影响力的前10位,当$0\le \alpha \le 0.2$,$1\geq \beta \geq 0.8$,或$1\geq \alpha \geq 0.8$,$0\le \beta \le 0.2$,时可以看到,影响力曲线波动较大,影响力排名靠后的节点影响反而大。即只是单纯的考虑内部属性或外部属性对发现最具有影响的节点来说准确性,适用性较差。
经实验分析得出四种不同拓扑结构网络的内、外影响因子经验取值情况,如表~\ref{tab:chap3alphabeta}所示。可以看出在不同类型的网络结构中,内、外因所起的权重是不同的,我们实验中的$\alpha$、$\beta$根据表~\ref{tab:chap3alphabeta}中取较好的经验值,这从另一方面证明了所提出模型的合理性及普遍性。
\begin{table}[htbp]
%		\centering
	\begin{minipage}[t]{0.8\linewidth}		
		\caption{内、外影响因子取值情况}
		\label{tab:chap3alphabeta}
		%\begin{tabular}{*{7}{p{.14\textwidth}}}
		\begin{tabular}{*{4}{p{.26\textwidth}}}
			\toprule[1.5pt]
			网络名称 & \multicolumn{1}{c}{内部影响因子} & \multicolumn{1}{c}{外部影响因子} & \multicolumn{1}{c}{较好的经验取值}   \\
			& \quad 取值范围$(\alpha)$ & \quad 取值范围$(\beta)$ & \qquad $\alpha $ \qquad $\beta$  \\ 
			
			\midrule[1pt]
			Blogs & \qquad 0.1-0.4 & \qquad 0.9-0.6 & \quad 0.2 \qquad  0.8 \\
			Netscience & \qquad 0.3-0.7 & \qquad 0.7-0.3 & \quad 0.4 \qquad  0.6 \\
			Router & \qquad 0.3-0.8 & \qquad 0.7-0.2 & \quad 0.7 \qquad  0.3 \\
			Email & \qquad 0.1-0.8 & \qquad 0.9-0.2 & \quad 0.5 \qquad  0.5 \\
			\bottomrule[1.5pt]
		\end{tabular}
	\end{minipage}
\end{table}
图3所表示的是在不同的网络中,中心度指标与实际影响力之间的关系图。
\begin{figure}[H]
	\centering%
	\subcaptionbox{\label{fig:chap03graphTblogs1}}
	{\includegraphics[scale=0.3]{./chap3/chap03graphTblogs1}}
	%\hspace{3em}%
	\subcaptionbox{\label{fig:chap03graphTblogs2}}
	{\includegraphics[scale=0.3]{./chap3/chap03graphTblogs2}}
	%\hspace{3em}%
	\subcaptionbox{\label{fig:chap03graphTblogs3}}
	{\includegraphics[scale=0.3]{./chap3/chap03graphTblogs3}}
	%\hspace{3em}%
	\subcaptionbox{\label{fig:chap03graphTblogs4}}
	{\includegraphics[scale=0.3]{./chap3/chap03graphTblogs4}}
	%\hspace{3em}%
	\subcaptionbox{\label{fig:chap03graphTblogs5}}
	{\includegraphics[scale=0.3]{./chap3/chap03graphTblogs5}}
	%\hspace{3em}%
	\subcaptionbox{\label{fig:chap03graphTemail1}}
	{\includegraphics[scale=0.3]{./chap3/chap03graphTemail1}}
		
\end{figure}

\addtocounter{figure}{-1}       %先欺骗LaTeX图形计数器
\begin{figure}[H]
	\addtocounter{figure}{1}
	\centering%
	\subcaptionbox{\label{fig:chap03graphTemail2}}
	{\includegraphics[scale=0.3]{./chap3/chap03graphTemail2}}
	%\hspace{3em}%
	\subcaptionbox{\label{fig:chap03graphTemail3}}
	{\includegraphics[scale=0.3]{./chap3/chap03graphTemail3}}
	\subcaptionbox{\label{fig:chap03graphTemail4}}
	{\includegraphics[scale=0.3]{./chap3/chap03graphTemail4}}
	%\hspace{3em}%
	\subcaptionbox{\label{fig:chap03graphTemail5}}
	{\includegraphics[scale=0.3]{./chap3/chap03graphTemail5}}
	%\hspace{3em}%
	\subcaptionbox{\label{fig:chap03graphTnetscience1}}
	{\includegraphics[scale=0.3]{./chap3/chap03graphTnetscience1}}
	%\hspace{3em}%
	\subcaptionbox{\label{fig:chap03graphTnetscience2}}
	{\includegraphics[scale=0.3]{./chap3/chap03graphTnetscience2}}
	%\hspace{3em}%
	\subcaptionbox{\label{fig:chap03graphTnetscience3}}
	{\includegraphics[scale=0.3]{./chap3/chap03graphTnetscience3}}
	%\hspace{3em}%
	\subcaptionbox{\label{fig:chap03graphTnetscience4}}
	{\includegraphics[scale=0.3]{./chap3/chap03graphTnetscience4}}	
	%\hspace{3em}%
\end{figure}

\addtocounter{figure}{-1}       %先欺骗LaTeX图形计数器
\begin{figure}[H]
	\addtocounter{figure}{1}
	\centering%
	\subcaptionbox{\label{fig:chap03graphTnetscience5}}
	{\includegraphics[scale=0.3]{./chap3/chap03graphTnetscience5}}
	%\hspace{3em}%
	\subcaptionbox{\label{fig:chap03graphTrouter1}}
	{\includegraphics[scale=0.3]{./chap3/chap03graphTrouter1}}	
	%\hspace{3em}%
	\subcaptionbox{\label{fig:chap03graphTrouter2}}
	{\includegraphics[scale=0.3]{./chap3/chap03graphTrouter2}}
	%\hspace{3em}%
	\subcaptionbox{\label{fig:chap03graphTrouter3}}
	{\includegraphics[scale=0.3]{./chap3/chap03graphTrouter3}}
	%\hspace{3em}%
	\subcaptionbox{\label{fig:chap03graphTrouter4}}
	{\includegraphics[scale=0.3]{./chap3/chap03graphTrouter4}}
	%\hspace{3em}%
	\subcaptionbox{\label{fig:chap03graphTrouter5}}
	{\includegraphics[scale=0.3]{./chap3/chap03graphTrouter5}}
	\caption{节点影响力分析比对,5种不同中心化指标,4种不同的复杂网络环境。传播影响力$F(t)(t=10)$,每个初始节点重复运行1000次取均值。}
	\label{fig:chap03ftall}
\end{figure}
在Blogs网络中,度、K-shell、介数指标与影响力$F(t)$关系都不是很明显,例如在介数指标结果中,一些介数大的节点$F(t)$比一些介数很小的节点影响力还要低,但KSC指标可以较好地区分出最具影响力的节点。Email网络中,所有指标表现都较好,紧密度与KSC更为理想,这与网络的拓扑结构有关系。在Netscience网络中,介数衡量结果最差,紧密度、K-shell与节点$F(t)$影响力关系也不明显,KSC的效果要优于度数。在路由网络(Router)中K-shell与KSC结果接近,其它指标结果很难得到传播力较大的节点。总之,在不同的网络中传统的中心度指标各有优缺点,但是我们提出的KSC整体效果最好,都能有效的区分出最具有影响力的节点,通用性较强。
对于单个传播源的情形,最具有影响力的节点并非度数最大的节点或者介数最大的节点,而是K-shell最大的节点,这一结论在Kitsak\cite{kitsak2010identification}在《Nature Physics》指出。图~\ref{fig:chap03KSCKshell}比较了KSC与K-shell的实验效果,KSC比K-shell增加了外部属性的影响因素。例如在Email网络中,当K-shell为10时,传播影响力$F(t)(t=10)$并不稳定,而且变化范围较大,并且影响力随着KSC值的增大呈现增大趋势。在4种复杂网络中明显可以看出当KSC一定时$F(t)$相对稳定,颜色变化范围较小。
\begin{figure}[H] 
	\centering
	\includegraphics[scale=0.8]{./chap3/chap03KSCKshell}
	\caption{KSC与K-shell指标在4种不同复杂网络中的分析比较,横轴表示KSC的指标值,纵轴表示K-shell的指标值,颜色坐标代表SIR模型仿真出的实际影响力大小$F(t)$。}
	\label{fig:chap03KSCKshell}
\end{figure}
图~\ref{fig:chap03KSCKbetweenness}比较了KSC与介数的实验效果。可出看出在Email网络中,KSC与介数表现正向同步较好,都能较好地发出最具影响力的节点,但在其它三个网络环境中KSC表现明显优于介数指标。KSC与其它度数、紧密度指标比较省略。实验结果表明KSC方法明显优于已有4种典型方法。由此可见,考虑节点的外部属性对于找出最有影响力的节点意义重大,且具有较强的通用性,说明验证了我们所提出模型的合理性与普遍性。
\begin{figure}[H] 
	\centering
	\includegraphics[scale=0.8]{./chap3/chap03KSCKbetweenness}
	\caption{KSC与介数指标在4种不同复杂网络中的分析比较,横轴表示KSC的指标值,纵轴表示K-shell的指标值,颜色坐标代表SIR模型仿真出的实际影响力大小$F(t)$。}
	\label{fig:chap03KSCKbetweenness}
\end{figure}
如图~\ref{fig:chap03Finalresult}所示,在Blogs图中,KSC曲线的某一点$(x,y)$,横坐标表示此点用KSC指标值得出的排名为$x$,纵坐标表示此点在SIR仿真的实际影响力为$y$。理论上,一个好的指标方法应该表现为向右倾斜单调下降的曲线。因为对于一个好的指标方法,如果某个节点按该指标得出的排名越靠后,其实际影响力 应该随之减小。
\begin{figure}[H] 
	\centering
	\includegraphics[scale=1.0]{./chap3/chap03Finalresult}
	\caption{5种指标的影响力$F(t)(t=10)$前Top-K的均值结果,横轴代表按各种指标值的排序,纵轴表示影响力$F(t)$,不同颜色的曲线代表不同的方法。}
	\label{fig:chap03Finalresult}
\end{figure}
从图~\ref{fig:chap03Finalresult}结果中分析,我们可以看到已有的4种典型方法在不同网络中影响力均有所波动,这些方法在不同网络中各有优劣,通用性不强,特别是在影响力前10中波动范围最为明显。指标值小的传播影响力反而比指标值大的传播影响力要大,特别是在Blogs网络中,已有的4种指标方法结果都不太精确,而我们提出的KSC指标几乎在所有的网络中都符合向右倾斜单调下降的理论曲线。
本实验充分论证了节点影响力不但由内部属性决定,而且还与其外部属性密切相关,说明我们提出的KSC模型的合理性及普遍适用性。

\section{本章小结}
在复杂网络中发现最具影响力的节点,可以帮助我们提高新知识、新产品的传播效率,同时可以有效地制定相应的策略来阻止疾病及谣言传播。复杂网络中鉴别最具影响力的节点一直以来是研究的热点与难点,我们提出了KSC中心化指标模型。实验选取现实生活中常见的四种复杂网络,博客网、邮件网、路由网和科学合著网,通过SIR模型模拟节点传播过程,并对各种中心化指标得出的影响力进行排序分析,验证了方法的高效性和正确性。

与传统的度、紧密度、介数和K-shell方法相比,传统方法在不同网络中各有优劣,通用性不强。而我们提出的KSC模型综合考虑了节点的内部属性与外部属性,实验证明用此方法鉴别节点的影响力要精确得多,适用范围更大。我们所提出的模型为这项具有挑战性研究提供了新的思想和方法,希望对以后的研究者给予启发。

