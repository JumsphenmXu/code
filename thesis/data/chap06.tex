% 下面这句用以支持中文
% !Mode:: "TeX:UTF-8"

%%% Local Variables:
%%% mode: latex
%%% TeX-master: t
%%% End:

\chapter{总结与展望}
\label{cha:5thChap06}
受哲学范畴“事物的变化是由内部因素与外部因素同时作用决定的”的思想启发,我们考虑了信息传播的内、外影响因素,并从宏观、中观和微观三个不同层次角度对信息传播进行了分析建模,并给出了相应的实验结果与分析。希望本文研究工作能为此领域的研究人员带来帮助和启示。
\section{对最有影响力节点发现的总结与展望}
我们提出的KSC最有影响力节点发现方法,从内部因素方面我们分析了节点的度数指标、介数指标、紧密度指标、K-shell指标等,在外部因素方面我们重点分析了节点所属社区的网络环境。并由实验分析给出了内、外影响因子$\alpha$、$\beta$在博客网、邮件网、路由网和科学合著网中的经验值。通过SIR模型对传播过程进行仿真,并对各种中心化指标得出的影响力进行排序分析,KSC指标几乎在所有的网络中都符合向右倾斜单调下降的理论曲线,实验证明用此方法发现节点的影响力要精确得多,适用范围更大。

我们工作中提出的影响力节点由内、外因素决定,这一想法新颖,希望能给以后的研究者给予启示。同时,本文也对该研究留下了几个新的问题以供未来继续研究探索。
\begin{enumerate}[(1)]
	\item 能否从复杂网络度分布的幂指数$\gamma$特征,总结出随机网络模型、小世界网络模型及无标度网络模型中相应的规律及经验值。
	\item 是否能在分析影响指标时不只是从单一角色分析,如一个节点可能既是度数较大的,又是网络的核心。从而探索一种新的模型能综合考虑节点在整个网络中所担负的角色重要性,等等。这些问题都可以被继续研究探索。
\end{enumerate}

\section{对影响力最大化问题的总结与展望}
影响力最大化问题是在一定限制条件下的多个影响力节点组合优化问题,也是NP-hard问题,其研究具有很强的商业及现实意义。我们提出的RDMN算法、RDMN++算法与已有经典算法实验运行结果相近,但是时间复杂度的优势提升显著。在只知道随机选择节点及其直接连接邻居节点的局部信息,就能发现传播源种子节点,从而巧妙地避开了必须了解全局节点信息的问题。同时我们根据复杂网络结构中度分布幂指数$\gamma$,从理论上推导分析了该方法选择传播源种子节点是度数最大的Top-k的概率情况。我们提出的算法实际应用极其简单,可行性、适用性更强。

在本文我们只是重点讨论了无标度网络模型中$2<\gamma <3$的情况,但是方法同样适用于其它模型的网络结构。同时,本文也对影响力最大化问题研究留下了几个新的问题以供未来继续研究探索。
\begin{enumerate}[(1)]
	\item 讨论RDMN算法在随机网络模型、小世界网络模型下的运行情况,同样可以从理论方面应用本文推导方式进行理论分析。并利用现实网络数据进行实验,得出相应的规律。
	\item 可以考虑时间因素,如何在有限时间内使得影响力最大化。
	\item 可以考虑在实际传播预算投入与产出的传播效果问题,得出影响力最大化的投入与效益的关系规律。
	\item 可以考虑不同商家相似产品影响力最大化的市场博弈关系,如何根据对手选择传播源种子节点进行商业布局,使其产品影响力最大化。
\end{enumerate}

\section{对信息传播结构多样化的总结与展望}
我们知道信息传播是个相当复杂的传播演变过程,只是简单地应用传统知识是无法解释信息传播的原理的,其目前是各领域学科研究的热点问题。我们分析了信息传播的时间变化特点及接受信息节点的内、外影响因素,给出了信息传播机理,提出了信息传播的5个假设,假设中不但考虑了信息传播中个体的内部属性特点,还特别考虑了个体接受信息的Ego网络的结构多样性影响,以及外界整个大信息环境这些外部影响因素。根据接收信息个体周围邻居节点的信息确认状态,按照传播信息状态邻居的关系紧密程度从连通性,无重叠社区Community,K-truesses子图,有重叠社区Kclique子图等4种结构特点进行分析。提出了信息传播结构多样化模型-ISSD模型,并给出了信息传播过程的形式化定义和描述。实验分析了信息传播过程中时间影响、结构多样化影响特点。模型使人们分析信息传播过程有了清晰的认识,并可以定量化分析,这也方便了信息传播过程的理论研究。
本文也对信息传播结构多样化模型研究留下了几个新的问题以供未来继续研究探索。
\begin{enumerate}[(1)]
	\item 根据不同用户自身喜好,从微观上得出符合每个用户个性化的信息传播规律。
	\item 分析用户对不同种类的信息(如政治新闻、体育新闻、娱乐新闻等)的传播规律。
	\item 研究用同用户之间的关系紧密程度(如闺蜜、好朋友、朋友)之间对信息传播可信度分析。
	\item 研究不同社区划分对信息传播范围的宏观预测。
	\item 研究不同粒度的信息传播监控问题,使人们在享受信息共享的同时,防止被别有用心的人利用。
\end{enumerate}