% 下面这句用以支持中文
% !Mode:: "TeX:UTF-8"

%%% Local Variables:
%%% mode: latex
%%% TeX-master: t
%%% End:

\chapter{总结与展望}
\label{cha:5thChap}

\section{论文总结}
互联网技术日异月新,发展非常之快,目前基于社交网络的应用得到了很好的开发。人们在生活中逐渐地与社交网络联系在一起,可以分享,传播,转发各种各样的关于旅游,关于食品,关于政治等等一系列的事情。互联网深入生活给我们带来了很多的便利,我们可以在任何地方得知世界上正在发生的任何新闻。

从大数据分析中获得有用信息是一个十分活跃的研究热点,而现在网络中社交网络平台海量的用户发布海量的信息可以为我们进行大数据研究提供海量的数据基础。影响力分析在近十年来一直被广泛研究,从传播模型,分析算法,社交网络图等等各方面都有很多学者进行新的研究或者对现有的成果进行改进。下面分以下几个方面对本文进行总结:
\begin{itemize}
\item 我们对社交网络平台进行了相应的调研,对影响力最大化相关课题也进行了深入的了解,掌握了目前主要的研究方法和经典的模型,分析讨论了目前研究存在的问题,并根据问题提出解决方案。
\item 通过分析,我们了解到目前一个很重要的但是没有得到很好的解决研究问题是对网络图中结点进行价值估计,现有的研究中基本是以单位价值来进行研究的,虽然也有其他价值估计的方法(如随机赋值),但都缺乏说服力。并且造成构建顶点估值模型困难的以个原因是很难获得整个网络中的所有结点现实生活中的实际价值,从而从理论上去分析创建合理并可行的算法模型。本文在这一方面主要基于PageRank算法的思路,认为一个重要的结点必然其价值越高,所以提出了PRBC(PageRank Based Cost Model)价值估计模型,并利用PRBC估值模型提出了有预算的影响力最大化算法BRMDN。
\item 进一步我们发现目前研究主要还是集中在单一信源的影响力最大化分析,很多的研究者在此方面都作了大量的工作,也提出了很多比较经典的传播模型(如IC,LT等模型),影响力最大化算法(如Greedy, DegreeDiscount,CELF等算法),但是只有部分的研究涉及多信源的影响力最大化分析。所以本文另一方面的贡献在于提出了竞争环境下的影响力分析一种可行的传播模型XLT4C(eXtended Linear Threshold for Competition),以及基于该模型的算法LG4C,LDH4C。
\item 对于我们提出的模型以及算法,我们在真实的数据集上进行了大量实验,实验结果表明我们所提出的传播模型,顶点估值模型,影响力最大化分析算法都获得了很好的效果。
\end{itemize}

\section{未来研究}
首先,社会是个互相竞争的舞台,然后我们在存在竞争的影响力最大化分析上做的研究还非常有限,其次在缺乏具体的价值数据情况下对顶点估值模型进行准确化也是一个比较有挑战的问题。

所以基于以上的客观事实,并且研究的时间有限很多问题还没有进行优化,所以我们未来可以在本文的基础上对竞争型影响力分析上提出更好的传播模型,并用数学方法严谨地给出相应成立的条件,以及希望能给出一些比较重要的结论或者一些经验性结果。同样我们可以多从网络中发掘有用关于价值估算的数据,并对其进行科学的分析,然后可以得出一些有实际利用价值的模型和方案。总的来说,未来研究就主要集中于两点:(1)进一步研究竞争环境下的影响力最大化;(2)对网络价值估算的模型进行优化。